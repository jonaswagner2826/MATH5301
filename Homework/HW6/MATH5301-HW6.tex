\documentclass[]{article}

\usepackage{graphicx}

\usepackage[margin=1in]{geometry}

\setlength\parindent{0pt}

\usepackage{physics}
\usepackage{amsmath, amsfonts, amssymb, amsthm}

\usepackage{listings}

\usepackage{enumitem}
\renewcommand{\theenumi}{\alph{enumi}}
\renewcommand*{\thesection}{Problem \arabic{section}}
\renewcommand*{\thesubsection}{\alph{subsection})}
\renewcommand*{\thesubsubsection}{\quad \quad \roman{subsubsection})}

%Custom Commands
\newcommand{\Rel}{\mathcal{R}}
\newcommand{\R}{\mathbb{R}}
\newcommand{\C}{\mathbb{C}}
\newcommand{\N}{\mathbb{N}}
\newcommand{\Z}{\mathbb{Z}}
\newcommand{\Q}{\mathbb{Q}}

\newcommand{\toI}{\xrightarrow{\textsf{\tiny I}}}
\newcommand{\toS}{\xrightarrow{\textsf{\tiny S}}}
\newcommand{\toB}{\xrightarrow{\textsf{\tiny B}}}

\newcommand{\divisible}{ \ \vdots \ }
\newcommand{\st}{\ : \ }


% Theorem Definition
\newtheorem{definition}{Definition}
\newtheorem{assumption}{Assumption}
\newtheorem{theorem}{Theorem}
\newtheorem{lemma}{Lemma}


%opening

\title{MATH 5301 Elementary Analysis - Homework 6}

\author{Jonas Wagner}

\date{2021, October 17\textsuperscript{th}}

\begin{document}

\maketitle

% Problem 1
\section{}
% Part a
\subsection{Show that $\forall_{x > 0} \ x\in \R \implies \lim_{n\to \infty} x^{1/n} = 1$}

\begin{definition}
    for $f : (S_1, d_1) \to (S_2,d_2)$, \\
    if $a \in \bar{S}_1$ then we say $$\lim_{x\to a} f(x) = b$$ if $$\forall_{\epsilon>0} \exists_{\delta(\epsilon) > 0} \st \forall_{x \in \dot{B}_\delta(a) \subset S_1} \implies f(x) \in B_{\epsilon}(b) \subset S_2$$
\end{definition}

\begin{theorem}
    $$\forall_{x > 0} \ x\in \R \implies \lim_{n\to \infty} x^{1/n} = 1$$
    \begin{proof}
        \begin{align*}
            \forall_{x > 0} \ x\in \R 
                &\implies \lim_{n\to \infty} f(x) = x^{1/n} = 1 &\\
            &\implies \forall_{\epsilon>0} \exists_{\delta(\epsilon) > 0} \st \forall_{x \in \dot{B}_\delta(a)} \implies f(x) \in B_{\epsilon}(b)\\
            &\implies \forall_{\epsilon>0} \exists_{\delta(\epsilon) > 0} \st \forall_{x\in \R} : d_1 (a, x) < \delta(\epsilon) \implies x^{1/n} \in \R : d_2(b,x^{1/n}) < \epsilon
        \end{align*}
    \end{proof}
\end{theorem}















% Problem 2
\newpage
\section{True or False?}

\begin{definition}
    for $f: (S_1, d_1) \to (S_2, d_2)$, $f(x)$ is a continuous function iff
    $$\forall_{x\in S_1} \forall_{\epsilon>0} \exists_{\delta(x,\epsilon)>0} \forall_{y\in S_1} : d_1(x,y) < \delta \implies d_2(f(x),f(y)) < \epsilon$$
\end{definition}

%Part a
\subsection{If $f: (S_1,d_1) \to (S_2,d_2)$ is continuous and $U\subset S_1$ is open then $f(U) \subset S_2$ is also open}
\begin{theorem}
    If $f: (S_1,d_1) \to (S_2,d_2)$ is continuous and $U\subset S_1$ is open then $f(U) \subset S_2$ is also open.
    \begin{proof}
        \begin{align*}
            \forall_{x\in S_1} \forall_{\epsilon>0} \exists_{\delta(x,\epsilon)>0} \forall_{y\in S_1} : d_1(x,y) < \delta &\implies d_2(f(x),f(y)) < \epsilon \ \land \\
                &\land \forall_{x \in U} \exists_{\epsilon>0} \st B_\epsilon(x) \subset U \implies\\
                &\implies \forall_{f(x) \in F(U)} \exists_{\epsilon>0} \st B_\epsilon(f(x)) \subset U\\
            \forall_{x\in S_1} \forall_{\epsilon>0} \exists_{\delta(x,\epsilon)>0} \forall_{y\in S_1} : d_1(x,y) < \delta &\implies d_2(f(x),f(y)) < \delta \land \\
                &\land \forall_{x \in U} \exists_{\epsilon_1>0} \st \forall_{y \in S_1} d(x,y) < \epsilon_1 \subset U \implies\\
                &\implies \forall_{f(x) \in F(U)} \exists_{\epsilon_2>0} \st \forall_{f(x) \in f(U)} : \forall_{f(y) \in S_2} : d_2(f(x),f(y)) < \epsilon_2 \subset U
        \end{align*}
        which is clearly true.
    \end{proof}
\end{theorem}

\subsection{$f: (S_1,d_1) \to (S_2,d_2)$ is continuous $\iff$ $\forall_{C \subset S_2}$ closed, the set $f^{-1} (C) \subset S_1$ is also closed.}
\begin{theorem}
    $f: (S_1,d_1) \to (S_2,d_2)$ is continuous $\iff$ $\forall_{C \subset S_2}$ closed, the set $f^{-1} (C) \subset S_1$ is also closed.
    \begin{proof}
        \begin{align*}
            \forall_{x\in S_1} \forall_{\epsilon>0} \exists_{\delta(x,\epsilon)>0} \forall_{y\in S_1} : d_1(x,y) < \delta &\implies d_2(f(x),f(y)) < \epsilon \iff\\
            &\iff \forall_{C\subset S_2} : \forall_{x \in C^\complement} \exists_{\epsilon>0} : B_{\epsilon}(x) \in C^\complement \implies\\
            &\implies \forall_{x \in f^{-1}(C)^\complement} \exists_{\epsilon>0} : B_{\epsilon}(x) \in f^{-1}(C)^\complement
        \end{align*}
    \end{proof}
\end{theorem}













% Problem 3
\newpage
\section{Prove the following properties of continuous functions:}

%Part a
\subsection{$\forall_{a,b\in\R} \forall_{f,g : S_1 \to S_2}$ continuous $\implies (a f + b g) (x) := a f(x) + b g(x) : S_1 \to S_2$ continuous.}
\begin{theorem}
    $\forall_{a,b\in\R} \forall_{f,g : S_1 \to S_2}$ continuous $\implies (a f + b g) (x) := a f(x) + b g(x) : S_1 \to S_2$ continuous.
    \begin{proof}
        \begin{align*}
            \forall_{a,b\in\R} \forall_{f,g : S_1 \to S_2} : &\\
                &\qty(\forall_{x\in S_1} \forall_{\epsilon>0} \exists_{\delta(x,\epsilon)>0} \forall_{y\in S_1} : d_1(x,y) < \delta \implies d_2(f(x),f(y)) < \epsilon) \land\\
                &\land \qty(\forall_{x\in S_1} \forall_{\epsilon>0} \exists_{\delta(x,\epsilon)>0} \forall_{y\in S_1} : d_1(x,y) < \delta \implies d_2(g(x),g(y)) < \epsilon) \implies\\
                &\implies \qty(\forall_{x\in S_1} \forall_{\epsilon>0} \exists_{\delta(x,\epsilon)>0} \forall_{y\in S_1} : d_1(x,y) < \delta \implies d_2(a f(x) + b g(x),a f(y) + b g(y)) < \epsilon)\\
            \forall_{a,b\in\R} \forall_{f,g : S_1 \to S_2} : &\\
                &\forall_{x\in S_1} \forall_{\epsilon_{1,2,3}>0} \exists_{\delta(x,\epsilon)>0} \forall_{y\in S_1} : d_1(x,y) < \delta \implies\\
                &\implies \qty(d_2(f(x),f(y) < \epsilon_1) \land \qty(d_2(g(x),g(y)) < \epsilon_2)) \implies\\
                    &\qquad \implies d_2(a f(x) + b g(x),a f(y) + b g(y)) < \epsilon_3
        \end{align*}
        Which by the additive and multiplicative properties of a metric space, along with the triangular inequality, this is clearly true:
        $$d_2(a f(x) + b g(x), a f(y) + b g(y)) \leq a d_2(f(x),f(y)) + b d_2(g(x),g(y))$$
        Therefore this statement is true.
    \end{proof}
\end{theorem}

%Part b
\subsection{$\forall_{f : S_1 \to S_2}$ continuous and $\forall_{h : S_1 \to \R}$ continuous $\implies (h f) (x) := h(x) \cdot f(x) : S_1 \to S_2$ continuous.}
\begin{theorem}
    $\forall_{f : S_1 \to S_2}$ continuous and $\forall_{h : S_1 \to \R}$ continuous $\implies (h f) (x) := h(x) \cdot f(x) : S_1 \to S_2$ continuous.
    \begin{proof}
        \begin{align*}
            &\forall_{f : S_1 \to S_2} : \forall_{x\in S_1} \forall_{\epsilon>0} \exists_{\delta(x,\epsilon)>0} \forall_{y\in S_1} : d_1(x,y) < \delta \implies d_2(f(x),f(y)) < \epsilon) \land\\
                &\qquad\land \forall_{h : S_1 \to \R} : \forall_{x\in S_1} \forall_{\epsilon>0} \exists_{\delta(x,\epsilon)>0} \forall_{y\in S_1} : d_1(x,y) < \delta \implies d_2(h(x),h(y)) < \epsilon) \implies\\
                &\qquad \qquad \implies \qty(\forall_{x\in S_1} \forall_{\epsilon>0} \exists_{\delta(x,\epsilon)>0} \forall_{y\in S_1} : d_1(x,y) < \delta \implies d_2(h(x) \cdot f(x),h(y) \cdot f(y)) < \epsilon)\\
            &\forall_{f : S_1 \to S_2} \land \forall_{h : S_1 \to \R} : \forall_{x\in S_1} \forall_{\epsilon_{1,2,3}>0} \exists_{\delta(x,\epsilon)>0} \forall_{y\in S_1} : d_1(x,y) < \delta \implies\\
                &\qquad \implies \qty(d_2(f(x),f(y))< \epsilon_1 \land d_2(h(x),h(y)) < \epsilon_2 \implies d_2(h(x) \cdot f(x), h(y) \cdot f(y)) < \epsilon_3)
        \end{align*}
        Which by the multiplicative properties of a metric space this is clearly true:
        $$d_2(h(x) \cdot f(x), h(y) \cdot f(y)) \leq d_2(h(x),h(y)) \cdot d_2(f(x),f(y))$$
        Therefore this statement is true.
    \end{proof}
\end{theorem}

\newpage
%Part c
\subsection{$h(x) \neq 0 \forall_{x\in S_1} \implies \frac{1}{h(x)}$ is continuous.}
Note: assuming $h : S_1 \to \R \backslash \{0\}$ and continuous.
\begin{theorem}
    $\forall_{h : S_1 \to \R \backslash \{0\}}$ continuous $\implies (h(x))^{-1} := \frac{1}{h(x)}: S_1 \to \R$ continuous.
    \begin{proof}
        \begin{align*}
            \forall_{h : S_1 \to \R \backslash \{0\}} : 
                &\forall_{x\in S_1} \forall_{\epsilon>0} \exists_{\delta(x,\epsilon)>0} \forall_{y\in S_1} : d_1(x,y) < \delta \implies d_2(h(x),h(y)) < \epsilon) \implies\\
                &\implies \qty(\forall_{x\in S_1} \forall_{\epsilon>0} \exists_{\delta(x,\epsilon)>0} \forall_{y\in S_1} : d_1(x,y) < \delta \implies d_2\qty(\frac{1}{h(x)},\frac{1}{h(y)}) < \epsilon)\\
            \forall_{h : S_1 \to \R \backslash \{0\}} :
                &\forall_{x\in S_1} \forall_{\epsilon_{1,2}>0} \exists_{\delta(x,\epsilon)>0} \forall_{y\in S_1} : d_1(x,y) < \delta \implies d_2(h(x),h(y)) < \epsilon) \implies\\
                &\implies d_2(h(x),h(y)) < \epsilon_1 \land d_2\qty(\frac{1}{h(x)},\frac{1}{h(y)})< \epsilon_2
        \end{align*}
        Within $(\R \backslash \{0\}, d_2)$, 
        $$d_2(a,b) < \epsilon_1 \implies \exists_{\epsilon_2>0} : d_2 \qty(\frac{1}{a}, \frac{1}{b}) < \epsilon_2$$
        Therefore it's true.
    \end{proof}
\end{theorem}


% Problem 4
\newpage
\section{Prove the following statement:}
If $A$ and $B$ are two closed nonempty disjoint % means A \cap B = \emptyset...
sets in the metric space (S,d) then there exists a continuous function $\mathcal{X}(x)$ such that $\mathcal{X}(x) = 0$ for all $x \in A$ and $\mathcal{X} = 1$ for all $x \in B$.
\begin{theorem}
    $$A, B \neq \emptyset \in (S,d) : A \cap B = \emptyset \implies \exists_{\mathcal{X}: S \to \{0,1\}} \textnormal{continuous} : \qty( \forall_{x \in A} \mathcal{X} = 0) \land \qty(\forall_{x \in B} \mathcal{X} = 1)$$
    \begin{proof}
        %Part a
        Define the distance from the point $x$ to the set $A$ as 
        $$\rho_A(x) := \inf_{y\in A} d(x,y)$$
        %Part b
        \begin{lemma}
            $$\rho_A(x) = 0 \iff x \in \bar{A}$$
            \begin{proof}
                
                % \begin{align*}
                %     \rho_A(x) = \forall_{a \in S}\forall_{y \in A} : a \leq d(x,y) = 0 
                %         &\iff x \in \qty{x \in S \st \forall_{\epsilon>0} B_{\epsilon}(x) \cap A \neq \emptyset}
                % \end{align*}
            \end{proof}
        \end{lemma}
    \end{proof}
\end{theorem}






% Problem 5
\newpage
\section{Which of following sets in $\R^2$ are compact?}
\begin{definition}
    Let $(S,d)$ be a metric space with $A \subset S$,
    \begin{enumerate}
        \item For $\{U_\alpha\}_{\alpha \in A}, \ U_\alpha \subset S$, is a \textbf{\underline{cover}} of the set $A$ if 
        $$A \subset \bigcup_{\alpha\in A} U_\alpha$$
        \item A cover $\{U_\alpha\}_{\alpha \in A}$ of $A$ is an \textbf{\underline{open cover}} if $\forall_{\alpha \in A}$ $U_\alpha$ is an open set.
        \item $\{V_\beta\}_{\beta\in B}$ is called a \textbf{\underline{subcover}} of $\{U_\alpha\}_{\alpha \in A}$ if
        \begin{enumerate}
            \item $\{V_\beta\}_{\beta\in B}$ is a cover of $A$
            \item $\forall_{\beta \in B} \exists_{\alpha \in A} V_\beta = U_\alpha$
        \end{enumerate}
        \item A cover with a finite number of sets is called a \textbf{\underline{finite cover}}.
    \end{enumerate}
\end{definition}
\begin{definition}
    For $A \subset (S,d)$, $A$ is \textbf{\underline{compact}} if for every open cover of $A$ there exists a finite sub cover.
\end{definition}

%Part a
\subsection{$A = \qty{(x,y) \st x^2 - y^2 \leq 1}$}
This set (a cone I believe) is not bounded and therefore no, through contradiction of the necessary (but insufficient) condition of boundedness, it is not compact.

%Part b
\subsection{$B = \qty{(x,y) \st 0 < x^2 + y^2 \leq 1}$}
This set is a disk, but because it excludes the origin it is no longer closed which is a necessary (but insufficient) condition of boundedness, therefore it is not compact.

%Part c
\subsection{$C = \qty{(x,y) \st x^2+y^4 \leq 1}$}
This set is compact. Although it is technically not a disk, it follows the same principle because it contains its boundary (meaning it is closed), but generally it is both complete and totally bounded, which is equivalent to being compact.

%Part d
\subsection{$D = \qty{\qty(1,\frac{1}{n}) \st n \in \N} \cup (1,0)$}
For the one single dimension ($1/n$) it is clear that it is closed and bounded (which are necessary but not sufficient conditions). 
Similarly, the dimension where 1 = 1 is also both closed and bounded. 
To prove compactness, one could use the equivalency of compactness with: $\forall_{{a_k}_{k\in \N}} \exists_{{a_{n_k}}} : a_{n_k} \to a \in A$.

% Problem 6
\newpage
\section{Let $A \subset S$ be a compact set. Show the following:}
Note: it is assumed that $B, C \subset S$ as well as I believe that was the intent.
%Part a
\subsection{$\partial A$ is compact.}
When $A$ is compact, it will be closed meaning $\partial A \subset A$. 
% Additionally, every sequence within $A$ will have a converging subsequence and since $\partial A$ will maintain both its completeness and its totally boundedness. 
From the original definition of a compact set, it is known that a finite subcover exists around the entirety of the set $A$, and from $\partial A \subset A$ it can be concluded that a finite subcover will exist for this boundary itself; therefore $\partial A$ is compact.

%Part b
\subsection{For any closed $B$, $A \cap B$ is compact.}




%Part c
\subsection{For any compact $C$, $A \cup C$ is compact.}




%Part d
\subsection{Union of infinitely many compact sets may be not compact.}
One of the necessary (but not sufficient) conditions of a compact set is that it is bounded. If an infinitely many compact sets are structured so that they become unbounded this could violate that necessary condition.





\end{document}
