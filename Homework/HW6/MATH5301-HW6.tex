\documentclass[]{article}

\usepackage{graphicx}

\usepackage[margin=1in]{geometry}

\setlength\parindent{0pt}

\usepackage{physics}
\usepackage{amsmath, amsfonts, amssymb, amsthm}

\usepackage{listings}

\usepackage{enumitem}
\renewcommand{\theenumi}{\alph{enumi}}
\renewcommand*{\thesection}{Problem \arabic{section}}
\renewcommand*{\thesubsection}{\alph{subsection})}
\renewcommand*{\thesubsubsection}{\quad \quad \roman{subsubsection})}

%Custom Commands
\newcommand{\Rel}{\mathcal{R}}
\newcommand{\R}{\mathbb{R}}
\newcommand{\C}{\mathbb{C}}
\newcommand{\N}{\mathbb{N}}
\newcommand{\Z}{\mathbb{Z}}
\newcommand{\Q}{\mathbb{Q}}

\newcommand{\toI}{\xrightarrow{\textsf{\tiny I}}}
\newcommand{\toS}{\xrightarrow{\textsf{\tiny S}}}
\newcommand{\toB}{\xrightarrow{\textsf{\tiny B}}}

\newcommand{\divisible}{ \ \vdots \ }
\newcommand{\st}{\ : \ }


% Theorem Definition
\newtheorem{definition}{Definition}
\newtheorem{assumption}{Assumption}
\newtheorem{theorem}{Theorem}
\newtheorem{lemma}{Lemma}


%opening

\title{MATH 5301 Elementary Analysis - Homework 6}

\author{Jonas Wagner}

\date{2021, October 17\textsuperscript{th}}

\begin{document}

\maketitle

% Problem 1
\section{}
% Part a
\subsection{Show that $\forall_{x > 0} \ x\in \R \implies \lim_{n\to \infty} x^{1/n} = 1$}

\begin{definition}
    for $f : (S_1, d_1) \to (S_2,d_2)$, \\
    if $a \in \bar{S}_1$ then we say 
    $$\lim_{x\to a} f(x) = b$$ if 
    $$\forall_{\epsilon>0} \exists_{\delta>0} \forall_{x \in S_1} : 0 < d_1(a,x) < \delta \implies d_2(f(x),b) < \epsilon$$
    or written in Ball form:
    $$\forall_{\epsilon>0} \exists_{\delta(\epsilon) > 0} \st \forall_{x \in \dot{B}_\delta(a) \subset S_1} \implies f(x) \in B_{\epsilon}(b) \subset S_2$$
\end{definition}

\begin{theorem}
    $$\forall_{x > 0} \ x\in \R \implies \lim_{n\to \infty} x^{1/n} = 1$$
    \begin{proof}
        \begin{align*}
            \forall_{x > 0} \ x\in \R = \forall_{x \in \R_+}
            &\implies \forall_{\epsilon>0} \exists_{\delta>0} \forall_{x \in S_1} : 0 < d_1(a,x) < \delta \implies d_2(f(x),b) < \epsilon
        \end{align*}
        % \begin{align*}
        %     \forall_{x > 0} \ x\in \R 
        %         &\implies \lim_{n\to \infty} f(x) = x^{1/n} = 1 &\\
        %     &\implies \forall_{\epsilon>0} \exists_{\delta(\epsilon) > 0} \st \forall_{x \in \dot{B}_\delta(a)} \implies f(x) \in B_{\epsilon}(b)\\
        %     &\implies \forall_{\epsilon>0} \exists_{\delta(\epsilon) > 0} \st \forall_{x\in \R} : d_1 (a, x) < \delta(\epsilon) \implies x^{1/n} \in \R : d_2(b,x^{1/n}) < \epsilon
        % \end{align*}
    \end{proof}
\end{theorem}





%Part b
\subsection{Show that for any bounded sequence $\{a_n\}$ and any sequence $\{b_n\}$, converging to zero, the sequence $\{a_n b_n\}$ converges to zero.}
\begin{theorem}
    $\forall \{a_n\}$ bounded $\land$ $\forall \{b_n\}$ converging to zero $\implies$ $\{a_n b_n \}$ converges to zero.
    \begin{proof}
        \begin{align*}
            \qty(\exists_{N_1>0} \st \forall_{n\in\N} a_n < N_1) \land \qty(\forall_{\epsilon>0} \exists_{N_2(\epsilon) > 0}  \forall_{n > N_2} b_n < \epsilon) \implies \forall_{\epsilon>0} \exists_{N_3(\epsilon) > 0}  \forall_{n > N_3} a_n b_n < \epsilon\\
            \forall_{\epsilon_{2,3}>0} \exists_{N_1,N_2,N_3 > 0} \st \forall{n > \max(N_2,N_3)} \qty((a_n < N_1) \land (b_n < \epsilon_{2}) \implies a_n b_n < \epsilon_3)
        \end{align*}
        Which clearly implies $\{a_n b_n\}$ converges to zero.
    \end{proof}
\end{theorem}


%Part c
\subsection{Find the limit $\lim_{n\to\infty} a_n$. Prove the convergence.}
$$a_n = \sqrt{2 + \sqrt{2 + \sqrt{2 + \cdots + \sqrt{2}}}}$$

This is a pretty simple that it is just $2$. This true for any of a similar type of problem.




%Part d
\subsection{Find the limit $\lim_{n\to\infty} a_n$ where $a_n = 2 + \cfrac{1}{2+\cfrac{1}{\ddots + \frac{1}{2}}}$. Prove the convergence.}

This one just converges to around $2.4$... not that that is the answer, I just ran out of time to do correctly...






% Problem 2
\newpage
\section{True or False?}

\begin{definition}
    for $f: (S_1, d_1) \to (S_2, d_2)$, $f(x)$ is a continuous function iff
    $$\forall_{x\in S_1} \forall_{\epsilon>0} \exists_{\delta(x,\epsilon)>0} \forall_{y\in S_1} : d_1(x,y) < \delta \implies d_2(f(x),f(y)) < \epsilon$$
\end{definition}

%Part a
\subsection{If $f: (S_1,d_1) \to (S_2,d_2)$ is continuous and $U\subset S_1$ is open then $f(U) \subset S_2$ is also open}
\begin{theorem}
    If $f: (S_1,d_1) \to (S_2,d_2)$ is continuous and $U\subset S_1$ is open then $f(U) \subset S_2$ is also open.
    \begin{proof}
        \begin{align*}
            \forall_{x\in S_1} \forall_{\epsilon>0} \exists_{\delta(x,\epsilon)>0} \forall_{y\in S_1} : d_1(x,y) < \delta &\implies d_2(f(x),f(y)) < \epsilon \ \land \\
                &\land \forall_{x \in U} \exists_{\epsilon>0} \st B_\epsilon(x) \subset U \implies\\
                &\implies \forall_{f(x) \in F(U)} \exists_{\epsilon>0} \st B_\epsilon(f(x)) \subset U\\
            \forall_{x\in S_1} \forall_{\epsilon>0} \exists_{\delta(x,\epsilon)>0} \forall_{y\in S_1} : d_1(x,y) < \delta &\implies d_2(f(x),f(y)) < \delta \land \\
                &\land \forall_{x \in U} \exists_{\epsilon_1>0} \st \forall_{y \in S_1} d(x,y) < \epsilon_1 \subset U \implies\\
                &\implies \forall_{f(x) \in F(U)} \exists_{\epsilon_2>0} \st \forall_{f(x) \in f(U)} : \forall_{f(y) \in S_2} : d_2(f(x),f(y)) < \epsilon_2 \subset U
        \end{align*}
        which is clearly true.
    \end{proof}
\end{theorem}

%Part b
\subsection{$f: (S_1,d_1) \to (S_2,d_2)$ is continuous $\iff$ $\forall_{C \subset S_2}$ closed, the set $f^{-1} (C) \subset S_1$ is also closed.}
\begin{theorem}
    $f: (S_1,d_1) \to (S_2,d_2)$ is continuous $\iff$ $\forall_{C \subset S_2}$ closed, the set $f^{-1} (C) \subset S_1$ is also closed.
    \begin{proof}
        \begin{align*}
            \forall_{x\in S_1} \forall_{\epsilon>0} \exists_{\delta(x,\epsilon)>0} \forall_{y\in S_1} : d_1(x,y) < \delta 
                &\implies d_2(f(x),f(y)) < \epsilon \iff\\
                &\iff \forall_{C\subset S_2} : \lnot \textnormal{open} \implies f^{-1}(C) \lnot \textnormal{open}
                &\iff \forall_{c \in C \subset S_2} \exists_{! x \in f^{-1}(C) \subset S_1} : f(x) = c
                % &\iff \forall_{C\subset S_2} : \forall_{x \in C^\complement} \exists_{\epsilon>0} : B_{\epsilon}(x) \in C^\complement \implies\\
                % &\implies \forall_{x \in f^{-1}(C)^\complement} \exists_{\epsilon>0} : B_{\epsilon}(x) \in f^{-1}(C)^\complement\\
        \end{align*}
        By definition of a function, $\forall_{y \in Y} \exists_{! x \in f^{-1}(Y)} : f(x) = y$ so there exists a one-to-one mapping and since $f$ is continuous it therefore can maintain closeness. So true.
    \end{proof}
\end{theorem}

%Part c
\subsection{If $f,g: (S,d) \to \R$ continuous $\implies m(x) := \max(f(x),g(x))$ and $n(x) := \min(f(x),g(x))$ are both continuous.}
\begin{theorem}
    If $f,g: (S,d) \to \R$ continuous $\implies m(x) := \max(f(x),g(x))$ and $n(x) := \min(f(x),g(x))$ are both continuous.
    \begin{proof}
        \begin{align*}
            \forall_{x,y\in S} \forall_{\epsilon_{1,2}>0} \exists_{\delta(x,\epsilon)>0} : d(x,y) < \delta 
                &\implies d(f(x),f(y)) < \epsilon_1 \land d(g(x),g(y)) < \epsilon_2 \implies\\
            &\implies \forall_{x,y\in S} \forall_{\epsilon_{3,4}>0} \exists_{\delta(x,\epsilon)>0} : d(x,y) < \delta \implies\\
            &\implies d(\max(f(x),g(x)),\max(f(y),g(y))) < \epsilon_3 \land\\
            &\qquad \land d(\min(f(x),g(x)),\min(f(y),g(y))) < \epsilon_4
        \end{align*}
        Upon inspection, $\epsilon_3, \epsilon_4 \leq \epsilon_1 + \epsilon_2$ at the worst case (at least thats what I learned when doing uncertainty calculations in the past...), therefore the implication is obvious.
    \end{proof}    
\end{theorem}

\newpage
%Part d
\subsection{If $f: (S,d) \to \R$ continuous the for any Cauchy sequence $\{x_n\} \in S$, the sequence $f(x_n)$ is a Cauchy sequence in $\R$}
\begin{definition}
    For the metric space $(S,d)$, a sequence $\{x_n\}$ is considered a \textbf{\underline{Cauchy Sequence}} if 
    $$\forall_{\epsilon>0} \exists_{N\in\N} \st \forall_{m,n > N} d(x_m,x_n) < \epsilon$$
\end{definition}
\begin{theorem}
    If $f: (S,d) \to \R$ continuous the for any Cauchy sequence $\{x_n\} \in S$, the sequence $f(x_n)$ is a Cauchy sequence in $\R$.
    \begin{proof}
        \begin{align*}
            \forall_{x,y\in S} \forall_{\epsilon_1>0} \exists_{\delta(x,\epsilon_1)>0} : d(x,y) < \delta \implies d(f(x),f(y)) < \epsilon_1 \implies\\
            \implies(\forall_{\epsilon_2>0} \exists_{N\in\N} \st \forall_{m,n > N} d(x_m,x_n) < \epsilon_2)\implies\\
            \implies(\forall_{\epsilon_3>0} \exists_{N\in\N} \st \forall_{m,n > N} d(f(x_n),f(x_m)) < \epsilon_3)
        \end{align*}
        From the continuous implication that: $\delta \to 0 \implies d(f(x),f(y)) \to 0$, the fact that $\{x_n\}$ converges mean that $\epsilon_2 \to 0 \implies d(f(x_n),f(x_m)) \to 0$. Therefore the sequence $\{f(x_n)\}$ would be considered a Cauchy sequence. 
    \end{proof}
\end{theorem}




% Problem 3
\newpage
\section{Prove the following properties of continuous functions:}

%Part a
\subsection{$\forall_{a,b\in\R} \forall_{f,g : S_1 \to S_2}$ continuous $\implies (a f + b g) (x) := a f(x) + b g(x) : S_1 \to S_2$ continuous.}
\begin{theorem}
    $\forall_{a,b\in\R} \forall_{f,g : S_1 \to S_2}$ continuous $\implies (a f + b g) (x) := a f(x) + b g(x) : S_1 \to S_2$ continuous.
    \begin{proof}
        \begin{align*}
            \forall_{a,b\in\R} \forall_{f,g : S_1 \to S_2} : &\\
                &\qty(\forall_{x\in S_1} \forall_{\epsilon>0} \exists_{\delta(x,\epsilon)>0} \forall_{y\in S_1} : d_1(x,y) < \delta \implies d_2(f(x),f(y)) < \epsilon) \land\\
                &\land \qty(\forall_{x\in S_1} \forall_{\epsilon>0} \exists_{\delta(x,\epsilon)>0} \forall_{y\in S_1} : d_1(x,y) < \delta \implies d_2(g(x),g(y)) < \epsilon) \implies\\
                &\implies \qty(\forall_{x\in S_1} \forall_{\epsilon>0} \exists_{\delta(x,\epsilon)>0} \forall_{y\in S_1} : d_1(x,y) < \delta \implies d_2(a f(x) + b g(x),a f(y) + b g(y)) < \epsilon)\\
            \forall_{a,b\in\R} \forall_{f,g : S_1 \to S_2} : &\\
                &\forall_{x\in S_1} \forall_{\epsilon_{1,2,3}>0} \exists_{\delta(x,\epsilon)>0} \forall_{y\in S_1} : d_1(x,y) < \delta \implies\\
                &\implies \qty(d_2(f(x),f(y) < \epsilon_1) \land \qty(d_2(g(x),g(y)) < \epsilon_2)) \implies\\
                    &\qquad \implies d_2(a f(x) + b g(x),a f(y) + b g(y)) < \epsilon_3
        \end{align*}
        Which by the additive and multiplicative properties of a metric space, along with the triangular inequality, this is clearly true:
        $$d_2(a f(x) + b g(x), a f(y) + b g(y)) \leq a d_2(f(x),f(y)) + b d_2(g(x),g(y))$$
        Therefore this statement is true.
    \end{proof}
\end{theorem}

%Part b
\subsection{$\forall_{f : S_1 \to S_2}$ continuous and $\forall_{h : S_1 \to \R}$ continuous $\implies (h f) (x) := h(x) \cdot f(x) : S_1 \to S_2$ continuous.}
\begin{theorem}
    $\forall_{f : S_1 \to S_2}$ continuous and $\forall_{h : S_1 \to \R}$ continuous $\implies (h f) (x) := h(x) \cdot f(x) : S_1 \to S_2$ continuous.
    \begin{proof}
        \begin{align*}
            &\forall_{f : S_1 \to S_2} : \forall_{x\in S_1} \forall_{\epsilon>0} \exists_{\delta(x,\epsilon)>0} \forall_{y\in S_1} : d_1(x,y) < \delta \implies d_2(f(x),f(y)) < \epsilon) \land\\
                &\qquad\land \forall_{h : S_1 \to \R} : \forall_{x\in S_1} \forall_{\epsilon>0} \exists_{\delta(x,\epsilon)>0} \forall_{y\in S_1} : d_1(x,y) < \delta \implies d_2(h(x),h(y)) < \epsilon) \implies\\
                &\qquad \qquad \implies \qty(\forall_{x\in S_1} \forall_{\epsilon>0} \exists_{\delta(x,\epsilon)>0} \forall_{y\in S_1} : d_1(x,y) < \delta \implies d_2(h(x) \cdot f(x),h(y) \cdot f(y)) < \epsilon)\\
            &\forall_{f : S_1 \to S_2} \land \forall_{h : S_1 \to \R} : \forall_{x\in S_1} \forall_{\epsilon_{1,2,3}>0} \exists_{\delta(x,\epsilon)>0} \forall_{y\in S_1} : d_1(x,y) < \delta \implies\\
                &\qquad \implies \qty(d_2(f(x),f(y))< \epsilon_1 \land d_2(h(x),h(y)) < \epsilon_2 \implies d_2(h(x) \cdot f(x), h(y) \cdot f(y)) < \epsilon_3)
        \end{align*}
        Which by the multiplicative properties of a metric space this is clearly true:
        $$d_2(h(x) \cdot f(x), h(y) \cdot f(y)) \leq d_2(h(x),h(y)) \cdot d_2(f(x),f(y))$$
        Therefore this statement is true.
    \end{proof}
\end{theorem}

\newpage
%Part c
\subsection{$h(x) \neq 0 \forall_{x\in S_1} \implies \frac{1}{h(x)}$ is continuous.}
Note: assuming $h : S_1 \to \R \backslash \{0\}$ and continuous.
\begin{theorem}
    $\forall_{h : S_1 \to \R \backslash \{0\}}$ continuous $\implies (h(x))^{-1} := \frac{1}{h(x)}: S_1 \to \R$ continuous.
    \begin{proof}
        \begin{align*}
            \forall_{h : S_1 \to \R \backslash \{0\}} : 
                &\forall_{x\in S_1} \forall_{\epsilon>0} \exists_{\delta(x,\epsilon)>0} \forall_{y\in S_1} : d_1(x,y) < \delta \implies d_2(h(x),h(y)) < \epsilon) \implies\\
                &\implies \qty(\forall_{x\in S_1} \forall_{\epsilon>0} \exists_{\delta(x,\epsilon)>0} \forall_{y\in S_1} : d_1(x,y) < \delta \implies d_2\qty(\frac{1}{h(x)},\frac{1}{h(y)}) < \epsilon)\\
            \forall_{h : S_1 \to \R \backslash \{0\}} :
                &\forall_{x\in S_1} \forall_{\epsilon_{1,2}>0} \exists_{\delta(x,\epsilon)>0} \forall_{y\in S_1} : d_1(x,y) < \delta \implies d_2(h(x),h(y)) < \epsilon) \implies\\
                &\implies d_2(h(x),h(y)) < \epsilon_1 \land d_2\qty(\frac{1}{h(x)},\frac{1}{h(y)})< \epsilon_2
        \end{align*}
        Within $(\R \backslash \{0\}, d_2)$, 
        $$d_2(a,b) < \epsilon_1 \implies \exists_{\epsilon_2>0} : d_2 \qty(\frac{1}{a}, \frac{1}{b}) < \epsilon_2$$
        Therefore it's true.
    \end{proof}
\end{theorem}


% Problem 4
\newpage
\section{Prove the following statement:}
If $A$ and $B$ are two closed nonempty disjoint % means A \cap B = \emptyset...
sets in the metric space (S,d) then there exists a continuous function $\mathcal{X}(x)$ such that $\mathcal{X}(x) = 0$ for all $x \in A$ and $\mathcal{X} = 1$ for all $x \in B$.
\begin{definition}
    A function $f : (S_1, d_1) \to (S_2, d_2)$ is considered \textbf{\underline{Lipschitz Continous Function}} if 
    $$\exists_{L>0} : \forall_{x,y \in S_1} d_2 (f(x),f(y)) \leq L d_1(x,y)$$
\end{definition}


\begin{theorem}
    $$A, B \neq \emptyset \in (S,d) : A \cap B = \emptyset \implies \exists_{\mathcal{X}: S \to \R} \textnormal{continuous} : \qty( \forall_{x \in A} \mathcal{X} = 0) \land \qty(\forall_{x \in B} \mathcal{X} = 1)$$
    \begin{proof}
        %Part a
        Define the distance from the point $x$ to the set $A$ as 
        $$\rho_A(x) := \inf_{y\in A} d(x,y)$$
        %Part b
        \begin{lemma}
            $$\rho_A(x) = 0 \iff x \in \bar{A}$$ % 0 if in closure... (compliment of interior)
            \begin{proof}
                The lower bound of $d(x,y)$ for $y\in A$ will never be zero unless $x \in \bar{A}$, so clearly the $\inf$ will be zero if and only if $x \in \bar{A}$.
            \end{proof}
        \end{lemma}
        %Part c
        \begin{lemma}
            $\rho_A(x)$ is lipshits with $L = 1$
            \begin{proof}
                \begin{align*}
                    \exists_{L>0} : \forall_{x,y \in S} d(\rho_A(x),\rho_A(y)) \leq L d(x,y)\\
                    \intertext{Let $L=1$,}
                    \forall_{x,y \in S} d(\rho_A(x),\rho_A(y)) \leq d(x,y)\\
                \end{align*}
                By definition of $\rho_A(x)$,
                $$\rho_A(x) = \inf_{y\in A} d(x,y) \leq d(x,y) \forall y \in A$$
                Therefore, $d(\rho_A(x),\rho_A(y)) \leq d(x,y)$ and $\rho_A(x)$ is lipshits with $L=1$.
            \end{proof}
        \end{lemma}
        %Part d
        Consider $\mathcal{X}(x) = \cfrac{\rho_B(x)}{\rho_A(x) + \rho_B(x)}$,
        Although I believe it has merit I will actually be using the similar version: 
        $\mathcal{X}(x) = \cfrac{\rho_A(x)}{\rho_A(x) + \rho_B(x)}$
        From lemma 2 it is clear that $\rho_A(x)$ and $\rho_B(x)$ are Lipschitz and therefore this rational combination of these will still be continuous within its defined regions.
        From Lemma 1, it is clear that  $\cfrac{\rho_A(x)}{\rho_A(x) + \rho_B(x)} = 0 \forall_{x\in \bar{A}} and \cfrac{\rho_A(x)}{\rho_A(x) + \rho_B(x)} = 1 \forall x \in \bar{B}$.
    \end{proof}
\end{theorem}






% Problem 5
\newpage
\section{Which of following sets in $\R^2$ are compact?}
\begin{definition}
    Let $(S,d)$ be a metric space with $A \subset S$,
    \begin{enumerate}
        \item For $\{U_\alpha\}_{\alpha \in A}, \ U_\alpha \subset S$, is a \textbf{\underline{cover}} of the set $A$ if 
        $$A \subset \bigcup_{\alpha\in A} U_\alpha$$
        \item A cover $\{U_\alpha\}_{\alpha \in A}$ of $A$ is an \textbf{\underline{open cover}} if $\forall_{\alpha \in A}$ $U_\alpha$ is an open set.
        \item $\{V_\beta\}_{\beta\in B}$ is called a \textbf{\underline{subcover}} of $\{U_\alpha\}_{\alpha \in A}$ if
        \begin{enumerate}
            \item $\{V_\beta\}_{\beta\in B}$ is a cover of $A$
            \item $\forall_{\beta \in B} \exists_{\alpha \in A} V_\beta = U_\alpha$
        \end{enumerate}
        \item A cover with a finite number of sets is called a \textbf{\underline{finite cover}}.
    \end{enumerate}
\end{definition}
\begin{definition}
    For $A \subset (S,d)$, $A$ is \textbf{\underline{compact}} if for every open cover of $A$ there exists a finite sub cover.
\end{definition}

%Part a
\subsection{$A = \qty{(x,y) \st x^2 - y^2 \leq 1}$}
This set (a cone I believe) is not bounded and therefore no, through contradiction of the necessary (but insufficient) condition of boundedness, it is not compact.

%Part b
\subsection{$B = \qty{(x,y) \st 0 < x^2 + y^2 \leq 1}$}
This set is a disk, but because it excludes the origin it is no longer closed which is a necessary (but insufficient) condition of boundedness, therefore it is not compact.

%Part c
\subsection{$C = \qty{(x,y) \st x^2+y^4 \leq 1}$}
This set is compact. Although it is technically not a disk, it follows the same principle because it contains its boundary (meaning it is closed), but generally it is both complete and totally bounded, which is equivalent to being compact.

%Part d
\subsection{$D = \qty{\qty(1,\frac{1}{n}) \st n \in \N} \cup (1,0)$}
For the one single dimension ($1/n$) it is clear that it is closed and bounded (which are necessary but not sufficient conditions). 
Similarly, the dimension where 1 = 1 is also both closed and bounded. 
To prove compactness, one could use the equivalency of compactness with: $\forall_{{a_k}_{k\in \N}} \exists_{{a_{n_k}}} : a_{n_k} \to a \in A$.

% Problem 6
\newpage
\section{Let $A \subset S$ be a compact set. Show the following:}
Note: it is assumed that $B, C \subset S$ as well as I believe that was the intent.
%Part a
\subsection{$\partial A$ is compact.}
When $A$ is compact, it will be closed meaning $\partial A \subset A$. 
Additionally, every sequence within $A$ will have a converging subsequence and since $\partial A$ will maintain both its completeness and its totally boundedness. 
From the original definition of a compact set, it is known that a finite subcover exists around the entirety of the set $A$, and from $\partial A \subset A$ it can be concluded that a finite subcover will exist for this boundary itself; therefore $\partial A$ is compact.

%Part b
\subsection{For any closed $B$, $A \cap B$ is compact.}
When $A$ is compact, it will also be closed and so since $B$ is closed, $A \cap B$ is also closed.
Additionally, by definition of an intersection of sets, $A \cap B \subset A$.
Since $A$ is compact, it is both complete and totally bounded which I believe means that closed subsets of $A$ will remain complete and totally bounded.
Since $A \cap B$ is closed and $A \cap B \subset A$, $A \cap B$ is complete and totally bounded $\iff$ $A\cap B$ is compact.

%Part c
\subsection{For any compact $C$, $A \cup C$ is compact.}
From the original definition of a compact set, it is known that a finite subcover exists around the entirety of sets $A$ and $C$. 
$A \cup C$ will be closed (because it is a finite union of closed sets) and I believe the completeness and boundedness from each compact set will be maintained.
Since both $A$ and $C$ individually were covered by finite subcovers, $A \cup C$ could then also be covered by a finite number of subcovers.
Therefore $A \cup C$ satisfies the criteria to be a compact set.

%Part d
\subsection{Union of infinitely many compact sets may be not compact.}
One of the necessary (but not sufficient) conditions of a compact set is that it is bounded. If an infinitely many compact sets are structured so that they become unbounded this could violate that necessary condition.

\end{document}
