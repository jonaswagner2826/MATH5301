% Standard Article Definition
\documentclass[]{article}

% Page Formatting
\usepackage[margin=1in]{geometry}
\setlength\parindent{0pt}

% Graphics
\usepackage{graphicx}

% Math Packages
\usepackage{physics}
\usepackage{amsmath, amsfonts, amssymb, amsthm}
\usepackage{mathtools}

% Extra Packages
\usepackage{listings}
\usepackage{hyperref}

% Section Heading Settings
\usepackage{enumitem}
\renewcommand{\theenumi}{\alph{enumi}}
\renewcommand*{\thesection}{Problem \arabic{section}}
\renewcommand*{\thesubsection}{\alph{subsection})}
\renewcommand*{\thesubsubsection}{\quad \quad \roman{subsubsection})}

%Custom Commands
\newcommand{\Rel}{\mathcal{R}}
\newcommand{\R}{\mathbb{R}}
\newcommand{\C}{\mathbb{C}}
\newcommand{\N}{\mathbb{N}}
\newcommand{\Z}{\mathbb{Z}}
\newcommand{\Q}{\mathbb{Q}}

\newcommand{\toI}{\xrightarrow{\textsf{\tiny I}}}
\newcommand{\toS}{\xrightarrow{\textsf{\tiny S}}}
\newcommand{\toB}{\xrightarrow{\textsf{\tiny B}}}

\newcommand{\divisible}{ \ \vdots \ }
\newcommand{\st}{\ : \ }


% Theorem Definition
\newtheorem{definition}{Definition}
\newtheorem{assumption}{Assumption}
\newtheorem{theorem}{Theorem}
\newtheorem{lemma}{Lemma}
\newtheorem{proposition}{Proposition}
\newtheorem{example}{Example}


%opening
\title{MATH 5301 Elementary Analysis - Homework 10}
\author{Jonas Wagner}
\date{2021, November 12\textsuperscript{th}}

\begin{document}

\maketitle

% Problem 1 ----------------------------------------------
\section{}
Prove that the closure and the interior of a convex set $A \subset \R^n$ are also convex.

% Convex Set
\begin{definition}
    The set $A$ is called \emph{\underline{convex}} if
    \[
        \forall_{x, y \in A} \forall_{t \in [0, 1]}
        \qty((t) x + (1 - t) y) \in A
    \]
\end{definition}

% Interior and Closure
\begin{definition}
    For a given set $A \subseteq (S,d)$,
    \begin{enumerate}
        \item the \emph{\underline{interior}} of $A$ is defined as
        \[
            \text{int}(A) = \qty{
                x \in A \st \exists_{\epsilon > 0} B_{\epsilon}(x) \subset A
            }
        \]
        \item the \emph{\underline{closure}} of $A$ is defined as
        \[
            \overline{A} = \qty{
                x \in S \st \forall_{\epsilon > 0} B_\epsilon(x) \cap A \neq \emptyset
            }
        \]
    \end{enumerate}
\end{definition}

% Problem 1 Theorem
\begin{theorem}
    If $A \subset \R^n$ is a convex set,
    then the closure of $A$, $\overline{A}$, is also convex.
    \begin{proof}
        $A$ being convex means that\[
            \forall_{x, y \in A} \forall_{t \in [0, 1]}
            \qty((t) x + (1 - t) y) \in A
        \]
        $\overline{A}$ is defined by\[
            \overline{A} = \qty{
                x \in \R^n \st \forall_{\epsilon > 0} B_\epsilon(x) \cap A \neq \emptyset
            }
        \]
        For $\overline{A}$ to be convex, the following would be true:\[
            \forall_{x, y \in \overline{A}} \forall_{t \in [0,1]} 
                \qty((t) x + (1 - t) y) \in \overline{A}
        \]
        Additionally, since $\overline{A} = A \cup \partial A$, $\overline{A}$ is convex if
        \[
            \qty(
                \forall_{x \in A} \forall_{y \in \overline{A}} \forall_{t \in [0,1]} 
                    \qty((t) x + (1 - t) y) \in \overline{A}
            ) \land \qty(
                \forall_{x \in \partial A} \forall_{y \in \overline{A}} \forall_{t \in [0,1]} 
                    \qty((t) x + (1 - t) y) \in \overline{A}
            )
        \]
        Since $A \subset \overline{A}$, by definition the first statement is true, \[
            \forall_{x \in A} \forall_{y \in \overline{A}} \forall_{t \in [0,1]} 
                \qty((t) x + (1 - t) y) \in \overline{A}
        \]
        Additionally, since the boundary of $A$, $\partial A$, is the collection of limit points of $A$ and the limit points all exist within the neighborhood of elements in $A$, \[
            \forall_{x \in \partial A} \forall_{y \in \overline{A}} \forall_{t \in [0,1]} 
                    \qty((t) x + (1 - t) y) \in \overline{A}
        \]
        Therefore,
        \[
            \forall_{x, y \in \overline{A}} \forall_{t \in [0,1]} 
                \qty((t) x + (1 - t) y) \in \overline{A}
        \]
    \end{proof}
\end{theorem}

% Problem 2 ----------------------------------------------
\newpage
\section{}
Prove that the intersection of an arbitrary collection of convex sets $\cap_{i \in I} C_i$ is also convex.

% Problem 2 Theorem
\begin{theorem}
    If each of the sets within the collection $C_i \subset (S,d)$ are convex,
    then the intersection of the collection, $\cap_{i \in I}$ is also convex.
    \begin{proof}
        For $\cap_{i \in I}$ to be convex, the following must be true: \[
            \forall_{x, y \in \cap_{i \in I} C_i} 
                \forall_{t \in [0,1]} (t) x + (1 - t) y \in \cap_{i \in I} C_i
        \]
        Which is the same as: \[
            \forall_{x,y \in S} \st \forall_{i \in I} x,y \in C_i 
                \implies \forall_{t \in [0,1]} \forall_{i \in I} (t) x + (1 - t) y \in C_i
        \]
        Since all the sets $C_i$ are convex, by definition: \[
            \forall_{x, y \in C_i} \forall_{t \in [0,1]} 
                (t) x + (1 - t) y \in C_i
        \]
        Therefore this is true $\forall_{i \in I}$: \[
            \land_{i \in I} \forall_{x,y \in C_i}
                \implies \forall_{t \in [0,1]} (t) x + (1 - t) y \in C_i
        \]
        Which is equivalent to: \[
            \forall_{x, y \in \cap_{i \in I} C_i} 
                \forall_{t \in [0,1]} (t) x + (1 - t) y \in \cap_{i \in I} C_i
        \]

    \end{proof}
\end{theorem}

% Problem 3 ----------------------------------------------
\newpage
\section{}
Let $\{C_i\}_{i \in \N}$ be a sequence of nested convex sets in $\R^n$, i.e. $C_i \subset C_{i+1}$.
Prove that $\cup_{i = 1}^\infty C_i$ is also convex.

% Problem 3 Theorem
\begin{theorem}
    For the sequence of nested convex sets in $\R^n$, $\qty{C_i}_{i \in \N}$, a union of all the elements, $\cup_{i = 1}^\infty C_i$, is also convex.
    \begin{proof}
        Proof by induction.
        
        For $n = 1$, the set $\cup_{i = 1}^n C_i = C_1$ is convex.
        
        For $n = 2$, the set $\cup_{i = 1}^n C_i = C_1 \cup C_2$ is convex.
        \begin{proof}
            Since $C_1 \subset C_2$, $C_1 \cup C_2 = C_2$ and $C_2$ is convex.
        \end{proof}

        Assuming for $n = k$, $\cup_{i = 1}^k C_i = C_k$ is convex, 
        then for $n = k+1$, $\cup_{i = 1}^{k + 1} C_i = C_{k + 1}$ is convex.
        \begin{proof}
            Since $C_k \subset C_{k+1}$, \[
                \cup_{i = 1}^{k + 1} C_i = \cup_{i = 1}^{k} C_i \cup C_{k + 1} = C_{k + 1}
            \]
            which is convex.
        \end{proof}

        Therefore, by induction, \[
            \forall_{n \in \N} \cup_{i = i}^n C_i
        \]
        is convex.
        This implies $\cup_{i = 1}^\infty C_i$.
    \end{proof}
\end{theorem}

% Problem 4 -----------------------------------------------
\newpage
\section{}

% Convex Hull
\begin{definition}
    The \emph{\underline{convex hull}} for set $A \in (S,d)$ is defined as \[
        \textnormal{conv}(A) = \cap_{C \supseteq A \st C \textnormal{ convex}}
    \]
    Additionally, for $A \subset \R^n$, \[
        \textnormal{cov}(A) = \cup_{m = 1}^{\infty} C_m
    \]\[
        C_m = \qty{
            x \in \R^n \st
            x = \alpha_1 a_1 + \cdot + \alpha_m a_m, \
            a_1, \dots, a_m \in A, \
            \alpha_i \geq 0, \
            \sum_{i} \alpha_i = 1
        }
    \]
\end{definition}
% Open Set
\begin{definition}
    The set $A \subset V$ is called \underline{open} if \[ 
        \forall_{x\in A} \exists_{\epsilon>0} : B_\epsilon(x)\subset A
    \]
    or equivalently, \[
        \forall_{x \in A} \exists_{\epsilon>0} : \forall_{y \in V}\norm{x - y} < \epsilon \implies y \in A
    \]
\end{definition}
% Closed Set
\begin{definition}
    The set $A \subset V$ is called \underline{closed} if $A^c$ is open.
\end{definition}

% Part a
\subsection{}
Show that the convex hull of any open sets in $\mathcal{R}^n$ is open.

% Problem 4a Theorem
\begin{theorem}
    For any open set $A \subset \R^n$,
    then the convex hull, $\textnormal{conv}(A)$,
    is open.
    \begin{proof}
        By definition, $A$ will satisfy the open condition:\[
            \forall_{x \in A} \exists_{\epsilon>0} : \forall_{y \in \R^n}\norm{x - y} < \epsilon \implies y \in A
        \]
        The convex hull of $A$, $\textnormal{conv}(A)$, is the intersection of all convex sets that contain $A$.
        This means that the smallest convex set containing $A$ will share a portion of the boundary of $A$.
        Then the open condition will be true along the portion of the boundary in which $A$ and $\textnormal{conv}(A)$ share will be open.
        The smallest convex set that shares this boundary will then also be open since the open $\textnormal{int}(C)$ is contained by any closed set with the same boundary.
        Therefore, the convex hull would be open.
    \end{proof}
\end{theorem}

% Part b
\subsection{}
Provide an example of a closed set $A \subset \mathcal{R}^n$, such that its convex hull is not closed.

\begin{example}
    Let closed set $A \subset \R^2$ be defined as\[
        A := \qty{
            (x,y) \in \R^2 \st y \geq e^{-x^2}
        }
    \]
    Clearly, $A$ is closed, however the convex hull of $A$ is actually open due to the asymptote that occurs on the $x$-axis:\[
        \textnormal{conv}(A) := \qty{
            (x,y) \in \R^2 \st y > 0
        }
    \]
\end{example}

% Problem 5 -----------------------------------------------
\newpage
\section{}
Let $f : \R^n \to \R$ be a convex function and $A \subset \R^n$ be a bounded set.
Prove that $f(A)$ is bounded in $\R$.

% Convex Function
\begin{definition}
    Function $f : [a,b] \to \R$ is considered a \emph{\underline{convex function}} if\[
        \forall_{x_1,x_2 \in [a,b]} \forall_{t \in [0,1]}
        f((t) x_1 + (1 - t) x_2) \leq (t) f(x_1) + (1 - t) f(x_2)
    \]
\end{definition}

% Problem 5 Theorem
\begin{theorem}
    If the function $f : \R^n \to \R$ is a convex function,
    then for a bounded set $A \subset \R^n$, the image $f(A)$ is bounded in $\R$.
    \begin{proof}
        $f$ being convex means that\[
            \forall_{\va{x},\va{y} \in \R^n} \forall_{t \in [0,1]}
                f((t) \va{x} + (1 - t) \va{y}) \leq (t) f(\va{x}) + (1 - t) f(\va{y})
        \]
        $A$ being bounded means\[
            \exists_{N} \st \forall_{\va{x} \in A}  \norm{\va{x}} \leq N
        \]
        It is known that the a convex function with a non-constant output cannot obtain its maximum within $\textnormal{int}(A)$;
        therefore, $\arg \max_{\va{x} \in A} f(x) \in \partial A$.
        Since $A$ is bounded, \[
            \exists_N \st \forall_{x \in \partial A} \norm{x} < N
        \]
        Therefore, \[
            \exists_N \st \forall_{\va{x} \in A} f(x) < \max_{\va{x} \in \partial A} f(x) < N
        \]
        Which means $f(A)$ is bounded.
    \end{proof}
\end{theorem}

% Problem 6 -----------------------------------------------
\newpage
\section{}
Show that the convex hull of a compact set $A \subset \R^n$ is compact.
(\textit{Hint:} Caratheodory theorem)



















\end{document}
