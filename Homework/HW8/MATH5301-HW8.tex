% Standard Article Definition
\documentclass[]{article}

% Page Formatting
\usepackage[margin=1in]{geometry}
\setlength\parindent{0pt}

% Graphics
\usepackage{graphicx}

% Math Packages
\usepackage{physics}
\usepackage{amsmath, amsfonts, amssymb, amsthm}
\usepackage{mathtools}

% Code Def
\usepackage{listings}

% Section Heading Settings
\usepackage{enumitem}
\renewcommand{\theenumi}{\alph{enumi}}
\renewcommand*{\thesection}{Problem \arabic{section}}
\renewcommand*{\thesubsection}{\alph{subsection})}
\renewcommand*{\thesubsubsection}{\quad \quad \roman{subsubsection})}

%Custom Commands
\newcommand{\Rel}{\mathcal{R}}
\newcommand{\R}{\mathbb{R}}
\newcommand{\C}{\mathbb{C}}
\newcommand{\N}{\mathbb{N}}
\newcommand{\Z}{\mathbb{Z}}
\newcommand{\Q}{\mathbb{Q}}

\newcommand{\toI}{\xrightarrow{\textsf{\tiny I}}}
\newcommand{\toS}{\xrightarrow{\textsf{\tiny S}}}
\newcommand{\toB}{\xrightarrow{\textsf{\tiny B}}}

\newcommand{\divisible}{ \ \vdots \ }
\newcommand{\st}{\ : \ }


% Theorem Definition
\newtheorem{definition}{Definition}
\newtheorem{assumption}{Assumption}
\newtheorem{theorem}{Theorem}
\newtheorem{lemma}{Lemma}
\newtheorem{proposition}{Proposition}


%opening

\title{MATH 5301 Elementary Analysis - Homework 8}

\author{Jonas Wagner}

\date{2021, October 29\textsuperscript{th}}

\begin{document}

\maketitle

% Problem 1 ----------------------------------------------
\section{}
Show that the norms $\norm{\cdot}_1$, $\norm{\cdot}_p$ for $p > 1$, and $\norm{\cdot}_\infty$ are equivalent.

\begin{definition}
    For $\norm{\cdot}_a, \norm{\cdot}_b$ on $S$, 
    $\norm{\cdot}_a$ is said to be \emph{stronger} then $\norm{\cdot}_b$ if 
    \[
        \forall \{x_n\} \subset S \st x_n \xrightarrow[d_a]{} x \implies x_n \xrightarrow[d_b]{} x
    \]
\end{definition}
\begin{definition}
    $\norm{\cdot}_a$ and $\norm{\cdot}_b$ are said to be \emph{equivalent},  $\norm{\cdot}_a \sim \norm{\cdot}_b$,
    if $\norm{\cdot}_a$ is stronger then $\norm{\cdot}_b$ 
    and $\norm{\cdot}_b$ is stronger then $\norm{\cdot}_a$. 
    This means that
    \[
        \norm{\cdot}_a \sim \norm{\cdot}_b 
            \iff \exists{\alpha,\beta \in \R_{>0}} : 
            \forall_{x\in S} \alpha \norm{\cdot}_b \leq \norm{\cdot}_a \leq \beta \norm{x}_b
    \]
\end{definition}
\begin{definition} The following norms are defined as
    \begin{enumerate}
        \item $\norm{\cdot}_1 := \norm{x}_1 = \sum_{i=1}^n \abs{x_i} = \abs{x_1} + \abs{x_2} + \dots + \abs{x_n}$
        \item $\norm{\cdot}_2 := \norm{x}_2 = \qty(\sum_{i=1}^n \abs{x_i}^2)^{1/2} = \qty(\abs{x_1}^2 + \abs{x_2}^2 + \dots + \abs{x_n}^2)^{1/2}$
        \item $\norm{\cdot}_p := \norm{x}_p = \qty(\sum_{i=1}^n \abs{x_i}^p)^{1/p} =\qty(\abs{x_1}^p + \abs{x_2}^p + \dots + \abs{x_n}^p)^{1/p}, \ p > 1$
        \item $\norm{\cdot}_\infty := \norm{x}_\infty = \max_{i=1}^n \abs{x_i} = \max(\abs{x_1}, \abs{x_2}, \dots, \abs{x_n})$
    \end{enumerate}
\end{definition}

\newpage
\begin{theorem}
    The norms $\norm{\cdot}_1, \norm{\cdot}_p,$ and $\norm{\cdot}_\infty$ are equivalent.
    \begin{proof}
        % 1-norm ~ p-norm
        \begin{lemma}\label{lem:1-1toP}
            $\norm{\cdot}_1 \sim \norm{\cdot}_p$
            \begin{proof}
                $\norm{\cdot}_1 \sim \norm{\cdot}_p$ is true iff
                \begin{multline*}
                    \forall_{x} \exists_{\alpha,\beta \in \R_+} :\\
                    \alpha \norm{x}_p 
                        \leq \norm{x}_1 
                        \leq \norm{x}_p\\
                    \alpha \qty(\sum_{i=1}^n \abs{x_i}^p)^{1/p}
                        \leq \sum_{i=1}^n \abs{x_i}
                        \leq \beta \qty(\sum_{i=1}^n \abs{x_i}^p)^{1/p}\\
                \end{multline*}
                From the Holder's inequality we have 
                \begin{align*}
                    \norm{x}_1 = \sum_{i=1}^n \abs{x_i}
                        &= \sum_{i=1}^n \abs{x_i} (1)\\
                        &\leq \qty(\sum_{i=1}^n \abs{x_i}^p)^{1/p}
                            \qty(\sum_{i=1}^n \abs{1}^{(1-p)})^{1/(1-p)}\\
                        &\leq n^{1/(1-p)} \qty(\sum_{i=1}^n \abs{x_i}^p)^{1/p} 
                \end{align*}
                So for $0 < \alpha \leq n^{1/(1-p)}$ and $\beta \geq n^{1/(1-p)}$,
                \begin{align*}
                    &\alpha \qty(\sum_{i=1}^n \abs{x_i}^p)^{1/p}
                        &&\leq \sum_{i=1}^n \abs{x_i}
                        &&\leq \beta \qty(\sum_{i=1}^n \abs{x_i}^p)^{1/p}\\
                    &\qty(\sum_{i=1}^n \abs{x_i}^p)^{1/p}
                        &&\leq n^{1/(1-p)} \qty(\sum_{i=1}^n \abs{x_i}^p)^{1/p} 
                        &&\leq n^{1/(1-p)} \qty(\sum_{i=1}^n \abs{x_i}^p)^{1/p}
                \end{align*}
                Therefore,
                \[\norm{x}_p \leq \norm{x}_1 \leq n^{\frac{1}{1-p}} \norm{x}_p\]
                which proves $\norm{\cdot}_1 \sim \norm{\cdot}_p$.
            \end{proof}
        \end{lemma}
        % \newpage
        %1-norm ~ \infty-norm
        \begin{lemma}\label{lem:1-1toInfty}
            $\norm{\cdot}_1 \sim \norm{\cdot}_\infty$
            \begin{proof}
                $\norm{\cdot}_1 \sim \norm{\cdot}_\infty$ is true iff
                \begin{multline*} 
                    \forall_{x} \exists_{\alpha,\beta \in \R_+} :\\
                    \alpha \norm{x}_\infty \leq \norm{x}_1 \leq \beta \norm{x}_\infty\\
                    \alpha \max_{i=1}^n \abs{x_i} \leq \sum_{i=1}^n \abs{x_i} \leq \beta \max_{i=1}^n \abs{x_i}\\
                \end{multline*}
                Clearly, this is true for when $\alpha \in (0,1]$. Similarily, when $\beta \geq n$ then $\sum_{i=1}^n \max_{i=1}^n \abs{x_i}$ and then clearly greater then the $\norm{x}_1$; therefore $\norm{\cdot}_1 \sim \norm{\cdot}_\infty$.
            \end{proof}
        \end{lemma}
        From, Lemma \ref{lem:1-1to2}, Lemma \ref{lem:1-1toP}, and Lemma \ref{lem:1-1toInfty}, it is clear that $\forall_{p > 1}$:
        \[
            \norm{x}_\infty 
            \leq \norm{x}_p 
            \leq \norm{x}_1 
            \leq n^{1/{1-p}} \norm{x}_p 
            \leq n \norm{x}_\infty
        \]
        Therefore, $\norm{\cdot}_1 \sim \norm{\cdot}_p \sim \norm{\cdot}_\infty$ ($\forall_{p > 1}$).
    \end{proof}
\end{theorem}

% Problem 2
\newpage
\section{}
Let $(S, \norm{\cdot})$ and $(S', \norm{\cdot}')$ to be two normed spaces. 
Show that the following norms on $S \cross S'$ are equivalent.
\begin{enumerate}
    \item $\norm{(x,y)}_1 = \norm{x} + \norm{y}'$
    \item $\norm{(x,y)}_2 = \sqrt{\norm{x}^2 + (\norm{y}')^2}$
    \item $\norm{(x,y)}_p = \qty(\norm{x}^p + (\norm{y}')^p)^{1/p}$
    \item $\norm{(x,y)}_\infty = \max\qty{\norm{x} + \norm{y}'}$
\end{enumerate}

\begin{theorem}
    The norms $\norm{\cdot}_1,\norm{\cdot}_2,\norm{\cdot}_p,$ and $\norm{\cdot}_\infty$ are all equivalent on $S \cross S'$.
    \begin{proof}
        % 1-norm ~ 2-norm
        \begin{lemma}\label{lem:2-1to2}
            $\norm{\cdot}_1 \sim \norm{\cdot}_2$
            \begin{proof}
                $\norm{\cdot}_1 \sim \norm{\cdot}_2$ is true iff
                \begin{multline*}
                    \forall_{(x,y) \in S \cross S'} \exists_{\alpha,\beta \in \R_+} :\\
                    \alpha \norm{(x,y)}_2 
                        \leq \norm{(x,y)}_1 
                        \leq \norm{(x,y)}_2\\
                    \alpha (\norm{x}^2 + (\norm{y}')^2)^{1/2}
                        \leq \norm{x} + \norm{y}'
                        \leq \beta (\norm{x}^2 + (\norm{y}')^2)^{1/2}\\
                \end{multline*}
                From the Holder's inequality we have 
                \begin{align*}
                    \norm{(x,y)}_1 
                        &= \norm{x} + \norm{y}'\\
                    &= \qty(\norm{x} (1) + \norm{y}' (1))\\
                    &\leq \qty(\norm{x}^2 + (\norm{y}')^2)^{\frac{1}{2}} \qty(1^2 + 1^2)^{1 - \frac{1}{2}}\\
                    &= \qty(2)^\frac{1}{2} \qty(\norm{x}^2 + (\norm{y}')^2)^{\frac{1}{2}}\\
                    &= \sqrt{2} \sqrt{\norm{x}^2 + (\norm{y}')^2}\\
                    \norm{(x,y)}_1
                        &\leq \sqrt{2} \norm{(x,y)}_2
                \end{align*}
                Similarily, 
                \begin{align*}
                    \norm{(x,y)}_2 
                        &= \sqrt{\norm{x}^2 + (\norm{y}')^2}\\
                        % &= \frac{\sqrt{2}}{\sqrt{2}} \qty(\norm{x}^2 + (\norm{y}')^2)^{\frac{1}{2}}\\
                        % &= \frac{1}{\sqrt{2}} \qty(1^2 + 1^2)^{\frac{1}{2}} \qty(\norm{x}^2(1)^2 + (\norm{y}')^2(1)^2)^{\frac{1}{2}}\\
                        &= \qty(\norm{x} \norm{x} + \norm{y}' \norm{y}')^{\frac{1}{2}}\\
                        &\leq \qty(\qty(\norm{x}^2 + \qty(\norm{y}')^2)^{\frac{1}{2}})^2 \qty(\qty(\norm{x}^2 + \qty(\norm{y}')^2)^{\frac{1}{2}})^2\\
                        &= 2\qty(\norm{x}^2 + \qty(\norm{y}')^2)
                    % &= \qty(\norm{x} (1) + \norm{y}' (1))\\
                    % &\leq \qty(\norm{x}^2 + (\norm{y}')^2)^{\frac{1}{2}} \qty(1^2 + 1^2)^{1 - \frac{1}{2}}\\
                    % &= \qty(2)^\frac{1}{2} \qty(\norm{x}^2 + (\norm{y}')^2)^{\frac{1}{2}}\\
                    % &= \sqrt{2} \sqrt{\norm{x}^2 + (\norm{y}')^2}\\
                    % \norm{(x,y)}_1
                    %     &\leq \sqrt{2} \norm{(x,y)}_2
                \end{align*}
                So.... this isn't complete... need to do it still... figure out why you can just take 1/epsilon for it to work...
                % So for $0 < \alpha \leq n^{1/2}$ and $\beta \geq n^{1/2}$,
                % \begin{align*}
                %     &\alpha \qty(\sum_{i=1}^n \abs{x_i}^2)^{1/2}
                %         &&\leq \sum_{i=1}^n \abs{x_i}
                %         &&\leq \beta \qty(\sum_{i=1}^n \abs{x_i}^2)^{1/2}\\
                %     &\qty(\sum_{i=1}^n \abs{x_i}^2)^{1/2}
                %         &&\leq n^{1/2} \qty(\sum_{i=1}^n \abs{x_i}^2)^{1/2} 
                %         &&\leq n^{1/2} \qty(\sum_{i=1}^n \abs{x_i}^2)^{1/2}
                % \end{align*}
                % Therefore,
                % \[\norm{x}_2 \leq \norm{x}_1 \leq n^{\frac{1}{1-p}} \norm{x}_2\]
                % which proves $\norm{\cdot}_1 \sim \norm{\cdot}_2$.
            \end{proof}
        \end{lemma}
    %     \newpage
    %     % 1-norm ~ p-norm
    %     \begin{lemma}\label{lem:2-1toP}
    %         $\norm{\cdot}_1 \sim \norm{\cdot}_p$
    %         \begin{proof}
    %             $\norm{\cdot}_1 \sim \norm{\cdot}_p$ is true iff
    %             \begin{multline*}
    %                 \forall_{x \in \R^n} \exists_{\alpha,\beta \in \R_+} :\\
    %                 \alpha \norm{x}_p 
    %                     \leq \norm{x}_1 
    %                     \leq \norm{x}_p\\
    %                 \alpha \qty(\sum_{i=1}^n \abs{x_i}^p)^{1/p}
    %                     \leq \sum_{i=1}^n \abs{x_i}
    %                     \leq \beta \qty(\sum_{i=1}^n \abs{x_i}^p)^{1/p}\\
    %             \end{multline*}
    %             From the Holder's inequality we have 
    %             \begin{align*}
    %                 \norm{x}_1 = \sum_{i=1}^n \abs{x_i}
    %                     &= \sum_{i=1}^n \abs{x_i} (1)\\
    %                     &\leq \qty(\sum_{i=1}^n \abs{x_i}^p)^{1/p}
    %                         \qty(\sum_{i=1}^n \abs{1}^{(1-p)})^{1/(1-p)}\\
    %                     &\leq n^{1/(1-p)} \qty(\sum_{i=1}^n \abs{x_i}^p)^{1/p} 
    %             \end{align*}
    %             So for $0 < \alpha \leq n^{1/(1-p)}$ and $\beta \geq n^{1/(1-p)}$,
    %             \begin{align*}
    %                 &\alpha \qty(\sum_{i=1}^n \abs{x_i}^p)^{1/p}
    %                     &&\leq \sum_{i=1}^n \abs{x_i}
    %                     &&\leq \beta \qty(\sum_{i=1}^n \abs{x_i}^p)^{1/p}\\
    %                 &\qty(\sum_{i=1}^n \abs{x_i}^p)^{1/p}
    %                     &&\leq n^{1/(1-p)} \qty(\sum_{i=1}^n \abs{x_i}^p)^{1/p} 
    %                     &&\leq n^{1/(1-p)} \qty(\sum_{i=1}^n \abs{x_i}^p)^{1/p}
    %             \end{align*}
    %             Therefore,
    %             \[\norm{x}_p \leq \norm{x}_1 \leq n^{\frac{1}{1-p}} \norm{x}_p\]
    %             which proves $\norm{\cdot}_1 \sim \norm{\cdot}_p$.
    %         \end{proof}
    %     \end{lemma}
    %     \newpage
    %     %1-norm ~ \infty-norm
    %     \begin{lemma}\label{lem:2-1toInfty}
    %         $\norm{\cdot}_1 \sim \norm{\cdot}_\infty$
    %         \begin{proof}
    %             $\norm{\cdot}_1 \sim \norm{\cdot}_\infty$ is true iff
    %             \begin{multline*} 
    %                 \forall_{x \in \R^n} \exists_{\alpha,\beta \in \R_+} :\\
    %                 \alpha \norm{x}_\infty \leq \norm{x}_1 \leq \beta \norm{x}_\infty\\
    %                 \alpha \max_{i=1}^n \abs{x_i} \leq \sum_{i=1}^n \abs{x_i} \leq \beta \max_{i=1}^n \abs{x_i}\\
    %             \end{multline*}
    %             Clearly, this is true for when $\alpha \in (0,1]$. Similarily, when $\beta \geq n$ then $\sum_{i=1}^n \max_{i=1}^n \abs{x_i}$ and then clearly greater then the $\norm{x}_1$; therefore $\norm{\cdot}_1 \sim \norm{\cdot}_\infty$.
    %         \end{proof}
    %     \end{lemma}
    %     From, Lemma \ref{lem:2-1to2}, Lemma \ref{lem:1-1toP}, and Lemma \ref{lem:1-1toInfty}, it is clear that $\forall_{p > 1}$ (which implies $\norm{x}_2$ as well):
    %     \[
    %         \norm{x}_\infty 
    %         \leq \norm{x}_p 
    %         \leq \norm{x}_1 
    %         \leq n^{1/{1-p}} \norm{x}_p 
    %         \leq n \norm{x}_\infty
    %     \]
    %     Therefore, $\norm{\cdot}_1 \sim \norm{\cdot}_2 \sim \norm{\cdot}_p \sim \norm{\cdot}_\infty$ on $\R^n$ ($\forall_{p > 1}$).
    \end{proof}
\end{theorem}



% Problem 3
\newpage
\section{}
Let $X$ be a vector space and $V$ be a normed space. 
The function $f : X \to V$ is called bounded if $\exists M \st \forall_{x\in X} \implies \norm{f(x)} < M$. 
Consider the set $\mathcal{B}(X,V)$ of all bounded functions from $X \to V$. 

%Part a
\subsection{}
Show that $\mathcal{B}(X,V)$ is a vector space.


do the thing...




%Part b
\subsection{}
Show that the function $\mathcal{B}(X,V) \to \R_{+} :$
\[
    \norm{f}_\infty := \sup_{x\in X} \norm{f(x)}
\]
defines a norm on $\mathcal{B}(X,V)$.


Show the things that the norm needs... (mainly triangular inequality)





% Problem 4
\newpage
\section{}
Let $A$ be a dense set in metric space $(S,d)$, let $(V,d_1)$ be a complete metric space, and $f : A \to Y$ be a uniformly continuous function. 

%Part a
\subsection{Show that if $\{x_n\}$ is a Cauchy sequence in $A$ then $\{f(x_n)\}$ is a Cauchy sequence in $Y$.}




%Part b
\subsection{Show that there is only one continuous function $g : X \to Y$ so that $g(x) = f(x)$ forall $x \in $.}














% Problem 5
\newpage
\section{}
Let $(L, \norm{\cdot})$ be a Banach space. 
Let $L_0$ be a closed subspace of $L$. 
Define the factor-space $L/L_0$ as $l_1 := L/L_0 = \qty{x + y \st x \in L, y \in L_0}$. 
In other works, $L_1$ consists of all subsets of $L$ obtained from $L_0$ by shifting all its elements by some element $x$. 

%Part a
\subsection{Show that $L_1$ is a vector space.}


%Part b
\subsection{}
Define the function $\norm{\cdot} : L_1 \to \R_{+}$ as $\norm{x}_1 = \inf_{x - y \in L_0} \norm{y}$. 
Show that this function defines a norm on the space $L_1$.


%Part c
\subsection{Show that $L_1$ is a Banach space.}







% Problem 6
\newpage
\section{}
Let $C([-1,1])$ be the space of all continuous real-valued functions $f(x)$ with $x \in [-1,1]$. 
Let $\norm{f}_\infty := \sup_{x \in [-1,1]} \abs{f(x)}$. 
Find the distance from point $p = x^{2021}$ to the space $P_{2020}$ of all polynomials of degree less than or equal to 2020.





\end{document}
