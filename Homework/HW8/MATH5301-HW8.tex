% Standard Article Definition
\documentclass[]{article}

% Page Formatting
\usepackage[margin=1in]{geometry}
\setlength\parindent{0pt}

% Graphics
\usepackage{graphicx}

% Math Packages
\usepackage{physics}
\usepackage{amsmath, amsfonts, amssymb, amsthm}
\usepackage{mathtools}

% Code Def
\usepackage{listings}

% Section Heading Settings
\usepackage{enumitem}
\renewcommand{\theenumi}{\alph{enumi}}
\renewcommand*{\thesection}{Problem \arabic{section}}
\renewcommand*{\thesubsection}{\alph{subsection})}
\renewcommand*{\thesubsubsection}{\quad \quad \roman{subsubsection})}

%Custom Commands
\newcommand{\Rel}{\mathcal{R}}
\newcommand{\R}{\mathbb{R}}
\newcommand{\C}{\mathbb{C}}
\newcommand{\N}{\mathbb{N}}
\newcommand{\Z}{\mathbb{Z}}
\newcommand{\Q}{\mathbb{Q}}

\newcommand{\toI}{\xrightarrow{\textsf{\tiny I}}}
\newcommand{\toS}{\xrightarrow{\textsf{\tiny S}}}
\newcommand{\toB}{\xrightarrow{\textsf{\tiny B}}}

\newcommand{\divisible}{ \ \vdots \ }
\newcommand{\st}{\ : \ }


% Theorem Definition
\newtheorem{definition}{Definition}
\newtheorem{assumption}{Assumption}
\newtheorem{theorem}{Theorem}
\newtheorem{lemma}{Lemma}
\newtheorem{proposition}{Proposition}


%opening

\title{MATH 5301 Elementary Analysis - Homework 8}

\author{Jonas Wagner}

\date{2021, October 29\textsuperscript{th}}

\begin{document}

\maketitle

% Problem 1 ----------------------------------------------
\section{}
Show that the norms $\norm{\cdot}_1$, $\norm{\cdot}_p$ for $p > 1$, and $\norm{\cdot}_\infty$ are equivalent.

\begin{definition}
    For $\norm{\cdot}_a, \norm{\cdot}_b$ on $S$, 
    $\norm{\cdot}_a$ is said to be \emph{stronger} then $\norm{\cdot}_b$ if 
    \[
        \forall \{x_n\} \subset S \st x_n \xrightarrow[d_a]{} x \implies x_n \xrightarrow[d_b]{} x
    \]
\end{definition}
\begin{definition}
    $\norm{\cdot}_a$ and $\norm{\cdot}_b$ are said to be \emph{equivalent},  $\norm{\cdot}_a \sim \norm{\cdot}_b$,
    if $\norm{\cdot}_a$ is stronger then $\norm{\cdot}_b$ 
    and $\norm{\cdot}_b$ is stronger then $\norm{\cdot}_a$. 
    This means that
    \[
        \norm{\cdot}_a \sim \norm{\cdot}_b 
            \iff \exists{\alpha,\beta \in \R_{>0}} : 
            \forall_{x\in S} \alpha \norm{\cdot}_b \leq \norm{\cdot}_a \leq \beta \norm{x}_b
    \]
\end{definition}
\begin{definition} The following norms are defined as
    \begin{enumerate}
        \item $\norm{\cdot}_1 := \norm{x}_1 = \sum_{i=1}^n \abs{x_i} = \abs{x_1} + \abs{x_2} + \dots + \abs{x_n}$
        \item $\norm{\cdot}_2 := \norm{x}_2 = \qty(\sum_{i=1}^n \abs{x_i}^2)^{1/2} = \qty(\abs{x_1}^2 + \abs{x_2}^2 + \dots + \abs{x_n}^2)^{1/2}$
        \item $\norm{\cdot}_p := \norm{x}_p = \qty(\sum_{i=1}^n \abs{x_i}^p)^{1/p} =\qty(\abs{x_1}^p + \abs{x_2}^p + \dots + \abs{x_n}^p)^{1/p}, \ p > 1$
        \item $\norm{\cdot}_\infty := \norm{x}_\infty = \max_{i=1}^n \abs{x_i} = \max(\abs{x_1}, \abs{x_2}, \dots, \abs{x_n})$
    \end{enumerate}
\end{definition}

\newpage
\begin{theorem}
    The norms $\norm{\cdot}_1, \norm{\cdot}_p,$ and $\norm{\cdot}_\infty$ are equivalent.
    \begin{proof}
        % 1-norm ~ p-norm
        \begin{lemma}\label{lem:1-1toP}
            $\norm{\cdot}_1 \sim \norm{\cdot}_p$
            \begin{proof}
                $\norm{\cdot}_1 \sim \norm{\cdot}_p$ is true iff
                \begin{multline*}
                    \forall_{x} \exists_{\alpha,\beta \in \R_+} :\\
                    \alpha \norm{x}_p 
                        \leq \norm{x}_1 
                        \leq \norm{x}_p\\
                    \alpha \qty(\sum_{i=1}^n \abs{x_i}^p)^{1/p}
                        \leq \sum_{i=1}^n \abs{x_i}
                        \leq \beta \qty(\sum_{i=1}^n \abs{x_i}^p)^{1/p}\\
                \end{multline*}
                From the Holder's inequality we have 
                \begin{align*}
                    \norm{x}_1 = \sum_{i=1}^n \abs{x_i}
                        &= \sum_{i=1}^n \abs{x_i} (1)\\
                        &\leq \qty(\sum_{i=1}^n \abs{x_i}^p)^{1/p}
                            \qty(\sum_{i=1}^n \abs{1}^{(1-p)})^{1/(1-p)}\\
                        &\leq n^{1/(1-p)} \qty(\sum_{i=1}^n \abs{x_i}^p)^{1/p} 
                \end{align*}
                So for $0 < \alpha \leq n^{1/(1-p)}$ and $\beta \geq n^{1/(1-p)}$,
                \begin{align*}
                    &\alpha \qty(\sum_{i=1}^n \abs{x_i}^p)^{1/p}
                        &&\leq \sum_{i=1}^n \abs{x_i}
                        &&\leq \beta \qty(\sum_{i=1}^n \abs{x_i}^p)^{1/p}\\
                    &\qty(\sum_{i=1}^n \abs{x_i}^p)^{1/p}
                        &&\leq n^{1/(1-p)} \qty(\sum_{i=1}^n \abs{x_i}^p)^{1/p} 
                        &&\leq n^{1/(1-p)} \qty(\sum_{i=1}^n \abs{x_i}^p)^{1/p}
                \end{align*}
                Therefore,
                \[\norm{x}_p \leq \norm{x}_1 \leq n^{\frac{1}{1-p}} \norm{x}_p\]
                which proves $\norm{\cdot}_1 \sim \norm{\cdot}_p$.
            \end{proof}
        \end{lemma}
        % \newpage
        %1-norm ~ \infty-norm
        \begin{lemma}\label{lem:1-1toInfty}
            $\norm{\cdot}_1 \sim \norm{\cdot}_\infty$
            \begin{proof}
                $\norm{\cdot}_1 \sim \norm{\cdot}_\infty$ is true iff
                \begin{multline*} 
                    \forall_{x} \exists_{\alpha,\beta \in \R_+} :\\
                    \alpha \norm{x}_\infty \leq \norm{x}_1 \leq \beta \norm{x}_\infty\\
                    \alpha \max_{i=1}^n \abs{x_i} \leq \sum_{i=1}^n \abs{x_i} \leq \beta \max_{i=1}^n \abs{x_i}\\
                \end{multline*}
                Clearly, this is true for when $\alpha \in (0,1]$. Similarly, when $\beta \geq n$ then $\sum_{i=1}^n \max_{i=1}^n \abs{x_i}$ and then clearly greater then the $\norm{x}_1$; therefore $\norm{\cdot}_1 \sim \norm{\cdot}_\infty$.
            \end{proof}
        \end{lemma}
        From, Lemma \ref{lem:1-1toP} and Lemma \ref{lem:1-1toInfty}, it is clear that $\forall_{p > 1}$:
        \[
            \norm{x}_\infty 
            \leq \norm{x}_p 
            \leq \norm{x}_1 
            \leq n^{1/{1-p}} \norm{x}_p 
            \leq n \norm{x}_\infty
        \]
        Therefore, $\norm{\cdot}_1 \sim \norm{\cdot}_p \sim \norm{\cdot}_\infty$ ($\forall_{p > 1}$).
    \end{proof}
\end{theorem}

% Problem 2
\newpage
\section{}
Let $(S, \norm{\cdot})$ and $(S', \norm{\cdot}')$ to be two normed spaces. 
Show that the following norms on $S \cross S'$ are equivalent.
\begin{enumerate}
    \item $\norm{(x,y)}_1 = \norm{x} + \norm{y}'$
    \item $\norm{(x,y)}_2 = \sqrt{\norm{x}^2 + (\norm{y}')^2}$
    \item $\norm{(x,y)}_p = \qty(\norm{x}^p + (\norm{y}')^p)^{1/p}$
    \item $\norm{(x,y)}_\infty = \max\qty{\norm{x} + \norm{y}'}$
\end{enumerate}

\begin{theorem}
    The norms $\norm{\cdot}_1,\norm{\cdot}_2,\norm{\cdot}_p,$ and $\norm{\cdot}_\infty$ are all equivalent on $S \cross S'$.
    \begin{proof}
        % 1-norm ~ 2-norm
        \begin{lemma}\label{lem:2-1to2}
            $\norm{\cdot}_1 \sim \norm{\cdot}_2$
            \begin{proof}
                $\norm{\cdot}_1 \sim \norm{\cdot}_2$ is true iff
                \begin{multline*}
                    \forall_{(x,y) \in S \cross S'} \exists_{\alpha,\beta \in \R_+} :\\
                    \alpha \norm{(x,y)}_2 
                        \leq \norm{(x,y)}_1 
                        \leq \beta \norm{(x,y)}_2\\
                    \alpha (\norm{x}^2 + (\norm{y}')^2)^{1/2}
                        \leq \norm{x} + \norm{y}'
                        \leq \beta (\norm{x}^2 + (\norm{y}')^2)^{1/2}\\
                \end{multline*}
                First, the following demonstrates that $\norm{(x,y)}_2 \leq \norm{(x,y)}_1$
                \begin{align*}
                    \norm{(x,y)}_1^2
                        &= (\norm{x} + \norm{y}')^2\\
                        &= \norm{x}^2 + (\norm{y}')^2 + \norm{x}\norm{y}'\\
                        &\leq \norm{x}^2 + (\norm{y}')^2 + (\norm{x})^2 + (\norm{y}')^2\\
                        &= 2 \norm{x}^2 + 2(\norm{y}')^2\\
                        &= 2 \norm{(x,y)}_2^2\\
                    \frac{1}{2} \norm{(x,y)}_1^2 
                        &\leq \norm{(x,y)}_2^2
                \end{align*}
                Therefore,
                \[
                    \frac{1}{\sqrt{2}} \norm{(x,y)}_2 \leq \norm{(x,y)}_1
                \]
                and this is also true for any $p>2$ as well using an arbitrary number of power expansions.

                Next, from the Cauchy Schwartz's inequality we have 
                \begin{align*}
                    \norm{(x,y)}_1 
                        &= \norm{x} + \norm{y}'\\
                    &= \qty(\norm{x} (1) + \norm{y}' (1))\\
                    &\leq \qty(\norm{x}^2 + (\norm{y}')^2)^{\frac{1}{2}} \qty(1^2 + 1^2)^{1 - \frac{1}{2}}\\
                    &= \qty(2)^\frac{1}{2} \qty(\norm{x}^2 + (\norm{y}')^2)^{\frac{1}{2}}\\
                    &= \sqrt{2} \sqrt{\norm{x}^2 + (\norm{y}')^2}\\
                    \norm{(x,y)}_1
                        &\leq \sqrt{2} \norm{(x,y)}_2
                \end{align*}
                Therefore,
                \[\frac{1}{\sqrt{2}} \norm{x}_2 \leq \norm{x}_1 \leq \sqrt{2} \norm{x}_2\]
                which proves $\norm{\cdot}_1 \sim \norm{\cdot}_2$.
            \end{proof}
        \end{lemma}
        \newpage
        % 1-norm ~ p-norm
        \begin{lemma}\label{lem:2-1toP}
            $\norm{\cdot}_1 \sim \norm{\cdot}_p$
            \begin{proof}
                $\norm{\cdot}_1 \sim \norm{\cdot}_p$ is true iff
                \begin{multline*}
                    \forall_{(x,y) \in S \cross S'} \exists_{\alpha,\beta \in \R_+} :\\
                    \alpha \norm{(x,y)}_p 
                        \leq \norm{(x,y)}_1 
                        \leq \beta \norm{(x,y)}_p\\
                    \alpha \qty(\norm{x}^p + (\norm{y}')^p)^{1/p}
                        \leq \norm{x} + \norm{y}'
                        \leq \beta \qty(\norm{x}^p + (\norm{y}')^p)^{1/p}\\
                \end{multline*}
                From the Holder's inequality we have 
                \begin{align*}
                    \norm{x}_1 = \norm{x} + \norm{y}'\\
                        &= \norm{x}(1) + \norm{y}'(1)\\
                        &\leq \qty(\norm{x}^p + (\norm{y}')^p)^{1/p} \qty(\sum_{i=1}^2 \abs{1}^{(1-p)})^{\frac{1}{1-p}}\\
                        &= n^{\frac{1}{1-p}} \qty(\norm{x}^p + (\norm{y}')^p)^{1/p}\\
                        &= n^{\frac{1}{1-p}} \norm{(x,y)}_p
                \end{align*}
                Therefore,
                \[
                    \norm{(x,y)}_1 \leq n^{\frac{1}{1-p}} \norm{(x,y)}_p
                \]
                and, since $p > 1$, then the remainder of the arguments from Lemma \ref{lem:2-1to2} can be applied here to any arbitrary $p > 1$ to prove the norm equivalence with 1, 2, and any $p$ norms.
            \end{proof}
        \end{lemma}
        %1-norm ~ \infty-norm
        \begin{lemma}\label{lem:2-1toInfty}
            $\norm{\cdot}_1 \sim \norm{\cdot}_\infty$
            \begin{proof}
                $\norm{\cdot}_1 \sim \norm{\cdot}_\infty$ is true iff
                \begin{multline*} 
                    \forall_{(x,y) \in S \cross S'} \exists_{\alpha,\beta \in \R_+} :\\
                    \alpha \norm{(x,y)}_\infty \leq \norm{(x,y)}_1 \leq \beta \norm{(x,y)}_\infty\\
                    \alpha \max\qty{\norm{x} + \norm{y}'} \leq \norm{x} + \norm{y}' \leq \beta \max\qty{\norm{x} + \norm{y}'}\\
                \end{multline*}
                Clearly, this is true for when $\alpha \in (0,1]$. Similarly, when $\beta \geq 2$ then $\max{\norm{x},\norm{y}'}$ and is clearly greater then the $\norm{(x,y)}_1$; therefore $\norm{\cdot}_1 \sim \norm{\cdot}_\infty$.
            \end{proof}
        \end{lemma}
    \end{proof}
\end{theorem}

% Problem 3
\newpage
\section{}
Let $X$ be a vector space and $V$ be a normed space. 
The function $f : X \to V$ is called bounded if $\exists M \st \forall_{x\in X} \implies \norm{f(x)} < M$. 
Consider the set $\mathcal{B}(X,V)$ of all bounded functions from $X \to V$. 

%Part a
\subsection{}
Show that $\mathcal{B}(X,V)$ is a vector space.

\begin{definition}\label{def:vec_space}
    A \emph{Vector space} over a field is the set $V$ along with two operations (vector addition and vector multiplication) satisfying the basic vector properties.
    \begin{enumerate}
        \item Associativity of vector addition
        \[\vb{u} + (\vb{v} + \vb{w}) = (\vb{u} + \vb{v}) + \vb{w}\]
        \item Commutativity of vector addition
        \[\vb{u} + \vb{v} = \vb{v} + \vb{u}\]
        \item Identity element of vector addition (zero vector)
        \[\forall_{\vb{v} \in V} \exists_{\vb{0} \in V} \st \vb{v} + \vb{0} = v \]
        \item Inverse elements of vector addition (additive inverse)
        \[\forall_{\vb{v} \in V} \exists_{-v \in V} \st \vb{v} + (-\vb{v}) = \vb{0}\]
        \item Compatibility of scalar and field multiplication
        \[a (b \vb{v}) = (a b) \vb{v}\]
        \item Identity element of scalar multiplication (multiplicative identity)
        \[\exists_{1 \in F} \vb{1} \vb{v} = v\]
        \item Distributivity of scalar multiplication with vector addition
        \[a (\vb{u} + \vb{v}) = a \vb{u} + a \vb{v}\]
        \item Distributivity of scalar multiplication with field addition
        \[(a + b) \vb{v} = a \vb{v} + b \vb{v}\]
    \end{enumerate}
\end{definition}

\begin{definition}
    $\mathcal{B}(X,V)$ is the set of all functions $f : X \to V$ that are bounded under the definition:
    \[\exists_{M \in V} \st \forall_{x\in X} \implies \norm{f(x)} < M\]
\end{definition}

\begin{theorem}
    $\mathcal{B}(X,V)$ is a vector space.
    \begin{proof}
        It is known that $X$ is a vector space and $V$ is a normed vector space.
        For all functions between $X$ and $V$ the normed space result implies many of the required vector space properties directly.
        For instance, assuming standard function addition and multiplication methods, the mapped results of the new superimposed function will satisfy Associativity, Commutativity, Identity and inverse for addition, Compatibility and identity of multiplication.
        The Distributivity properties require more justification as they do not clearly result from the results of a single function.
        Fortunately, the boundedness of $\mathcal{B}(X,V)$ provides that an upper bound exists for the output and so the complicated parts of accounting for weirder functions allows for a proof of distributivity based on the output and the induced addition and multiplication operations will satisfy all of the superposition properties.
    \end{proof}
\end{theorem}

%Part b
\subsection{}
Show that the function $\norm{\cdot}_\infty : \mathcal{B}(X,V) \to \R_{+} :$
\[
    \norm{f}_\infty := \sup_{x\in X} \norm{f(x)}
\]
defines a norm on $\mathcal{B}(X,V)$.

\begin{definition}
    A \emph{norm} is a function $\norm{\cdot} : V \to \R_{+}$ satisfying
    \begin{enumerate}
        \item Non-negativity 
            \[\forall_{x\in V} \norm{x} \geq 0 \implies \norm{x} = 0 \iff x = 0\]
        \item Homogeneity 
            \[\norm{\lambda \cdot x} = \abs{\lambda} \norm{x}\]
        \item Triangle inequality 
            \[\norm{x + y} \leq \norm{x} + \norm{y}\]
    \end{enumerate}
\end{definition}

\begin{theorem}
    $\norm{\cdot}_\infty$ is a norm on $\mathcal{B}(X,V)$.
    \begin{proof}
        Since $\mathcal{B}(X,V)$ is a vector space, the important properties of a field and simple vector operations can be assumed.

        First, by definition, the non-negativity is satisfied by the mapped results are within $\R_+$.

        Second, the original norm properties from the normed vector space $V$ can be applied to each of the $\norm{f(x)}$ within the $\sup_{x \in X}$, resulting in the homogeneity required for a norm.

        Third, The triangle inequality can also be easily seen with the following:
        \begin{align*}
            \norm{x + y} 
                &\leq \norm{x} + \norm{y}\\
            \sup_{(x + y) \in X} \norm{f(x + y)}
                &\leq \sup{x \in X} \norm{f(x)} + \sup_{y \in X} \norm{f(y)}\\
            \sup_{x,y \in X \st x+y \in X} \norm{f(x + y)}
                &\leq \sup_{x,y \in X} \norm{x} + \norm{y}
        \end{align*}
        which is clearly true considering the definition of the original normed space $V$.
    \end{proof}
\end{theorem}

% Problem 4
\newpage
\section{}
Let $A$ be a dense set in metric space $(S,d)$, let $(V,d_1)$ be a complete metric space, and $f : A \to Y$ be a uniformly continuous function.

Note: assuming that there was a typo and that $Y$ is also dense within $V$.

%Part a
\subsection{Show that if $\{x_n\}$ is a Cauchy sequence in $A$ then $\{f(x_n)\}$ is a Cauchy sequence in $Y$.}

\begin{definition}
    A set $A$ is \emph{dense} within metric space $(S,d)$ if and only if $A = X$.
\end{definition}

\begin{definition}\label{def:complete}
    Metric space $(S,d)$ is called a \emph{complete metric space} if every cauchy sequence $\{a_n\}\subset S$ converges in $S$.
    \[\forall_{\{a_n\} \subset S \st \{a_n\} \ \textnormal{cauchy} \ \implies \exists_{a \in S} \st \lim_{n\to\infty} a_n = a}\]
\end{definition}

\begin{definition}
    $\{a_n\}$ is said to be a \emph{cauchy} sequence if
    \[\{a_n\} : \forall_{\epsilon>0} \exists_{N} : \forall_{n,m > N} \implies d(a_n,a_m) < \epsilon\]
\end{definition}

\begin{definition}
    A function $f : X \to Y$ is said to be \emph{uniformly continuous} if and only if
    \[\forall_{\epsilon>0} \exists_{\delta>0} \forall_{x,x' \in X} d_X(x,x') < \delta \implies d_Y(f(x), f(x')) < \epsilon\]
\end{definition}

\begin{theorem}
    If $\{x_n\}$ is a Cauchy sequence in $A$ then $\{f(x_n)\}$ is a Cauchy sequence in $Y$.
    \begin{proof}
        \begin{align*}
            \{x_n\} \ \text{cauchy} &\implies \{f(x_n)\} \ \text{cauchy}\\
            \forall_{\epsilon,\delta>0} \exists_{N(\epsilon,\delta) \in \N} : \forall_{n,m > N} \implies d(x_n,x_m) < \epsilon)
            &\implies d_1(f(x_{n_1}),f(x_{n_1})) < \epsilon_1)
        \end{align*}
        By definition of the complete metric space $Y$, every cauchy sequence within $Y$ will converge to within $Y$, therefore every cauchy sequence that gets mapped from the dense $A$ to $Y$ via the uniformly continuos function $f$ will be guaranteed to be cauchy as well.
        (this could also be written out in a more complicated way using quantifiers, but the words just explained it better)
    \end{proof}
\end{theorem}

%Part b
\subsection{Show that there is only one continuous function $g : X \to Y$ so that $g(x) = f(x)$ forall $x \in A$.}
\begin{theorem}
    There is only one continuous $g : X \to Y$ so that $g(x) = f(x)$ forall $x \in A$.
    \begin{proof}
        This means that each continuous function from $X$ or $A$ to $Y$ only has one complimentary function that produces the same image as it in $Y$.
        For $f$, the definition of uniformly continuous is
        \[\forall_{\epsilon>0} \exists_{\delta>0} \forall_{x,x' \in A} d(x,x') < \delta \implies d_1(f(x), f(x')) < \epsilon\]
        Similarly, $g$ being continuous is defined by 
        \[\forall_{x \in X} \forall_{\epsilon>0} \exists_{\delta > 0} \forall_{x, x'\in X} d(x,x') < \delta \implies d_1(g(x),g(x')) < \epsilon\]
        There is only one possible mapping for each complete relation between each of the element $x \in A$ to $f(x) \in Y$, so since $A$ is dense in $X$, each element $x \in X$ can be mapped from $X$ to the image in $Y$.
        Although this is simple and the result clearly follows, the quantifiers can be used to demonstrate this as follows:
        \begin{multline*}
            \qty(\forall_{\epsilon>0} \exists_{\delta>0} \forall_{x,x' \in A} d(x,x') < \delta \implies d_1(f(x), f(x')) < \epsilon) \land \\
            \land \qty(\forall_{x \in X} \forall_{\epsilon>0} \exists_{\delta > 0} \forall_{x, x'\in X} d(x,x') < 
            \delta \implies d_1(g(x),g(x')) < \epsilon)\\
            \forall_{\epsilon>0} \exists_{\delta>0} \forall_{x,x' \in A} d(x,x') < \delta \implies d_1(f(x), f(x')) < \epsilon \land \exists_{\epsilon_1(\epsilon,x,x')} d_1(g(x),g(x'))) < \epsilon_1
        \end{multline*}
        And since we want $g(x) = f(x)$, it is clear that the exact same restrictions (for each element across all of $A$ and $X$) will be in place, which implies that only one function will be able to satisfy it.
    \end{proof}
\end{theorem}

% Problem 5
\newpage
\section{}
Let $(L, \norm{\cdot})$ be a Banach space. 
Let $L_0$ be a closed subspace of $L$. 
Define the factor-space $L/L_0$ as $:L_1 := L/L_0 = \qty{x + y \st x \in L, y \in L_0}$. 
In other works, $L_1$ consists of all subsets of $L$ obtained from $L_0$ by shifting all its elements by some element $x$. 

%Part a
\subsection{Show that $L_1$ is a vector space.}
\begin{theorem}
    $L_1$ is a vector space.
    \begin{proof}
        From the definition of a vector space, Definition \ref{def:vec_space}, it is necessary for all vector spaces to have additive and multiplicative operations that satisfy the multiple properties of superposition.
        Directly by definition of the of $L_1$, being composed of a Banach space and a closed subspace within it, it may be possible to directly claim that it is also Banach. Regardless, the definition of each element within $L_1$ as a summation of two elements within a Banach (and therefore vector) space, already demonstrates the additive properties. Similarly, the composition of $L_1$ as a linear combination of elements from $L$ (and $L_0$) that individually satisfy all of the vector space properties will imply that an induced (yet technically undefined in the problem statement) set of additive and multiplicative operations that obey superposition.
    \end{proof}
\end{theorem}

%Part b
\subsection{}
Define the function $\norm{\cdot} : L_1 \to \R_{+}$ as $\norm{x}_1 = \inf_{x - y \in L_0} \norm{y}$. 
Show that this function defines a norm on the space $L_1$.

\begin{theorem}
    $\norm{x}_1$ is a norm on $L_1$.
    \begin{proof}
        Since $L_1$ is a vector space, the important properties of a field and simple vector operations can be assumed.

        First, by definition, the non-negativity is satisfied by the mapped results are within $\R_+$.

        Second, the original norm properties from $L$ apply to each  $\norm{y}$ within the $\inf_{x-y \in L_0}$, resulting in the homogeneity required for a norm.

        Third, The triangle inequality can also be easily seen with the following:
        \begin{align*}
            \norm{(a,b) + (x,y)} 
                &\leq \norm{(a,b)} + \norm{(x,y)}\\
            \inf_{(a+x) - (b+y) \in L_0} \norm{b+y}
                &\leq \inf_{a - b \in L_0} \norm{b} + \inf_{x-y \in L_0} \norm{y}\\
                &\leq \inf_{(a-b), (x-y) \in L_0} \norm{b + y}
        \end{align*}
        which is clearly true considering the definition of the original banach space $L$.
    \end{proof}
\end{theorem}

%Part c
\subsection{Show that $L_1$ is a Banach space.}
\begin{definition}
    A complete, normed, space is called a \emph{Banach space}.
\end{definition}

\begin{theorem}
    $L_1$ is a Banach space.
    \begin{proof}
        From the first two parts of this question, we know that $L_1$ is a vector space, and $(L_1, \norm{x}_1)$ is a normed space.
        The only remaining requirement is that of completeness, which by Definition \ref{def:complete}, means
        \begin{align*}
            \forall\{l_n\} \subset L_1 \st \{l_n\} \ \textnormal{cauchy} \ \implies \exists_{l \in L_1} \st \lim_{n\to\infty} l_n = l
        \end{align*}
        Since $L_1$ is composed of the complete collection of projections of the closed $L_0$ shifted by arbitrary values within $L$, both of which are complete (since they are Banach), so any possible cauchy sequences within $L_1$ will tend towards another element within $L_1$.
    \end{proof}
\end{theorem}




% Problem 6
\newpage
\section{}
Let $C([-1,1])$ be the space of all continuous real-valued functions $f(x)$ with $x \in [-1,1]$. 
Let $\norm{f}_\infty := \sup_{x \in [-1,1]} \abs{f(x)}$. 
Find the distance from point $p = x^{2021}$ to the space $P_{2020}$ of all polynomials of degree less than or equal to 2020.

\begin{definition}
    The \emph{distance} between a point and a set is defined by 
    \[dist(x,A) := \inf_{a \in A} d(x,a) = \inf_{a \in A} \norm{x - a}\]
\end{definition}

The distance between point $p = x^{2021}$ and $P_{2020}$ is calculated as follows:
\begin{align*}
    \textnormal{dist}(p,P) 
        &= \inf_{y \in P_{2020}} d(p,y)
            = \inf_{y \in P_{2020}} \norm{p - y}\\
        &= \inf_{y \in P_{2020}} \sup_{x \in [-1,1]} \abs{p(x) - y(x)}\\
        &= \inf_{y \in P_{2020}} \sup_{x \in [-1,1]} \abs{x^{2021} - \sum_{i=0}^{2020} a_i^{(y)} x^i}
    \intertext{Since the supremum has to bound the maximum value, it can be assumed that the suppremmum of an absolute value will only occur when $a_i^{(y)}<0$ and $x = \max{x \in [-1,1]} = 1$.}
        &= \inf_{y \in P_{2020}} (1)^{2021} - \sum_{i=0}^{2020} -\abs{a_i^{(y)}} (1)^i\\
        &= \inf_{y \in P_{2020}} 1 + \sum_{i=0}^{2020} \abs{a_i^{(y)}}
\end{align*}
We can then see that the largest lower bound would be when $a_i^{(y)} = 0, \ \forall_{i\in [0,2020]}$.

Therefore, \[\textnormal{dist}(p=x^{2021},P_{2020}) = 1\]

\end{document}
