% Standard Article Definition
\documentclass[]{article}

% Page Formatting
\usepackage[margin=1in]{geometry}
\setlength\parindent{0pt}

% Graphics
\usepackage{graphicx}

% Math Packages
\usepackage{physics}
\usepackage{amsmath, amsfonts, amssymb, amsthm}
\usepackage{mathtools}

% Code Def
\usepackage{listings}

% Section Heading Settings
\usepackage{enumitem}
\renewcommand{\theenumi}{\alph{enumi}}
\renewcommand*{\thesection}{Problem \arabic{section}}
\renewcommand*{\thesubsection}{\alph{subsection})}
\renewcommand*{\thesubsubsection}{\quad \quad \roman{subsubsection})}

%Custom Commands
\newcommand{\Rel}{\mathcal{R}}
\newcommand{\R}{\mathbb{R}}
\newcommand{\C}{\mathbb{C}}
\newcommand{\N}{\mathbb{N}}
\newcommand{\Z}{\mathbb{Z}}
\newcommand{\Q}{\mathbb{Q}}

\newcommand{\toI}{\xrightarrow{\textsf{\tiny I}}}
\newcommand{\toS}{\xrightarrow{\textsf{\tiny S}}}
\newcommand{\toB}{\xrightarrow{\textsf{\tiny B}}}

\newcommand{\divisible}{ \ \vdots \ }
\newcommand{\st}{\ : \ }


% Theorem Definition
\newtheorem{definition}{Definition}
\newtheorem{assumption}{Assumption}
\newtheorem{theorem}{Theorem}
\newtheorem{lemma}{Lemma}
\newtheorem{proposition}{Proposition}


%opening

\title{MATH 5301 Elementary Analysis - Homework 8}

\author{Jonas Wagner}

\date{2021, October 29\textsuperscript{th}}

\begin{document}

\maketitle

% Problem 1 ----------------------------------------------
\section{}
Show that the norms $\norm{\cdot}_1$, $\norm{\cdot}_p$ for $p > 1$, and $\norm{\cdot}_\infty$ are equivalent.

\begin{definition}
    For $\norm{\cdot}_a, \norm{\cdot}_b$ on $S$, 
    $\norm{\cdot}_a$ is said to be \emph{stronger} then $\norm{\cdot}_b$ if 
    \[
        \forall \{x_n\} \subset S \st x_n \xrightarrow[d_a]{} x \implies x_n \xrightarrow[d_b]{} x
    \]
\end{definition}
\begin{definition}
    $\norm{\cdot}_a$ and $\norm{\cdot}_b$ are said to be \emph{equivalent},  $\norm{\cdot}_a \sim \norm{\cdot}_b$,
    if $\norm{\cdot}_a$ is stronger then $\norm{\cdot}_b$ 
    and $\norm{\cdot}_b$ is stronger then $\norm{\cdot}_a$. 
    This means that
    \[
        \norm{\cdot}_a \sim \norm{\cdot}_b 
            \iff \exists{\alpha,\beta \in \R_{>0}} : 
            \forall_{x\in S} \alpha \norm{\cdot}_b \leq \norm{\cdot}_a \leq \beta \norm{x}_b
    \]
\end{definition}
\begin{definition} The following norms are defined as
    \begin{enumerate}
        \item $\norm{\cdot}_1 := \norm{x}_1 = \sum_{i=1}^n \abs{x_i} = \abs{x_1} + \abs{x_2} + \dots + \abs{x_n}$
        \item $\norm{\cdot}_2 := \norm{x}_2 = \qty(\sum_{i=1}^n \abs{x_i}^2)^{1/2} = \qty(\abs{x_1}^2 + \abs{x_2}^2 + \dots + \abs{x_n}^2)^{1/2}$
        \item $\norm{\cdot}_p := \norm{x}_p = \qty(\sum_{i=1}^n \abs{x_i}^p)^{1/p} =\qty(\abs{x_1}^p + \abs{x_2}^p + \dots + \abs{x_n}^p)^{1/p}, \ p > 1$
        \item $\norm{\cdot}_\infty := \norm{x}_\infty = \max_{i=1}^n \abs{x_i} = \max(\abs{x_1}, \abs{x_2}, \dots, \abs{x_n})$
    \end{enumerate}
\end{definition}

\newpage
\begin{theorem}
    The norms $\norm{\cdot}_1, \norm{\cdot}_p,$ and $\norm{\cdot}_\infty$ are equivalent.
    \begin{proof}
        % 1-norm ~ p-norm
        \begin{lemma}\label{lem:1-1toP}
            $\norm{\cdot}_1 \sim \norm{\cdot}_p$
            \begin{proof}
                $\norm{\cdot}_1 \sim \norm{\cdot}_p$ is true iff
                \begin{multline*}
                    \forall_{x} \exists_{\alpha,\beta \in \R_+} :\\
                    \alpha \norm{x}_p 
                        \leq \norm{x}_1 
                        \leq \norm{x}_p\\
                    \alpha \qty(\sum_{i=1}^n \abs{x_i}^p)^{1/p}
                        \leq \sum_{i=1}^n \abs{x_i}
                        \leq \beta \qty(\sum_{i=1}^n \abs{x_i}^p)^{1/p}\\
                \end{multline*}
                From the Holder's inequality we have 
                \begin{align*}
                    \norm{x}_1 = \sum_{i=1}^n \abs{x_i}
                        &= \sum_{i=1}^n \abs{x_i} (1)\\
                        &\leq \qty(\sum_{i=1}^n \abs{x_i}^p)^{1/p}
                            \qty(\sum_{i=1}^n \abs{1}^{(1-p)})^{1/(1-p)}\\
                        &\leq n^{1/(1-p)} \qty(\sum_{i=1}^n \abs{x_i}^p)^{1/p} 
                \end{align*}
                So for $0 < \alpha \leq n^{1/(1-p)}$ and $\beta \geq n^{1/(1-p)}$,
                \begin{align*}
                    &\alpha \qty(\sum_{i=1}^n \abs{x_i}^p)^{1/p}
                        &&\leq \sum_{i=1}^n \abs{x_i}
                        &&\leq \beta \qty(\sum_{i=1}^n \abs{x_i}^p)^{1/p}\\
                    &\qty(\sum_{i=1}^n \abs{x_i}^p)^{1/p}
                        &&\leq n^{1/(1-p)} \qty(\sum_{i=1}^n \abs{x_i}^p)^{1/p} 
                        &&\leq n^{1/(1-p)} \qty(\sum_{i=1}^n \abs{x_i}^p)^{1/p}
                \end{align*}
                Therefore,
                \[\norm{x}_p \leq \norm{x}_1 \leq n^{\frac{1}{1-p}} \norm{x}_p\]
                which proves $\norm{\cdot}_1 \sim \norm{\cdot}_p$.
            \end{proof}
        \end{lemma}
        % \newpage
        %1-norm ~ \infty-norm
        \begin{lemma}\label{lem:1-1toInfty}
            $\norm{\cdot}_1 \sim \norm{\cdot}_\infty$
            \begin{proof}
                $\norm{\cdot}_1 \sim \norm{\cdot}_\infty$ is true iff
                \begin{multline*} 
                    \forall_{x} \exists_{\alpha,\beta \in \R_+} :\\
                    \alpha \norm{x}_\infty \leq \norm{x}_1 \leq \beta \norm{x}_\infty\\
                    \alpha \max_{i=1}^n \abs{x_i} \leq \sum_{i=1}^n \abs{x_i} \leq \beta \max_{i=1}^n \abs{x_i}\\
                \end{multline*}
                Clearly, this is true for when $\alpha \in (0,1]$. Similarly, when $\beta \geq n$ then $\sum_{i=1}^n \max_{i=1}^n \abs{x_i}$ and then clearly greater then the $\norm{x}_1$; therefore $\norm{\cdot}_1 \sim \norm{\cdot}_\infty$.
            \end{proof}
        \end{lemma}
        From, Lemma \ref{lem:1-1toP} and Lemma \ref{lem:1-1toInfty}, it is clear that $\forall_{p > 1}$:
        \[
            \norm{x}_\infty 
            \leq \norm{x}_p 
            \leq \norm{x}_1 
            \leq n^{1/{1-p}} \norm{x}_p 
            \leq n \norm{x}_\infty
        \]
        Therefore, $\norm{\cdot}_1 \sim \norm{\cdot}_p \sim \norm{\cdot}_\infty$ ($\forall_{p > 1}$).
    \end{proof}
\end{theorem}

% Problem 2
\newpage
\section{}
Let $(S, \norm{\cdot})$ and $(S', \norm{\cdot}')$ to be two normed spaces. 
Show that the following norms on $S \cross S'$ are equivalent.
\begin{enumerate}
    \item $\norm{(x,y)}_1 = \norm{x} + \norm{y}'$
    \item $\norm{(x,y)}_2 = \sqrt{\norm{x}^2 + (\norm{y}')^2}$
    \item $\norm{(x,y)}_p = \qty(\norm{x}^p + (\norm{y}')^p)^{1/p}$
    \item $\norm{(x,y)}_\infty = \max\qty{\norm{x} + \norm{y}'}$
\end{enumerate}

\begin{theorem}
    The norms $\norm{\cdot}_1,\norm{\cdot}_2,\norm{\cdot}_p,$ and $\norm{\cdot}_\infty$ are all equivalent on $S \cross S'$.
    \begin{proof}
        % 1-norm ~ 2-norm
        \begin{lemma}\label{lem:2-1to2}
            $\norm{\cdot}_1 \sim \norm{\cdot}_2$
            \begin{proof}
                $\norm{\cdot}_1 \sim \norm{\cdot}_2$ is true iff
                \begin{multline*}
                    \forall_{(x,y) \in S \cross S'} \exists_{\alpha,\beta \in \R_+} :\\
                    \alpha \norm{(x,y)}_2 
                        \leq \norm{(x,y)}_1 
                        \leq \beta \norm{(x,y)}_2\\
                    \alpha (\norm{x}^2 + (\norm{y}')^2)^{1/2}
                        \leq \norm{x} + \norm{y}'
                        \leq \beta (\norm{x}^2 + (\norm{y}')^2)^{1/2}\\
                \end{multline*}
                First, the following demonstrates that $\norm{(x,y)}_2 \leq \norm{(x,y)}_1$
                \begin{align*}
                    \norm{(x,y)}_1^2
                        &= (\norm{x} + \norm{y}')^2\\
                        &= \norm{x}^2 + (\norm{y}')^2 + \norm{x}\norm{y}'\\
                        &\leq \norm{x}^2 + (\norm{y}')^2 + (\norm{x})^2 + (\norm{y}')^2\\
                        &= 2 \norm{x}^2 + 2(\norm{y}')^2\\
                        &= 2 \norm{(x,y)}_2^2\\
                    \frac{1}{2} \norm{(x,y)}_1^2 
                        &\leq \norm{(x,y)}_2^2
                \end{align*}
                Therefore,
                \[
                    \frac{1}{\sqrt{2}} \norm{(x,y)}_2 \leq \norm{(x,y)}_1
                \]
                and this is also true for any $p>2$ as well using an arbitrary number of power expansions.

                Next, from the Cauchy Schwartz's inequality we have 
                \begin{align*}
                    \norm{(x,y)}_1 
                        &= \norm{x} + \norm{y}'\\
                    &= \qty(\norm{x} (1) + \norm{y}' (1))\\
                    &\leq \qty(\norm{x}^2 + (\norm{y}')^2)^{\frac{1}{2}} \qty(1^2 + 1^2)^{1 - \frac{1}{2}}\\
                    &= \qty(2)^\frac{1}{2} \qty(\norm{x}^2 + (\norm{y}')^2)^{\frac{1}{2}}\\
                    &= \sqrt{2} \sqrt{\norm{x}^2 + (\norm{y}')^2}\\
                    \norm{(x,y)}_1
                        &\leq \sqrt{2} \norm{(x,y)}_2
                \end{align*}
                Therefore,
                \[\frac{1}{\sqrt{2}} \norm{x}_2 \leq \norm{x}_1 \leq \sqrt{2} \norm{x}_2\]
                which proves $\norm{\cdot}_1 \sim \norm{\cdot}_2$.
            \end{proof}
        \end{lemma}
        \newpage
        % 1-norm ~ p-norm
        \begin{lemma}\label{lem:2-1toP}
            $\norm{\cdot}_1 \sim \norm{\cdot}_p$
            \begin{proof}
                $\norm{\cdot}_1 \sim \norm{\cdot}_p$ is true iff
                \begin{multline*}
                    \forall_{(x,y) \in S \cross S'} \exists_{\alpha,\beta \in \R_+} :\\
                    \alpha \norm{(x,y)}_p 
                        \leq \norm{(x,y)}_1 
                        \leq \beta \norm{(x,y)}_p\\
                    \alpha \qty(\norm{x}^p + (\norm{y}')^p)^{1/p}
                        \leq \norm{x} + \norm{y}'
                        \leq \beta \qty(\norm{x}^p + (\norm{y}')^p)^{1/p}\\
                \end{multline*}
                From the Holder's inequality we have 
                \begin{align*}
                    \norm{x}_1 = \norm{x} + \norm{y}'\\
                        &= \norm{x}(1) + \norm{y}'(1)\\
                        &\leq \qty(\norm{x}^p + (\norm{y}')^p)^{1/p} \qty(\sum_{i=1}^2 \abs{1}^{(1-p)})^{\frac{1}{1-p}}\\
                        &= n^{\frac{1}{1-p}} \qty(\norm{x}^p + (\norm{y}')^p)^{1/p}\\
                        &= n^{\frac{1}{1-p}} \norm{(x,y)}_p
                \end{align*}
                Therefore,
                \[
                    \norm{(x,y)}_1 \leq n^{\frac{1}{1-p}} \norm{(x,y)}_p
                \]
                and, since $p > 1$, then the remainder of the arguments from Lemma \ref{lem:2-1to2} can be applied here to any arbitrary $p > 1$ to prove the norm equivalence with 1, 2, and any $p$ norms.
            \end{proof}
        \end{lemma}
        %1-norm ~ \infty-norm
        \begin{lemma}\label{lem:2-1toInfty}
            $\norm{\cdot}_1 \sim \norm{\cdot}_\infty$
            \begin{proof}
                $\norm{\cdot}_1 \sim \norm{\cdot}_\infty$ is true iff
                \begin{multline*} 
                    \forall_{(x,y) \in S \cross S'} \exists_{\alpha,\beta \in \R_+} :\\
                    \alpha \norm{(x,y)}_\infty \leq \norm{(x,y)}_1 \leq \beta \norm{(x,y)}_\infty\\
                    \alpha \max\qty{\norm{x} + \norm{y}'} \leq \norm{x} + \norm{y}' \leq \beta \max\qty{\norm{x} + \norm{y}'}\\
                \end{multline*}
                Clearly, this is true for when $\alpha \in (0,1]$. Similarly, when $\beta \geq 2$ then $\max{\norm{x},\norm{y}'}$ and is clearly greater then the $\norm{(x,y)}_1$; therefore $\norm{\cdot}_1 \sim \norm{\cdot}_\infty$.
            \end{proof}
        \end{lemma}
    \end{proof}
\end{theorem}

% Problem 3
\newpage
\section{}
Let $X$ be a vector space and $V$ be a normed space. 
The function $f : X \to V$ is called bounded if $\exists M \st \forall_{x\in X} \implies \norm{f(x)} < M$. 
Consider the set $\mathcal{B}(X,V)$ of all bounded functions from $X \to V$. 

%Part a
\subsection{}
Show that $\mathcal{B}(X,V)$ is a vector space.

\begin{definition}
    A \emph{Vector space} over a field is the set $V$ along with two operations (vector addition and vector multiplication) satisfying the basic vector properties.
    \begin{enumerate}
        \item Associativity of vector addition
        \[\vb{u} + (\vb{v} + \vb{w}) = (\vb{u} + \vb{v}) + \vb{w}\]
        \item Commutativity of vector addition
        \[\vb{u} + \vb{v} = \vb{v} + \vb{u}\]
        \item Identity element of vector addition (zero vector)
        \[\forall_{\vb{v} \in V} \exists_{\vb{0} \in V} \st \vb{v} + \vb{0} = v \]
        \item Inverse elements of vector addition (additive inverse)
        \[\forall_{\vb{v} \in V} \exists_{-v \in V} \st \vb{v} + (-\vb{v}) = \vb{0}\]
        \item Compatibility of scalar and field multiplication
        \[a (b \vb{v}) = (a b) \vb{v}\]
        \item Identity element of scalar multiplication (multiplicative identity)
        \[\exists_{1 \in F} \vb{1} \vb{v} = v\]
        \item Distributivity of scalar multiplication with vector addition
        \[a (\vb{u} + \vb{v}) = a \vb{u} + a \vb{v}\]
        \item Distributivity of scalar multiplication with field addition
        \[(a + b) \vb{v} = a \vb{v} + b \vb{v}\]
    \end{enumerate}
\end{definition}

\begin{definition}
    $\mathcal{B}(X,V)$ is the set of all functions $f : X \to V$ that are bounded under the definition:
    \[\exists_{M \in V} \st \forall_{x\in X} \implies \norm{f(x)} < M\]
\end{definition}

\begin{theorem}
    $\mathcal{B}(X,V)$ is a vector space.
    \begin{proof}
        It is known that $X$ is a vector space and $V$ is a normed vector space.
        For all functions between $X$ and $V$ the normed space result implies many of the required vector space properties directly.
        For instance, assuming standard function addition and multiplication methods, the mapped results of the new superimposed function will satisfy Associativity, Commutativity, Identity and inverse for addition, Compatibility and identity of multiplication.
        The Distributivity properties require more justification as they do not clearly result from the results of a single function.
        Fortunately, the boundedness of $\mathcal{B}(X,V)$ provides that an upper bound exists for the output and so the complicated parts of accounting for weirder functions allows for a proof of distributivity based on the output and the induced addition and multiplication operations will satisfy all of the superposition properties.
    \end{proof}
\end{theorem}

%Part b
\subsection{}
Show that the function $\norm{\cdot}_\infty : \mathcal{B}(X,V) \to \R_{+} :$
\[
    \norm{f}_\infty := \sup_{x\in X} \norm{f(x)}
\]
defines a norm on $\mathcal{B}(X,V)$.

\begin{definition}
    A \emph{norm} is a function $\norm{\cdot} : V \to \R_{+}$ satisfying
    \begin{enumerate}
        \item Non-negativity 
            \[\forall_{x\in V} \norm{x} \geq 0 \implies \norm{x} = 0 \iff x = 0\]
        \item Homogeneity 
            \[\norm{\lambda \cdot x} = \abs{\lambda} \norm{x}\]
        \item Triangle inequality 
            \[\norm{x + y} \leq \norm{x} + \norm{y}\]
    \end{enumerate}
\end{definition}

\begin{theorem}
    $\norm{\cdot}_\infty$ is a norm on $\mathcal{B}(X,V)$.
    \begin{proof}
        Since $\mathcal{B}(X,V)$ is a vector space, the important properties of a field and simple vector operations can be assumed.

        First, by definition, the non-negativity is satisfied by the mapped results are within $\R_+$.

        Second, the original norm properties from the normed vector space $V$ can be applied to each of the $\norm{f(x)}$ within the $\sup_{x \in X}$, resulting in the homogeneity required for a norm.

        Third, The triangle inequality can also be easily seen with the following:
        \begin{align*}
            \norm{x + y} 
                &\leq \norm{x} + \norm{y}\\
            \sup_{(x + y) \in X} \norm{f(x + y)}
                &\leq \sup{x \in X} \norm{f(x)} + \sup_{y \in X} \norm{f(y)}\\
            \sup_{x,y \in X \st x+y \in X} \norm{f(x + y)}
                &\leq \sup_{x,y \in X} \norm{x} + \norm{y}
        \end{align*}
        which is clearly true considering the definition of the original normed space $V$.
    \end{proof}
\end{theorem}

% Problem 4
\newpage
\section{}
Let $A$ be a dense set in metric space $(S,d)$, let $(V,d_1)$ be a complete metric space, and $f : A \to Y$ be a uniformly continuous function. 

%Part a
\subsection{Show that if $\{x_n\}$ is a Cauchy sequence in $A$ then $\{f(x_n)\}$ is a Cauchy sequence in $Y$.}




%Part b
\subsection{Show that there is only one continuous function $g : X \to Y$ so that $g(x) = f(x)$ forall $x \in $.}














% Problem 5
\newpage
\section{}
Let $(L, \norm{\cdot})$ be a Banach space. 
Let $L_0$ be a closed subspace of $L$. 
Define the factor-space $L/L_0$ as $l_1 := L/L_0 = \qty{x + y \st x \in L, y \in L_0}$. 
In other works, $L_1$ consists of all subsets of $L$ obtained from $L_0$ by shifting all its elements by some element $x$. 

%Part a
\subsection{Show that $L_1$ is a vector space.}


%Part b
\subsection{}
Define the function $\norm{\cdot} : L_1 \to \R_{+}$ as $\norm{x}_1 = \inf_{x - y \in L_0} \norm{y}$. 
Show that this function defines a norm on the space $L_1$.


%Part c
\subsection{Show that $L_1$ is a Banach space.}







% Problem 6
\newpage
\section{}
Let $C([-1,1])$ be the space of all continuous real-valued functions $f(x)$ with $x \in [-1,1]$. 
Let $\norm{f}_\infty := \sup_{x \in [-1,1]} \abs{f(x)}$. 
Find the distance from point $p = x^{2021}$ to the space $P_{2020}$ of all polynomials of degree less than or equal to 2020.





\end{document}
