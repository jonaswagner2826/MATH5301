\documentclass[]{article}

\usepackage{graphicx}

\usepackage[margin=1in]{geometry}

\setlength\parindent{0pt}

\usepackage{physics}
\usepackage{amsmath, amsfonts, amssymb, amsthm}

\usepackage{listings}

\usepackage{enumitem}
\renewcommand{\theenumi}{\alph{enumi}}
\renewcommand*{\thesection}{Problem \arabic{section}}
\renewcommand*{\thesubsection}{\alph{subsection})}
\renewcommand*{\thesubsubsection}{\quad \quad \roman{subsubsection})}

%Custom Commands
\newcommand{\Rel}{\mathcal{R}}
\newcommand{\R}{\mathbb{R}}
\newcommand{\C}{\mathbb{C}}
\newcommand{\N}{\mathbb{N}}
\newcommand{\Z}{\mathbb{Z}}
\newcommand{\Q}{\mathbb{Q}}

\newcommand{\toI}{\xrightarrow{\textsf{\tiny I}}}
\newcommand{\toS}{\xrightarrow{\textsf{\tiny S}}}
\newcommand{\toB}{\xrightarrow{\textsf{\tiny B}}}

\newcommand{\divisible}{ \ \vdots \ }


% Theorem Definition
\newtheorem{definition}{Definition}
\newtheorem{theorem}{Theorem}


%opening

\title{MATH 5301 Elementary Analysis - Homework 4}

\author{Jonas Wagner}

\date{2021, September 24}

\begin{document}

\maketitle

% Problem 1
\section{}
Use the axioms of the ordered field, prove the following:
% Part a
\subsection{$(a > c) \land (b > d) \implies a + b > c + d$}
\begin{align*}
    (a > c) \land (b > d) &\implies (a + b) > (c + d)
\end{align*}
From (O3):
\begin{align*}
    (a > c) &\implies ((a + b) \geq (b + c)) \land ((a + d) \geq (c + d))\\
    (b > d) &\implies ((a + b) \geq (a + d)) \land ((b + c) \geq (c + d))\\
\end{align*}
From (02):
\begin{align*}
    ((a + b) \geq (b + c)) \land ((b + c) \geq (c + d))
        &\implies (a + b) > (c + d)
\end{align*}

% Part b
\subsection{$(a > c > 0) \land (b > d > 0) \implies ab > cd > 0$}
\begin{align*}
    (a > c > 0) \land (b > d > 0) &\implies ab > cd > 0
\end{align*}
From (O4):
\begin{align*}
    (a > c > 0) \land (b > 0) &\implies ab > bc > 0\\
    % (a > c > 0) \land (d > 0) &\implies ad > cd > 0\\
    % (b > d > 0) \land (a > 0) &\implies ab > ad > 0\\
    (b > d > 0) \land (c > 0) &\implies bc > cd > 0
\end{align*}
From (O2):
\begin{align*}
    (ab > bc > 0) \land (bc > cd > 0) &\implies ab > cd > 0
\end{align*}

\newpage
% Part c
\subsection{$a > b > 0 \implies \frac{1}{a} < \frac{1}{b}$}
\begin{align*}
    a > b > 0 &\implies \frac{1}{b} < \frac{1}{b}
\end{align*}
From 
\begin{align*}
    a > 0 & \implies a^{-1} > 0\\
    b > 0 & \implies b^{-1} > 0
\end{align*}
From (O4):
\begin{align*}
    (a > b > 0) \land (a^{-1} > 0) &\implies a a^{-1} = 1 > b a^{-1} = \frac{b}{a}> 0\\
    (a > b > 0) \land (b^{-1} > 0) &\implies a b^{-1} = \frac{a}{b} > b b^{-1} = 1 > 0\\
    (\frac{a}{b} > 1 > 0) \land (a^{-1} > 0)
        &\implies \frac{a}{b} a^{-1} = \frac{1}{b} > (1)(a^{-1}) = \frac{1}{a} > 0
\end{align*}
Therefore, $$\frac{1}{a} < \frac{1}{b}$$

\newpage
% Part d
\subsection{
    Let,
    $$\abs*{x} = 
        \begin{cases}
            x, & x \geq 0\\
            -x,& X < 0
        \end{cases}
    $$
    prove,
    $$\forall a, b \implies \abs*{a - b} \geq \abs*{\abs*{a} - \abs*{b}}$$
}

\begin{align*}
    \forall a, b \implies \abs*{a - b} \geq \abs*{\abs*{a} - \abs*{b}}
\end{align*}
When $a>b>0$ (or $b>a>0$),
\begin{align*}
    \abs*{a - b}&= a - b\\
    \abs*{a}    &= a\\
    \abs*{b}    &= b\\
    \abs*{\abs*{a} - \abs*{b}} &= a - b\\
    \abs*{a - b} = a-b &= \abs*{\abs*{a} - \abs*{b}}
\end{align*}
The same is true for $0<a<b$ and $0 < b < a$ by similar arguments.\\
For $a > 0 > b$,
\begin{align*}
    \abs*{a}    &= a\\
    \abs*{b}    &= -b\\
    \abs*{a - b}&= \abs*{a} + \abs*{b}\\
    \abs*{a} - \abs*{b} &= a - (-b) = a + b\\
    \abs*{\abs*{a} - \abs*{b}} &= 
        \begin{cases}
            \abs*{a} - \abs*{b}   & \abs*{a} > \abs*{b}\\
            \abs*{b} - \abs*{a}   & \abs*{a} < \abs*{b}
        \end{cases}
    \intertext{From (03):}
    \abs*{a - b} = \abs*{a} + \abs*{b} &\geq \abs*{a} - \abs*{b}\\
    \abs*{a - b} = \abs*{a} + \abs*{b} &\geq \abs*{b} - \abs*{a}\\
    \therefore \ \abs*{a - b} &\geq \abs*{\abs*{a} - \abs*{b}}
\end{align*}
Therefore $\forall a, b$,
\begin{align*}
     \abs*{a - b} \geq \abs*{\abs*{a} - \abs*{b}}
\end{align*}


% Problem 2
\newpage
\section{}
Determine which of the axioms satisfied by the set of real numbers 
are not satisfied by the following set:

% Part a
\subsection{Set $\Q$ of all rational numbers.}
Set $\Q$ of rational numbers can be an ordered field, 
$\langle \Q, +, 0, \cdots, 1\rangle$, but lacks (C) completeness:
$$\forall A \subset \Q \not{\exists} c \in \Q : c = \sup A$$

% Part b
\subsection{Set $\Q(\sqrt{2})$ of all numbers of form $a + b\sqrt{2}$, where $a,b\in\Q$}
Set $\Q := \qty{a + b \sqrt{2} : a, b \in \Q}$ can be an ordered field, 
$\langle \Q(\sqrt{2}), +, 0, \cdots, 1\rangle$, but lacks completeness (C):
$$\forall A \subset \Q(\sqrt{2}) \not{\exists} c \in \Q : c = \sup A$$

% Part c
\subsection{Set $\C$ of all pairs of real numbers $(a,b)$ 
    with addition ${(a,b) + (c,d) = (a+c,b+d)}$, 
    multiplication $(a,b)\cdot(c,d) = (ac-bd,ad+bc)$, 
    and ordered relation $(a,b)<(c,d)\iff(b \leq d)\land((b=d \lor a<c))$.
}
Set $\C := \qty{(a,b) : a,b \in \R}$ can satisfy the field conditions, 
$\langle \C, +, 0, \cdots, 1\rangle$,
but it is not ordered because it does not satisfy (O1).

% Problem 3
\newpage
\section{}
Using the method of mathematical induction, 
prove the following statements: $(n \in \N)$

% Part a
\subsection{Bernoulli inequality: $\forall n\in\N, \ \forall x > -1$,
    $(1+x)^n \geq 1 + nx$
}

\begin{theorem}
    $\forall n\in\N, \ \forall x > -1$, 
        $$(1+x)^n \geq 1 + nx$$
\end{theorem}

\begin{proof}
    Proof by induction:\\
    For $n=1$,
    \begin{align*}
        (1+x)^n &\geq 1 + nx\\
        (1+x)^1 &\geq 1 + (1)x\\
        1+x &\geq 1 + x
    \end{align*}
    For $n > 1$,
    \begin{align*}
        (1+x)^n &\geq 1 + n x\\
        (1+x)^n (1+x) &\geq (1+nx) (1+x)\\
        (1+x)^{n+1} &\geq (1 + x + nx + nx^2)\\
            &\geq 1 + (n+1)x + nx^2\\
    \end{align*}
    Since $n \geq 2 \implies nx^2 > 0$
    \begin{align*}    
        1 + (n+1)x + nx^2 &\geq 1 + (n+1)x\\
    \end{align*}
    From (O2):
        $$(1+x)^{n+1} \geq 1 + (n+1)x$$
    Therefore $\forall n > 1$,
        $$(1+x)^n \geq 1 + nx \implies (1+x)^{n+1} \geq 1 + (n+1)x$$
    Therefore $\forall n\in\N, \ \forall x > -1$,
        $$(1+x)^n \geq 1 + nx$$
\end{proof}

\newpage
% Part b
\subsection{For $n \in \N$, 
    $\frac{1}{2} + \frac{2}{2^2} + \cdots + \frac{n}{2^n} = 2 - \frac{n+2}{2^n}$
}
\begin{theorem}
    For $n \in \N$, 
        $$\frac{1}{2} + \frac{2}{2^2} + \cdots + \frac{n}{2^n} = 2 - \cfrac{n+2}{2^n}$$
\end{theorem}
\begin{proof}
    Proof by induction:
    For $n = 1$,
    \begin{align*}
        \frac{1}{2} + \frac{2}{2^2} + \cdots + \frac{n}{2^n} 
            &= 2 - \cfrac{n+2}{2^n}\\
        \frac{1}{2} = 2 - \cfrac{1+2}{2^1} 
            &= 2 - \frac{3}{2}\\
        \frac{1}{2} &= \cfrac{1}{2}\\
    \end{align*}
    For $n > 1$,
    $$\frac{1}{2} + \frac{2}{2^2} + \cdots + \frac{n}{2^n} = 2 - \cfrac{n+2}{2^n}
    \implies \frac{1}{2} + \frac{2}{2^2} + \cdots + \frac{n}{2^{n+1}} = 2 - \cfrac{n+2}{2^{n+1}}$$
    \begin{align*}
        \frac{1}{2} + \frac{2}{2^2} + \cdots + \frac{n}{2^n} 
            &= 2 - \cfrac{n+2}{2^n}\\
        \frac{1}{2} + \frac{2}{2^2} + \cdots + \frac{n}{2^n} + \frac{n+1}{2^{n+1}}
            &= 2 - \cfrac{n+2}{2^n} + \cfrac{n+1}{2^{n+1}}\\
            &= 2 - \cfrac{n+2}{2^n}\cfrac{2}{2} + \cfrac{n+1}{2^{n+1}}\\
            &= 2 - \cfrac{2(n+2)}{2^{n+1}} + \cfrac{n+1}{2^{n+1}}\\
            &= 2 + \cfrac{n+1 - 2(n+2) }{2^{n+1}}\\
            &= 2 + \cfrac{n+1 - 2n - 2}{2^{n+1}}\\
            &= 2 + \cfrac{-n - 1}{2^{n+1}}\\
            &= 2 + \cfrac{-(n+1) -2}{2^{n+1}}\\
            &= 2 - \cfrac{(n+1) +2}{2^{n+1}}\\
    \end{align*}
    Therefore $\forall n > 1$,
    $$\frac{1}{2} + \frac{2}{2^2} + \cdots + \frac{n}{2^n}
        = 2 - \cfrac{n+2}{2^n}
        \implies \frac{1}{2} + \frac{2}{2^2} + \cdots + \frac{n}{2^n} + \frac{n+1}{2^{n+1}}
        = 2 - \cfrac{(n+1) +2}{2^{n+1}}$$
    Therefore For $n \in \N$,
    $$\frac{1}{2} + \frac{2}{2^2} + \cdots + \frac{n}{2^n} = 2 - \frac{n+2}{2^n}$$
\end{proof}

\newpage
% Part c
\subsection{For $q \in \R, n \in \N$,
    $(1+q)(1+q^2)(1+q^4)\cdots(1+q^{2^n}) = \frac{1-q^{2^{n+1}}}{1-q}$
}
\begin{theorem}
    For $q \in \R, n \in \N$,
    $$(1+q)(1+q^2)(1+q^4)\cdots(1+q^{2^n}) = \cfrac{1-q^{2^{n+1}}}{1-q}$$
\end{theorem}
\begin{proof}
    Proof by induction:\\
    For $n=1$,
    \begin{align*}
        (1+q)(1+q^2)(1+q^4)\cdots(1+q^{2^n})
            &= \cfrac{1-q^{2^{n+1}}}{1-q}\\
        (1+q)(1+q^{2^1}) 
            &= \cfrac{1-q^{2^{1+1}}}{1-q}\\
        (1+q^{2} + q + q^{3}) 
            &= \cfrac{1-q^{4}}{1-q}\\
        (1 + q + q^{2} + q^{3}) (1-q)
            &= \cfrac{1-q^{4}}{1-q}(1-q)\\
        1 + q + q^{2} + q^{3} - q - q^2 - q^3 -q^4
            &= 1-q^{4}\\
        1 - q^4
            &= 1-q^{4}\\
    \end{align*}
    For $n>1$
    \begin{align*}
        (1+q)(1+q^2)(1+q^4)\cdots(1+q^{2^n}) &= \cfrac{1-q^{2^{n+1}}}{1-q}\\
        (1+q)(1+q^2)(1+q^4)\cdots(1+q^{2^n})(1+q^{2^{n+1}})
            &= \cfrac{1-q^{2^{n+1}}}{1-q}(1+q^{2^{n+1}})\\
            &= \cfrac{\qty(1-q^{2^{n+1}})\qty(1+q^{2^{n+1}})}{1-q}\\
            &= \cfrac{1-q^{2^{n+1}} + q^{2^{n+1}} + \qty(-q^{2^{n+1}})\qty(q^{2^{n+1}})}{1-q}\\
            &= \cfrac{1 -q^{2^{n+1} + 2^{n+1}})}{1-q}\\
            &= \cfrac{1 -q^{2\qty(2^{n+1})}}{1-q}\\
        (1+q)(1+q^2)(1+q^4)\cdots(1+q^{2^n})(1+q^{2^{n+1}})
            &= \cfrac{1 -q^{2^{n+2}}}{1-q}\\
    \end{align*}
    Therefore $\forall n>1$,
    $$(1+q)(1+q^2)(1+q^4)\cdots(1+q^{2^n}) = \cfrac{1-q^{2^{n+1}}}{1-q}
    \implies (1+q)(1+q^2)(1+q^4)\cdots(1+q^{2^n})(1+q^{2^{n+1}}) = \cfrac{1 -q^{2^{n+2}}}{1-q}
    $$
    Therefore $\forall n \in \N$,
    $$(1+q)(1+q^2)(1+q^4)\cdots(1+q^{2^n}) = \cfrac{1-q^{2^{n+1}}}{1-q}$$
\end{proof}

\newpage
% Part d
\subsection{For $n \in \N$,
    $1^3 + 3^3 + \cdots + (2n + 1)^3 = (n+1)^2 (2n^2 + 4n + 1)$
}
\begin{theorem}
    For $n \in \N$,
    $$1^3 + 3^3 + \cdots + (2n + 1)^3 = (n+1)^2 (2n^2 + 4n + 1)$$
\end{theorem}
\begin{proof}
    Proof by induction:\\
    For $n = 1$,
    \begin{align*}
        1^3 + 3^3 + \cdots + (2n + 1)^3 &= (n+1)^2 (2n^2 + 4n + 1)\\
        1^3 + (2(1) + 1)^3 &= ((1)+1)^2 (2(1)^2 + 4(1) + 1)\\
        1^3 + 3^3 &= (2)^2 (2(1) + 4 + 1)\\
        1 + 27 &= (4) (2 + 4 + 1)\\
        28 &= (4)(7)\\
        28 &= (4)(7)\\
        28 &= 28
    \end{align*}
    For $n > 1$,
    \begin{align*}
        1^3 + 3^3 + \cdots + (2n + 1)^3 &= (n+1)^2 (2n^2 + 4n + 1)\\
        1^3 + 3^3 + \cdots + (2n + 1)^3 + (2(n+1)+1)^3
            &= (n+1)^2 (2n^2 + 4n + 1) + (2(n+1)+1)^3\\
            &= (n+1)^2 (2n^2 + 4n + 1) + (2n + 3)^3
    \end{align*}
    \begin{align*}
        &= (n+1)(n+1) (2n^2 + 4n + 1) + (2n + 3)(2n + 3)(2n + 3)\\
        &= (n^2 + 2n + 1) (2n^2 + 4n + 1) + 27 + 54 n + 36 n^2 + 8 n^3\\
        &= 2n^4 + 8n^3 + 11n^2 + 6n + 1 + 8n^3 + 36 n^2 + 54n + 27\\
        &= 2n^4 + 16n^3 + 47n^2 + 60n + 28\\
        &= (n+2)^2(2n^2 + 8n + 7)\\
        &= ((n+1)+1)^2 (2(n+1)^2 - 4n - 2 + 4(n+1) + 4n - 4 + 7)\\
        &= ((n+1)+1)^2 (2(n+1)^2 + 4(n+1) + 1)\\
    \end{align*}
    Therefore $\forall n > 1$,
    $$
    1^3 + 3^3 + \cdots + (2n + 1)^3 = (n+1)^2 (2n^2 + 4n + 1)\implies
    $$
    $$\implies 1^3 + 3^3 + \cdots + (2n + 1)^3 + (2(n+1)+1)^3
    = ((n+1)+1)^2 (2(n+1)^2 + 4(n+1) + 1)
    $$
    Therefore $\forall n \in \N$,
    $$1^3 + 3^3 + \cdots + (2n + 1)^3 = (n+1)^2 (2n^2 + 4n + 1)$$
\end{proof}

\newpage
% Part e
\subsection{For $n,k \in \N$,
    $\sum_{k=0}^{n} \qty(-1)^k \cfrac{n!}{k!(n-k)!} = 0, 
    \sum_{k=0}^{n} \cfrac{n!}{k!(n-k)!} = 2^n$
}
\begin{definition}
    The factorial of a number, $n!$, is defined as
    $$n! := (1)(2)(3)\cdots(n-1)(n)$$
\end{definition}
\begin{definition}
    The combination of two numbers, $\mqty(n\\k)$, is defined as
    $$\mqty(n\\k) := \cfrac{n!}{k!(n-k)!}$$
\end{definition}
\subsubsection{$\sum_{k=0}^{n} \qty(-1)^k \cfrac{n!}{k!(n-k)!} = 0$}
\begin{theorem}
    For $n,k \in \N$,
    $$\sum_{k=0}^{n} \qty(-1)^k \mqty(n\\k) = 0$$
\end{theorem}
\begin{proof}
    For $n=1$,
    \begin{align*}
        \sum_{k=0}^{n} \qty(-1)^k \mqty(n\\k) &= 0\\
        \sum_{k=0}^{1} \qty(-1)^k \mqty(1\\k) 
            &= (-1)^0 \mqty(1\\0) + (-1)^1 \mqty(1\\1)\\
            &= (1)(1) + (-1) (1)\\
            &=0
    \end{align*}
    Therefore,
    \begin{align*}
        \sum_{k=0}^{1} \qty(-1)^k \mqty(1\\k) &= 0
    \end{align*}
    For $n>1$,
    \begin{align*}
        \sum_{k=0}^{n} \qty(-1)^k \mqty(n\\k) &= 0\\
        \sum_{k=0}^{n} \qty(-1)^k \frac{n!}{k!(n-k)!} &= 0\\
        \qty(-1)^0 \frac{n!}{0!(n-0)!} + \qty(-1)^1 \frac{n!}{1!(n-1)!} + \cdots +
            \qty(-1)^n \frac{n!}{n!(n-n)!} &= 0\\
        (1)\frac{n!}{0!n!} + (-1) \frac{n!}{1!(n-1)!} + (1)\frac{n!}{(2!(n-2)!} + \cdots +
            \qty(-1)^{n-2} \frac{n!}{(n-2)!2!} + \qty(-1)^{n-1} \frac{n!}{(n-1)!1!} + 
            \qty(-1)^n \frac{n!}{n!0!}\\
        (1)\frac{n!}{n!} + (-1) \frac{n!}{(n-1)!} + (1)\frac{n!}{(n-2)!2} + \cdots +
        \qty(-1)^{n-2} \frac{n!}{(n-2)!2} + \qty(-1)^{n-1} \frac{n!}{(n-1)!} 
        + \qty(-1)^n \frac{n!}{n!}&=0\\
    \end{align*}
    multiply by $\frac{(n+1)}{(n+1)}$ and add another add/subtract set...
\end{proof}

\newpage
\subsubsection{$\sum_{k=0}^{n} \cfrac{n!}{k!(n-k)!} = 2^n$}
\begin{theorem}
    For $n,k \in \N$,
    $$\sum_{k=0}^{n} \mqty(n\\k) = 2^n$$
\end{theorem}
\begin{proof}
    By induction:\\
    For $n=1$,
    \begin{align*}
        test
    \end{align*}
\end{proof}












% Problem 4
\newpage
\section{}
Show that $\forall n \in \N, n \geq 2$,

% Part a
\subsection{
    $\frac{1}{\sqrt{1}} + \frac{1}{\sqrt{2}} + \cdots + \frac{1}{\sqrt{n}} > \sqrt{n}$
}
\begin{theorem}
    For $n \in \N$ and $n \geq 2$,
    $\frac{1}{\sqrt{1}} + \frac{1}{\sqrt{2}} + \cdots + \frac{1}{\sqrt{n}} > \sqrt{n}$
\end{theorem}
\begin{proof}
    By Induction:\\
    For $n=2$,
    \begin{align*}
        \frac{1}{\sqrt{1}} + \frac{1}{\sqrt{2}} + \cdots + \frac{1}{\sqrt{n}} &> \sqrt{n}\\
    \end{align*}
\end{proof}



% Part b
\subsection{
    $$\cfrac{1}{n+1} + \cfrac{1}{n+2} + \cdots + \frac{1}{3n+1} > 1$$
}





% Part c
\subsection{
    $$\qty(\cfrac{n+1}{2})^n > n!$$
}





% Part d
\subsection{
    $$2^{2^n} - 6 \divisible 10$$
}






% Problem 5
\newpage
\section{}

% Part a
\subsection{Show that $\sqrt{2} \notin \Q$}
\begin{definition}
    $\sqrt{2} := x > 0 : x^2 = 2$
\end{definition}
\begin{theorem}
    $\sqrt{2} \notin \Q$
\end{theorem}
\begin{proof}
    Assume $\sqrt{2} \in \Q$,
    $$\sqrt{2} \in \Q \implies \exists m,n \in \N : \frac{m}{n} = \sqrt{2}$$
    Also assume that $m,n$ are coprime. (i.e) $\gcd(m,n)=1$\\
    Let $m = \sqrt{2} n$,
    \begin{align*}
        m = \sqrt{2} n \implies m^2 = 2 n^2 \implies m^2 \divisible 2 \implies m \divisible 2\\
        m \divisible 2 \implies \exists k \in \N : m = 2k \implies m^2 = (2k)^2 = 4 k^2\\
        4k^2 = 2 n^2 \implies 2k^2 = n^2 \implies n^2 \divisible 2 \implies n \divisible 2
    \end{align*}
    This is false becouse with $\gcd(m,n)=1$, $m$ and $n$ cannot both be even.
\end{proof}

% Part b
\subsection{Show that 
    $\forall a,b \in \Q, a < b \implies \exists x \in \R \backslash \Q : a < x < b$
}
\begin{theorem}
    $\forall a,b \in \Q, a < b \implies \exists x \in \R \backslash \Q : a < x < b$
\end{theorem}
\begin{proof}
    idk
\end{proof}





% Part c
\subsection{Show that
    $\forall a,b \in \R \backslash \Q, a < b \implies \exists x \in Q : a < x < b$
}
\begin{theorem}
    $\forall a,b \in \R \backslash \Q, a < b \implies \exists x \in Q : a < x < b$
\end{theorem}
\begin{proof}
    idk
\end{proof}






\newpage
% Problem 6
\section{}

% Part a
\subsection{}
look at original doc....

























\end{document}
