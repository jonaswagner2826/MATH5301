\documentclass[]{article}

\usepackage{graphicx}

\usepackage[margin=1in]{geometry}

\setlength\parindent{0pt}

\usepackage{physics}
\usepackage{amsmath, amsfonts, amssymb, amsthm}

\usepackage{listings}

\usepackage{enumitem}
\renewcommand{\theenumi}{\alph{enumi}}
\renewcommand*{\thesection}{Problem \arabic{section}}
\renewcommand*{\thesubsection}{\alph{subsection})}
\renewcommand*{\thesubsubsection}{\quad \quad \roman{subsubsection})}

%Custom Commands
\newcommand{\Rel}{\mathcal{R}}
\newcommand{\R}{\mathbb{R}}
\newcommand{\C}{\mathbb{C}}
\newcommand{\N}{\mathbb{N}}
\newcommand{\Z}{\mathbb{Z}}
\newcommand{\Q}{\mathbb{Q}}

\newcommand{\toI}{\xrightarrow{\textsf{\tiny I}}}
\newcommand{\toS}{\xrightarrow{\textsf{\tiny S}}}
\newcommand{\toB}{\xrightarrow{\textsf{\tiny B}}}

\newcommand{\divisible}{ \ \vdots \ }


% Theorem Definition
\newtheorem{definition}{Definition}
\newtheorem{assumption}{Assumption}
\newtheorem{theorem}{Theorem}


%opening

\title{MATH 5301 Elementary Analysis - Homework 4}

\author{Jonas Wagner}

\date{2021, September 24}

\begin{document}

\maketitle

% Problem 1
\section{}
Let $(S_1,d_1)$ and $(S_2,d_2)$ be two metric spaces. 
Show that each of the following determines the metric on $S_1 \cross S_2$.\\
Let $x_j \in S_1, y_j \in S_2$:
% Part a
\subsection{$d((x_1,y_1),(x_2,y_2)) = \max\qty{d_1(x_1,x_2),d_2(y_1,y_2)}$}

\begin{theorem}
    The metric $$d((x_1,y_1),(x_2,y_2)) = \max\qty{d_1(x_1,x_2),d_2(y_1,y_2)}$$ 
    is a metric on $S_1 \cross S_2$.
\end{theorem}
\begin{proof}
    A metric $d: S_1 \cross S_2 \to \R$ must satisfy (i) non-negativity, 
    (ii) Symmetry, and (iii) Triangle Inequality.
    \subsubsection{Non-negativy $$d((x_1,y_1), (x_2,y_2)) \geq 0$$}
        Since $d_1(x_1,x_2) \geq 0$ and $d_2(y_1,y_2) \geq 0$,
        $$d((x_1,y_1), (x_2,y_2)) = \max\qty{d_1(x_1,x_2),d_2(y_1,y_2)} \geq 0$$
    \subsubsection{Symmetry $$d((x_1,y_1),(x_2,y_2))=d((x_1,y_1),(x_2,y_2))$$}
    \begin{align*}
        d((x_1,y_1), (x_2,y_2)) = \max\qty{d_1(x_1,x_2),d_2(y_1,y_2)} 
        &= \max\qty{d_2(y_1,y_2),d_1(x_1,x_2)} = d((x_2,y_2),(x_1,x_2)
    \end{align*}
    \subsubsection{Triangle Inequality 
    $$d((x_1,y_1),(x_3,y_3)) \leq d((x_1,y_1),(x_2,y_2)) + d((x_2,y_2),(x_3,y_3))$$
    }
    \begin{align*}
        d((x_1,y_1),(x_3,y_3)) &= \max\qty{d_1(x_1,x_3), d_2(y_1,y_3)}\\
        d((x_1,y_1),(x_2,y_2)) &= \max\qty{d_1(x_1,x_2), d_2(y_1,y_2)}\\
        d((x_2,y_2),(x_3,y_3)) &= \max\qty{d_1(x_2,x_3), d_2(y_2,y_3)}\\
        \max\qty{d_1(x_1,x_3), d_2(y_1,y_3)} 
            &\leq \max\qty{d_1(x_1,x_2), d_2(y_1,y_2)}
            + \max\qty{d_1(x_2,x_3), d_2(y_2,y_3)}\\
        d((x_1,y_1),(x_3,y_3)) &\leq d((x_1,y_1),(x_2,y_2)) + d((x_2,y_2),(x_3,y_3))
    \end{align*}
\end{proof}

\newpage
% Part b
\subsection{$d((x_1,y_1),(x_2,y_2)) = d_1(x_1,x_2)  + d_2(y_1,y_2)$}

\begin{theorem}
    The metric $$d((x_1,y_1),(x_2,y_2)) = d_1(x_1,x_2)  + d_2(y_1,y_2)$$
    is a metric on $S_1 \cross S_2$.
\end{theorem}
\begin{proof}
    A metric $d: S_1 \cross S_2 \to \R$ must satisfy (i) non-negativity, 
    (ii) Symmetry, and (iii) Triangle Inequality.
    \subsubsection{Non-negativy $$d((x_1,y_1), (x_2,y_2)) \geq 0$$}
        Since $d_1(x_1,x_2) \geq 0$ and $d_2(y_1,y_2) \geq 0$,
        $$d((x_1,y_1), (x_2,y_2)) = d_1(x_1,x_2)  + d_2(y_1,y_2) \geq 0$$
    \subsubsection{Symmetry $$d((x_1,y_1),(x_2,y_2))=d((x_1,y_1),(x_2,y_2))$$}
    \begin{align*}
        d((x_1,y_1), (x_2,y_2)) = d_1(x_1,x_2)  + d_2(y_1,y_2) 
        &= d_1(x_2,x_1)  + d_2(y_2,y_1) = d((x_2,y_2),(x_1,x_2)
    \end{align*}
    \subsubsection{Triangle Inequality 
    $$d((x_1,y_1),(x_3,y_3)) \leq d((x_1,y_1),(x_2,y_2)) + d((x_2,y_2),(x_3,y_3))$$
    }
    \begin{align*}
        d((x_1,y_1),(x_3,y_3)) &= d_1(x_1,x_3)  + d_2(y_1,y_3)\\
        d((x_1,y_1),(x_2,y_2)) &= d_1(x_1,x_2)  + d_2(y_1,y_2)\\
        d((x_2,y_2),(x_3,y_3)) &= d_1(x_2,x_3)  + d_2(y_2,y_3)\\
        d_1(x_1,x_3)  + d_2(y_1,y_3)
            &\leq d_1(x_1,x_2)  + d_2(y_1,y_2)
            + d_1(x_2,x_3)  + d_2(y_2,y_3)\\
        d((x_1,y_1),(x_3,y_3)) &\leq d((x_1,y_1),(x_2,y_2)) + d((x_2,y_2),(x_3,y_3))
    \end{align*}
\end{proof}

\newpage
% Part c
\subsection{$d((x_1,y_1),(x_2,y_2)) = \sqrt{(d_1(x_1,x_2))^2+(d_2(y_1,y_2))^2}$}

\begin{theorem}
    The metric $$d((x_1,y_1),(x_2,y_2)) = \sqrt{(d_1(x_1,x_2))^2+(d_2(y_1,y_2))^2}$$
    is a metric on $S_1 \cross S_2$.
\end{theorem}
\begin{proof}
    A metric $d: S_1 \cross S_2 \to \R$ must satisfy (i) non-negativity, 
    (ii) Symmetry, and (iii) Triangle Inequality.
    \subsubsection{Non-negativy $$d((x_1,y_1), (x_2,y_2)) \geq 0$$}
        Since $d_1(x_1,x_2) \geq 0$ and $d_2(y_1,y_2) \geq 0$,
        $$d((x_1,y_1), (x_2,y_2)) = \sqrt{(d_1(x_1,x_2))^2+(d_2(y_1,y_2))^2} \geq 0$$
    \subsubsection{Symmetry $$d((x_1,y_1),(x_2,y_2))=d((x_1,y_1),(x_2,y_2))$$}
    \begin{align*}
        d((x_1,y_1), (x_2,y_2)) = \sqrt{(d_1(x_1,x_2))^2+(d_2(y_1,y_2))^2}
        &= \sqrt{(d_1(x_2,x_1))^2+(d_2(y_2,y_1))^2} = d((x_2,y_2),(x_1,x_2)
    \end{align*}
    \subsubsection{Triangle Inequality 
    $$d((x_1,y_1),(x_3,y_3)) \leq \sqrt{(d_1(x_1,x_3))^2+(d_2(y_1,y_3))^2}$$
    }
    \begin{align*}
        d((x_1,y_1),(x_3,y_3)) &= \sqrt{(d_1(x_1,x_3))^2+(d_2(y_1,y_3))^2}\\
        d((x_1,y_1),(x_2,y_2)) &= \sqrt{(d_1(x_1,x_2))^2+(d_2(y_1,y_2))^2}\\
        d((x_2,y_2),(x_3,y_3)) &= \sqrt{(d_1(x_2,x_3))^2+(d_2(y_2,y_3))^2}\\
        \sqrt{(d_1(x_1,x_3))^2+(d_2(y_1,y_3))^2}
            &\leq \sqrt{(d_1(x_1,x_2))^2+(d_2(y_1,y_2))^2}
            + \sqrt{(d_1(x_2,x_3))^2+(d_2(y_2,y_3))^2}\\
        d((x_1,y_1),(x_3,y_3)) &\leq d((x_1,y_1),(x_2,y_2)) + d((x_2,y_2),(x_3,y_3))
    \end{align*}
\end{proof}


\newpage
% Problem 2
\section{}
% Part a
\subsection{A set $A$ in the metric space $(S, d)$ is called bounded, if
    $\exists_{R>0} \land \exists x \in S : A \subset B_R(x)$.
    Prove that if $A$ is unbounded then there exists a sequence 
    $\{x_n\} \subset A$ such that $\forall_{m,n\in\N} \implies d(x_n,x_m) > 1$.
}
\begin{assumption}
    $m \neq n$, $m > n$
\end{assumption}
\begin{definition}
    The open ball set $B_r(x)$ over metric space $(S,d)$ is defined as
    $$B_r(x) := \qty{y \in S : d(x,y) < r}$$
\end{definition}
\begin{definition}
    A set $A$ in the metric space $(S, d)$ is called \underline{bounded}, if
    $$\exists_{R>0} \land \exists_{x \in S} : A \subset B_R(x)$$
\end{definition}
\begin{definition}
    A set $A$ in the metric space $(S, d)$ is called \underline{unbounded}, 
    if it is not bounded, (i.e.)
    $$\forall_{R>0} \land \forall_{x \in S} : A \not \subset  B_R(x)$$
\end{definition}
\begin{theorem}
    If $A \in (S,d)$ is unbounded, then 
    $$\exists \{x_n\} \subset A : \forall_{m,n\in\N} \implies d(x_n,x_m) > 1$$
\end{theorem}
\begin{proof}
    $A \in (S,d)$ being unbounded means that 
    $$\forall_{R>0} \land \forall x \in S : A \not\subset  B_R(x)$$
    Since $\forall_{R>0} \land \forall_{x \in S} : A \not \subset  B_R(x)$ 
    and $B_R(x) := \qty{y \in S : d(x,y) < R}$, 
    $$\forall_{x \in A} \exists_{y \in A} : d(x,y) \geq R$$
    Since it is true for any $R > 0$ a sequence $\{x_n\}$ can be constructed 
    with subsequent $x_{n+1}$ so that $d(x_n,x_{n+1})>R=1$.
\end{proof}

% \newpage
% Part b
\subsection{Show that in the normed space $(V,\abs{\cdot})$ the open unit ball 
    $B_r = \qty{x \in V : \abs{x} < 1}$ is a convex set. (i.e)
    $\forall_{x,y\in B_r}, \forall_{t\in \qty[0,1]} \implies t x + (1-t) y \in B_r$
}
\begin{definition}
    The \underline{open unit ball} is defined as: 
    $$B_r := \qty{x \in V : \abs{x} < 1}$$
\end{definition}
\begin{definition}
    The set $A$ is convex if and only if 
    $$\forall_{x,y\in A}, \forall_{t\in \qty[0,1]} \implies t x + (1-t) y \in A$$
\end{definition}
\begin{theorem}
    The open unit ball set $B_r$ is convex in $(V,\abs{\cdot})$.
\end{theorem}
\begin{proof}
    \begin{align*}
        \forall_{x,y\in B_r}, \forall_{t\in \qty[0,1]}
            &\implies t x + (1-t) y \in B_r\\
        \forall_{x,y\in V} : (\abs{x} < 1) \land (\abs{y} < 1), \forall_{t\in \qty[0,1]}
            &\implies t x + (1-t) y \in V : \abs{t x + (1-t) y}
        \intertext{Since
        $\forall_{x,y \in V}, \forall_{t \in \qty[0,1]} \implies t x + (1-t) y \in V$
        }
        \qty(\abs{x} < 1 \land \abs{y} < 1 \implies \abs{t x + (1-t) y} < 1)
            &\iff t x + (1-t) y \in B_r
    \end{align*}
    Clearly,
    $$\forall_{x,y\in V} \abs{x},\abs{y}<1, \forall_{t\in[0,1]} t x + (1-t) y < 1$$
    Therefore, the open unit ball set $B_r$ is convex in $(V,\abs{\cdot})$.
\end{proof}

\newpage
% Problem 3
\section{}
For $(\R^2 = (x,y),d=\sqrt{x^2 + y^2})$,
% Part a
\subsection{Show that $D = \qty{(x,y) : x^2 + y^2 \leq 1}$ is a closed set.}
\begin{definition}
    The set $A \subset V$ is called \underline{open} if 
    $$\forall_{x\in A} \exists_{\epsilon>0} : B_\epsilon(x)\subset A$$
\end{definition}
\begin{definition}
    The set $A \subset V$ is called \underline{closed} if $A^c$ is open.
\end{definition}
\begin{theorem}
    The set $D = \qty{(x,y) : x^2 + y^2 \leq 1}$ is closed.
\end{theorem}
\begin{proof}
    By definition, $D$ is closed iff $D^c$ is open.\\
    $D^c$ is defined by
    $$D^c = \qty{(x,y) : x^2 + y^2 > 1}$$
    By definition, $D^c$ is open if
    $$\forall_{(x,y) \in D^c} \exists_{\epsilon>0}: B_\epsilon ((x,y)) \subset D^c$$
    This means that every element in $D^c$ must have an associated open ball set centered 
    at that element with a positive radius that is fully contained by $D^c$.\\
    Let $(x_b,y_b) \in B_\epsilon ((x,y))$ for $\epsilon>0$.
    This means
    $$d((x,y),(x_b,y_b)) = \sqrt{x_b^2 + y_b^2} < \epsilon$$
    By definition,
    $$(x,y) \in D^c \implies x^2 + y^2 > 1$$
    and therefore,
    $$\sqrt{x^2 + y^2} = d((0,0),(x,y)) > 1$$

    From the triangle inequality, we have
    \begin{align*}
        d((x_b,y_b),(0,0)) &\leq d((0,0),(x,y)) + d((x,y),(x_b,y_b))\\
        d((x,y),(x_b,y_b)) &\geq d((x_b,y_b),(0,0)) - d((0,0),(x,y))\\
        d((x,y),(x_b,y_b)) = \epsilon &> d((x_b,y_b),(0,0)) - 1 > 0
    \end{align*}
    Therefore $D^c$ is open and therefore $D$ is closed.
\end{proof}

\newpage
% Part b
\subsection{Find the infinite collection of open sets 
$\{A_n\}$ so that}% $\bigcap_{n} A_n = \overline{B_1(0)}$} 

$$\qty{A_n : \bigcap_n A_n = \overline{B_1(0)}}$$
This means that the intersection of all sets in $\{A_n\}$ 
is the closure of the unit ball set.

\begin{definition}
    The \underline{interier} of set $A$ in $(S,d)$ is the union of all open sets 
    contained within $A$. (i.e.)
    $$\text{int}(A) = \qty{x \in A : \exists_{\epsilon>0} B_\epsilon(x) \subset A}$$
\end{definition}

\begin{definition}
    The \underline{closure} of set $A$ in $(S,d)$ is the intersection of all closed sets 
    containing $A$, (i.e) 
    $$\overline{A} = \qty{x \in S : 
        \forall_{\epsilon> 0} B_\epsilon(x) \cap A \neq \emptyset}$$
\end{definition}
Note: 
The interior and closures are complementary sets. (i.e.) $\overline{A} = (\text{int}(A))^c$


$B_1(0)$ is defined as 
$$B_1(0) := \qty{(x,y) \in \R^2 : d((0,0),(x,y)) < 1} = \qty{(x,y) : \sqrt{x^2 + y^2} < 1}$$

The closure of $B_1(0)$, $\overline{B_1(0)}$ is defined by
\begin{align*}
    \overline{B_1(0)} &:= \qty{(x,y) \in \R^2 : 
        \forall_{\epsilon > 0} B_\epsilon((x,y)) \cap B_1(0) \neq \emptyset}\\
    &= \qty{(x,y) \in \R^2 : \forall_{\epsilon> 0} 
        \exists_{(x_b,y_b) \in \R^2} (d((x,y),(x_b,y_b)) < \epsilon) \land (d((0,0),(x,y)) < 1)}
\end{align*}

Therefore,
\begin{align*}
    {A_n} &:= \qty{A_n \subset \R^2: \forall_{(x,y) \in \R^2} \forall_{\epsilon> 0} 
        B_\epsilon((x,y)) \cap B_1(0) \neq \emptyset \implies (x,y) \in A}\\
    &= \qty{A \subset \R^2: \qty(\forall_{(x,y) \in \R^2} 
        \exists_{(x_b,y_b) \in \R^2} d((x,y),(x_b,y_b)) < \epsilon \implies d((0,0),(x,y)) < 1)}\\
    &= \qty{A \subset \R^2: \qty(\forall_{(x,y) \in \R^2} 
        \exists_{(x_b,y_b) \in \R^2} \sqrt{(x-x_b)^2 + (y-y_b)^2} < \epsilon 
        \implies \sqrt{x^2 + y^2} < 1)}\\
\end{align*}

\newpage
% Problem 4
\section{}
Let $S = \R^2$. Are the following sets open or closed within the metrics below?
\begin{align*}
    A &= \qty{(x,y) : x^2 + y^2 < 1}\\
    B &= \qty{(x,y) : x = 0 \land -1 \leq y \leq 1}\\
    C &= \qty{(x,y) : 1 < x < 2 \land -1 \leq y \leq 1}\\
    D &= \qty{(x,y) : \abs{x} + \abs{y} < 2}\\
    E &= \qty{(x,y) : x^2 - y^2 < 1 \land \abs{x} + \abs{y} < 4}
\end{align*}

% Part a
\subsection{Euclidean Metric: 
$d((x_1,y_1),(x_2,y_2)) = \sqrt{(x_1-x_2)^2 + (y_1-y_2)^2}$% = \norm{(x_1,y_1)-(x_2,y_2)}_2$
}
\subsubsection{$A = \qty{(x,y) : x^2 + y^2 < 1}$}
Open
\subsubsection{$B = \qty{(x,y) : x = 0 \land -1 \leq y \leq 1}$}
Closed
\subsubsection{$C = \qty{(x,y) : 1 < x < 2 \land -1 \leq y \leq 1}$}
Neither
\subsubsection{$D = \qty{(x,y) : \abs{x} + \abs{y} < 2}$}
Open
\subsubsection{$E = \qty{(x,y) : x^2 - y^2 < 1 \land \abs{x} + \abs{y} < 4}$}
Open

\subsection{Manhattan Metric: 
$$d((x_1,y_1),(x_2,y_2)) = \abs{x_1-y_2} + \abs{y_1-y_2}$$% = \norm{(x_1,y_1)-(x_2,y_2)}_1$$
}
\subsubsection{$A = \qty{(x,y) : x^2 + y^2 < 1}$}
Open
\subsubsection{$B = \qty{(x,y) : x = 0 \land -1 \leq y \leq 1}$}
Closed
\subsubsection{$C = \qty{(x,y) : 1 < x < 2 \land -1 \leq y \leq 1}$}
Neither
\subsubsection{$D = \qty{(x,y) : \abs{x} + \abs{y} < 2}$}
Open
\subsubsection{$E = \qty{(x,y) : x^2 - y^2 < 1 \land \abs{x} + \abs{y} < 4}$}
Open


\subsection{Highway Metric:}
\begin{definition}
    The highway metric is defined as 
    \begin{displaymath}
        d_h((x_1,y_1),(x_2,y_2)) := 
        \begin{cases}
            \abs{y_1 - y_2}, &x_1 = x_2\\
            \abs{y_1} + \abs{y_2} + \abs{x_1 - x_2}, &x_1 \neq x_2
        \end{cases}
    \end{displaymath}
\end{definition}

\subsubsection{$A = \qty{(x,y) : x^2 + y^2 < 1}$}
Neither
\subsubsection{$B = \qty{(x,y) : x = 0 \land -1 \leq y \leq 1}$}
Neither
\subsubsection{$C = \qty{(x,y) : 1 < x < 2 \land -1 \leq y \leq 1}$}
Neither
\subsubsection{$D = \qty{(x,y) : \abs{x} + \abs{y} < 2}$}
Open
\subsubsection{$E = \qty{(x,y) : x^2 - y^2 < 1 \land \abs{x} + \abs{y} < 4}$}
Neither



\newpage
% Problem 5
\section{}
Let $(S,d)$ be a metric space.
% Part a
\subsection{Show that for all $A \subset B \subset S$ 
one has $\text{int}(A)\subseteq \text{int}(B)$ and $\overline{A}\subseteq\overline{B}$. 
Also provide an example of non-strictness.}

\begin{theorem}
    For the metric space $(S,d)$, and $\forall A \subset B \subset S$ the following are true: 
    \subsubsection{$\text{int}(A) \subseteq \text{int}(B)$}
    \begin{proof}
        \begin{align*}
            \text{int}(A) &= \qty{x \in A : \exists_{\epsilon>0} B_\epsilon(x) \subset A}\\
            &= \qty{x \in A : 
                \exists_{\epsilon>0} \forall_{x_b \in S} d(x,x_b)<\epsilon 
                \land x_b \in A}\\
            \text{int}(B) &= \qty{x \in B : \exists_{\epsilon>0} B_\epsilon(x) \subset B}\\
            &= \qty{x \in B : 
                \exists_{\epsilon>0} \forall_{x_b \in S} d(x,x_b)<\epsilon 
                \land x_b \in B}\\
        \end{align*}
        
        Since $(\text{int}(A) \subset A) \land (\text{int}(B) \subset B) \land (A \subset B)$,
        \begin{gather*}
            \qty{x \in A\subset B : 
                \exists_{\epsilon>0} \forall_{x_b \in S} d(x,x_b)<\epsilon 
                \land x_b \in A \subset B}\\
            \forall x \in \qty{x \in A : 
                \exists_{\epsilon>0} \forall_{x_b \in S} d(x,x_b)<\epsilon 
                    \land x_b \in A} 
                \implies x \in \qty{x \in B : 
                    \exists_{\epsilon>0} \forall_{x_b \in S} d(x,x_b)<\epsilon 
                        \land x_b \in B}
        \end{gather*}
        Therefore,
        $$\text{int}(A) \subseteq \text{int}(B)$$\\
        This cannot be a strict inequality becouse it would be equal when $A = B$.
    \end{proof}
    \subsubsection{$\overline{A} \subseteq \overline{B}$}
    \begin{proof}
        \begin{align*}
            \overline{A} &= \qty{x \in S : 
                \forall_{\epsilon> 0} B_\epsilon(x) \cap A \neq \emptyset}\\
            &= \qty{x \in S : 
                \forall_{\epsilon> 0} \exists_{x_b \in S} : d(x,x_b) < \epsilon 
                \land x_b \in A}\\
            \overline{B} &= \qty{x \in S : 
                \forall_{\epsilon> 0} B_\epsilon(x) \cap B \neq \emptyset}\\
            &= \qty{x \in S : 
                \forall_{\epsilon> 0} \exists_{x_b \in S} : d(x,x_b) < \epsilon 
                \land x_b \in B}\\
        \end{align*}

        Since $(\overline{A} \subset A) \land (\overline{A} \subset B) \land (A \subset B)$,
        \begin{gather*}
            \qty{x \in S : 
                \forall_{\epsilon> 0} \exists_{x_b \in S} : d(x,x_b) < \epsilon 
                \land x_b \in A \subset B}\\
            \forall x \in \qty{x \in S : 
                \forall_{\epsilon> 0} \exists_{x_b \in S} : d(x,x_b) < \epsilon 
                    \land x_b \in A}
                \implies x \in \qty{x \in S : 
                    \forall_{\epsilon> 0} \exists_{x_b \in S} : d(x,x_b) < \epsilon 
                        \land x_b \in B}
        \end{gather*}
        Therefore,
        $$\overline{A} \subseteq \overline{B}$$
        This cannot be a strict inequality becouse it would be equal if $A = B$.
    \end{proof}
\end{theorem}

\newpage
% Part b
\subsection{Is the following true:
$\text{int}(A\cup B) = \text{int}(A) \cup \text{int}(B)$?}

\begin{align*}
    \text{int}(A \cup B) &= 
        \qty{x \in A : \exists_{\epsilon>0} B_\epsilon(x) \subset A \cup B}\\
    &= \qty{x \in A \cup B : 
        \exists_{\epsilon>0} \forall_{x_b \in S} d(x,x_b)<\epsilon 
        \land (x_b \in A \lor x_b \in B)}\\
    &= \qty{x \in A \cup B : 
            \exists_{\epsilon>0} \forall_{x_b \in S} d(x,x_b)<\epsilon 
            \land x_b \in A}\\
        &\ \cup \qty{x \in A \cup B : 
            \exists_{\epsilon>0} \forall_{x_b \in S} d(x,x_b)<\epsilon 
            \land x_b \in B}\\
    &= \qty{x \in A: \exists_{\epsilon>0} \forall_{x_b \in S} d(x,x_b)<\epsilon 
            \land x_b \in A}\\
        &\ \cup \qty{x \in A: 
            \exists_{\epsilon>0} \forall_{x_b \in S} d(x,x_b)<\epsilon 
            \land x_b \in B}\\
        &\ \cup \qty{x \in B : 
            \exists_{\epsilon>0} \forall_{x_b \in S} d(x,x_b)<\epsilon 
            \land x_b \in A}\\
        &\ \cup \qty{x \in B : 
            \exists_{\epsilon>0} \forall_{x_b \in S} d(x,x_b)<\epsilon 
            \land x_b \in B}\\
    &= \text{int}(A) \cup \text{int}(B)\\
        &\ \cup \qty{x \in A: 
            \exists_{\epsilon>0} \forall_{x_b \in S} d(x,x_b)<\epsilon 
            \land x_b \in B}\\
        &\ \cup \qty{x \in B : 
            \exists_{\epsilon>0} \forall_{x_b \in S} d(x,x_b)<\epsilon 
            \land x_b \in B}
\end{align*}
Therefore, $\text{int}(A\cup B) \subseteq \text{int}(A) \cup \text{int}(B)$ 
and $\text{int}(A\cup B) = \text{int}(A) \cup \text{int}(B)$ is not true.

% Part c
\subsection{Is the following true:
$\overline{A\cap B} = \overline{A} \cap \overline{B}$?}
\begin{align*}
    \overline{A \cap B} &= \qty{x \in S : 
                \forall_{\epsilon> 0} B_\epsilon(x) \cap (A \cap B) 
                \neq \emptyset}\\
    &= \qty{x \in S :  \forall_{\epsilon> 0} \exists_{x_b \in S} : 
        d(x,x_b) < \epsilon \land x_b \in A \cap B}\\
    &= \qty{x \in S : \forall_{\epsilon> 0} \exists_{x_b \in S} : 
        d(x,x_b) < \epsilon \land x_b \in A \land x_b \in B}\\
    &= \qty{x \in S : \forall_{\epsilon> 0} \exists_{x_b \in S} : 
            d(x,x_b) < \epsilon \land x_b \in A}\\
        &\ \cap \qty{x \in S : \forall_{\epsilon> 0} \exists_{x_b \in S} : 
            d(x,x_b) < \epsilon \land x_b \in B}\\
    &= \overline{A} \cap \overline{B}
\end{align*}
Therefore, $\overline{A\cap B} = \overline{A} \cap \overline{B}$ is true.

\newpage
% Problem 6
\section{}
Give a topological proof of the infinitude of the set of prime numbers. (H. Furstenberg, 1955)\\
Denote $N_{a,b} := \{a + nb : b \in \Z\} \subset \Z$. Define the topology on $\Z$ as follows: 
The set $U$ will be called open if for any $a \in U$ there exists $b \in \Z$ so that 
$N_{a,b} \in U$. Note that every open set is infinite.

\begin{definition}
    $$N_{a,b} := \{a + nb : b \in \Z\} \subset \Z$$
\end{definition}
\begin{definition}
    The set $U$ will be called open if
    $$\forall_{a\in U} \exists_{b\in\Z} : N_{a,b} \subset U$$
\end{definition}

% Part a
\subsection{Show that it is indeed a topology.}
(i.e): any union of open sets is open and any finite intersection of open sets is open.
\subsubsection{$\emptyset$ and $\Z$ are open sets.}
\subsubsection{Any union of open sets is an open set}
\begin{align*}
    \forall_{a\in U} \exists_{b\in\Z} : N_{a,b} \in U\\
    \forall_{a\in U} \exists_{b\in\Z} : \{a + nb : b \in \Z\} \subset Z \subset U
\end{align*}
Trivially, it can be seen that $\{U_i\}_{i\in I}$ open $\implies \bigcup_{i\in I} U_i$ open.
\subsubsection{Finite intersections is open. (i.e.) $U_1, U_2$ open $\implies U_1 \cap U_2$ open.}
$$x \in U_1 \cap U2 \implies \exists_{a_1,a_2\in S} N_{a_1,x} \subset U_1 \land N_{a_2,x} \subset U_2$$
Let $a = \text{lcm}\qty{a_1,a_2}$,
$$(N_{a,x} \subseteq N_{a_1,x}) \land (N_{a,x} \subseteq N_{a_2,x})$$
Therefore,
$$x\in S_{a,x} \subseteq U_1 \cap U_2$$
meaning that any finite interesection of open sets is open.

% Part b
\subsection{Show that $N_{a,b}$ is closed.}
\begin{align*}
    N_{a,b} &= \{a + nb : b \in \Z\} \subset \Z\\
    N_{a,b}^c &= \Z \backslash \{a + nb : b \in \Z\}\\
    N_{a,b} &= \Z \backslash (N_{a,b+1} \cup N_{a,b+2} \cup \dots \cup N_{a,b_a-1})
\end{align*}
Since $N_{a,b}^c$ is open, $N_{a,b}$ is closed.


% Part c
\subsection{Show that $\Z \backslash \{-1,1\}$ is open}
\begin{align*}
    &\forall_{x \in \Z \backslash \{-1,1\}} \exists_{\epsilon > 0} : 
        B_\epsilon(x) \subset \Z \backslash \{-1,1\}\\
    &\forall_{x \in \Z \backslash \{-1,1\}} \exists_{\epsilon > 0} : 
        \forall_{x_b \in Z} d(x,x_b) < \epsilon \implies x_b \in \Z \backslash \{-1,1\}
\end{align*}
Which is clearly open since the ball sets are always contained within the set itself.

% Part d
\subsection{Prove that the set $\mathbb{P}$ of prime numbers cannot be finite.}
Hint: $\Z \backslash \{-1,1\} = \bigcup_{p \in \mathbb{P}} N_{0,p}$\\

Assume $\mathbb{P}$ is finite. 
Since $\forall_{p\in \mathbb{p}} N_{0,p}$ closed, the union over $p \in \mathbb{P}$ would also be closed.\\

However, since $\Z\backslash \{-1,1\}$ is open, this can't be true, so $\mathbb{P}$ must be infinite.





\end{document}
