\documentclass[]{article}

\usepackage{graphicx}

\usepackage[margin=1in]{geometry}

\setlength\parindent{0pt}

\usepackage{physics}
\usepackage{amsmath}
\usepackage{amsfonts}
\usepackage{amssymb}

\usepackage{listings}

\usepackage{enumitem}
\renewcommand{\theenumi}{\alph{enumi}}
\renewcommand*{\thesection}{Problem \arabic{section}}
\renewcommand*{\thesubsection}{\arabic{section}\alph{subsection})}
\renewcommand*{\thesubsubsection}{\quad \roman{subsubsection})}

%Custom Commands
\newcommand{\Rel}{\mathcal{R}}
\newcommand{\R}{\mathbb{R}}
\newcommand{\C}{\mathbb{C}}
\newcommand{\N}{\mathbb{N}}
\newcommand{\Z}{\mathbb{Z}}
\newcommand{\Q}{\mathbb{Q}}

%opening

\title{MATH 5301 Elementary Analysis - Homework 2}

\author{Jonas Wagner}

\date{2021, September 08}



\begin{document}

\maketitle

% Problem 1
\section{}
For a function $f : A \rightarrow B$, show the follwoing for any $X \subset A, Y, Z \subset B$

\subsection{$X \subset f^{-1}(f(X))$}

\begin{align*}
	f(X) &= \{y \in B : \exists x \in X : y = f(x)\}\\
	f^{-1}(Y) &= \{x \in A : \exists y \in Y : y = f(x)\}\\
	f(f^{-1}(X)) &= \{\tilde{x} \in \tilde{X} : \exists x \in A : \tilde{x} = f(f^{-1}(x))\}\\
	&= \{\tilde{x} \in \tilde{X} : \exists y \in Y : y = f(\tilde{x}) :
		\exists x \in X : y = f(x)\} \implies\\
	&\implies \tilde{X} \subset X\\
	&\therefore X \subset f^{-1}(f(X))
\end{align*}



% Let, $$Y = f(X) \subset B$$
% The preimage of $Y$ is then $$\tilde{X} = f^{-1}(Y) \subset A$$
% Additionally, from the knowledge of image then pre-image properties functions,
% $$\tilde{X} \subset X$$
% Therefore, $$ X \subset \tilde{X} = f^{-1}(f(X))$$

\subsection{$f(f^{-1}(Y)) \subset Y$}
% Let, $$X = f^{-1}(Y) \subset A$$
% The maping of $X$ is then $$\tilde{Y} = f(X) \subset B$$
% Additionally, from knowledge pre-image and maping properties functions,
% $$\tilde{Y} \subset Y$$
% Therefore, via the transitive property, $$f(f^{-1}(Y)) \subset Y$$

\begin{align*}
	f^{-1}(Y) &= \{x \in A : \exists y \in Y : y = f(x)\}\\
	f(X) &= \{y \in B : \exists x \in X : y = f(x)\}\\
	f^{-1}(f(Y)) &= \{\tilde{x} \in \tilde{X} : \exists x \in A : \tilde{x} = f^{-1}(f(x))\}\\
	&= \{\tilde{x} \in \tilde{X} : \exists y \in Y, \exists x \in X : (y = f(\tilde{x})) 
		\land (y = f(x)\} \implies\\
	&\implies \tilde{X} \subset X\\
	&\therefore X \subset f^{-1}(f(X))
\end{align*}


\newpage
\subsection{$f^{-1}(Y \cup Z) = f^{-1}(Y) \cup f^{-1}(Z)$}
\begin{align*}
	f^{-1}(Y) &= \{x \in A : \exists y \in Y : y = f(x)\}\\
	f^{-1}(Z) &= \{x \in A : \exists z \in Z : z = f(x)\}\\
	f^{-1}(Y \cup Z) &= \{x \in A : (\exists y \in Y y = f(x))
		\lor (\exists z \in Z : z = f(x))\}\\
	&= \{x \in A : (\exists y \in Y y = f(x))\} 
		\cup \{x \in A :\lor (\exists z \in Z : z = f(x))\}\\
	&= f^{-1}(Y) \cup f^{-1}(Z)\\
	&\therefore f^{-1}(Y \cup Z) = f^{-1}(Y) \cup f^{-1}(Z)
\end{align*}

\subsection{$f^{-1}(Y \cap Z) = f^{-1}(Y) \cap f^{-1}(Z)$}
\begin{align*}
	f^{-1}(Y) &= \{x \in A : \exists y \in Y : y = f(x)\}\\
	f^{-1}(Z) &= \{x \in A : \exists z \in Z : z = f(x)\}\\
	f^{-1}(Y \cap Z) &= \{x \in A : (\exists y \in Y y = f(x))
		\land (\exists z \in Z : z = f(x))\}\\
	&= \{x \in A : (\exists y \in Y y = f(x))\} 
		\cap \{x \in A :\lor (\exists z \in Z : z = f(x))\}\\
	&= f^{-1}(Y) \cap f^{-1}(Z)\\
	&\therefore f^{-1}(Y \cap Z) = f^{-1}(Y) \cap f^{-1}(Z)
\end{align*}




\newpage
% Problem 2
\section{}
Show that:

\subsection{
	$A \cap \bigcup\limits_{\lambda \in \Lambda}  A_\lambda 
	= \bigcup\limits_{\lambda \in \Lambda} (A_\lambda \cap A)$
}

Let $\Lambda := \{1, 2, \cdots, n\}$,

\begin{align*}
	A \cap \bigcup\limits_{\lambda \in \Lambda}  A_\lambda
	&= A \cap \qty(A_1 \cup A_2 \cup \cdots \cup A_n)\\
	&= (A \cap A_1) \cup (A \cap A_2) \cup \cdots \cup (A \cap A_n)\\
	% A \cap \bigcup\limits_{\lambda \in \Lambda}  A_\lambda
	&= \bigcup\limits_{\lambda \in \Lambda} (A_\lambda \cap A)
\end{align*}

Therefore,
\begin{displaymath}
	\boxed{A \cap \bigcup\limits_{\lambda \in \Lambda}  A_\lambda
	= \bigcup\limits_{\lambda \in \Lambda} (A_\lambda \cap A)}
\end{displaymath}


\subsection{
	$\qty(\bigcap\limits_{\lambda\in\Lambda} A_{\lambda}) \cup
	 \qty(\bigcap\limits_{\lambda\in\Lambda} B_{\lambda})
	 \subseteq \bigcap\limits_{\lambda\in\Lambda} (A_\lambda \cup B_\lambda)$
}

Let $\Lambda_A := \{1, 2, \cdots, n\}$ and $\Lambda_B := \{1, 2, \cdots, m\}$,

\begin{align*}
	\bigcap\limits_{\lambda\in\Lambda} (A_\lambda \cup B_\lambda)
	&= (A_1 \cup B_1) \cap (A_1 \cup B_2) \cap 
	\cdots \cap (A_1 \cup B_m) \cap (A_2 \cup B_1) \cap
	\cdots \cap (A_n \cup B_m)\\
	% &= \qty((A_1 \cap A_2) \cup (A_1 \cap B_2) \cup (B_1 \cap A_2) \cup (B_1 \cap B_2)) \cap
	% (A_1)
	&= \qty(A_1 \cup \qty(B_1 \cap B_2 \cap \cdots \cap B_m)) \cap \cdots
	\cap \qty(A_n \cup \qty(B_1 \cap B_2 \cap \cdots \cap B_m))\\
	&= (A_1 \cap A_2) \cup (A_1 \cap A_3) \cup \cdots (A_2 \cap A_3) \cdots 
	\cup (A_{n-1} \cap A_n)\\
	&\qquad \cup (A_1 \cap B_1) \cup \cdots \cup (A_1 \cap B_m) \cup \cdots \cup (A_n \cap B_m)\\
	&\qquad \cup (B_1 \cap B_2) \cup \cdots	\cup (B_1 \cap B_m) \cup \cdots \cup (B_{m-1} \cap B_{m})\\
	&= \qty(\bigcap\limits_{\lambda\in\Lambda} A_{\lambda}) \cup 
	\qty(\bigcap\limits_{\lambda\in\Lambda} B_{\lambda})
	\cup (A_1 \cap B_1) \cup \cdots \cup (A_1 \cap B_m) \cup \cdots \cup (A_n \cap B_m)\\
\end{align*}
Therefore,
\begin{align*}
	\boxed{
	\qty(\bigcap\limits_{\lambda\in\Lambda} A_{\lambda}) \cup
	\qty(\bigcap\limits_{\lambda\in\Lambda} B_{\lambda})
	\subseteq \qty(\bigcap\limits_{\lambda\in\Lambda} A_{\lambda}) \cup 
	\qty(\bigcap\limits_{\lambda\in\Lambda} B_{\lambda})
	\cup (A_1 \cap B_1) \cup \cdots \cup (A_1 \cap B_m) \cup \cdots \cup (A_n \cap B_m)
	}
\end{align*}

\newpage
% Problem 3
\section{}
\textbf{Problem:} Which of these are equivalence relations?\\

% \textbf{Note:}
% Equivalence relations must satisfy the following:
% \subsubsection{Reflective:}
% $$x \Rel x$$
% \subsubsection{Symetric:}
% $$x \Rel y \implies y \Rel x$$
% \subsubsection{Transitive:}
% $$x \Rel y \land y \Rel z \implies x \Rel z$$

\textbf{Solution:}
a, c, \& d\\
(The following explain why)


\subsection{}
for $a, b \in \R$,
let $a \Rel b$
if $a - b \in \Q$

\subsubsection{Reflective:}
\begin{align*}
	x \Rel x 
		&= x - x = 0 \in \Q
\end{align*}

\subsubsection{Symetric:}
\begin{align*}
	x \Rel y 
		&\implies y \Rel x\\
	x \Rel y = x - y \in \Q
		&\implies y - x \in \Q
	\intertext{Since $x-y = -(y-x)$, $(x-y)$ and $(y-x)$ will both be either rational
	or not rational, this is true.}
\end{align*}

\subsubsection{Transitive:}
\begin{align*}
	x \Rel y \land y \Rel z 
		&\implies x \Rel z\\
		\qty(x - y \in \Q) \land \qty(y - z \in \Q)
		&\implies \qty(x - z \in \Q)\\
	\qty(\frac{x_a}{x_b} - \frac{y_a}{y_b} \in \Q) \land \qty(\frac{y_a}{y_b} - \frac{z_a}{z_b} \in \Q)
		&\implies \qty(\frac{x_a}{x_b} - \frac{z_a}{z_b} \in \Q)
\end{align*}
This also means that:
\begin{align*}
	(x_a y_b - x_b y_a \in \N) \land (x_b y_b \neq 0 \in N)
		&\land (y_a z_b - y_b z_a \in \N) \land (y_b z_b \neq 0 \in N)\\
		&\implies (x_a z_b - x_b z_a \in \N) \land (x_b z_b \neq 0 \in N)
\end{align*}
Since this statments indicates that $x_b, y_b, z_b \neq 0$
and that $x_a y_b - x_b y_a, y_a z_b - y_b z_b \in \N$,
the following will always be true as well: $x_a z_b - x_b z_a$.
Therefore, the relation is transitive.


\subsection{}
for $a, b \in \R$,
let $a \Rel b$
if $a - b \notin \Q$

\subsubsection{Reflective:}
The relationship is NOT reflective:
\begin{align*}
	a \Rel b &= a - b \notin \Q\\
	x \Rel x 
		&= x - x = 0 \in \Q
\end{align*}

\newpage
\subsection{}
for $a, b \in \R$,
let $a \Rel b$
if $a - b$ is a square root of a rational number.\\
i.e.
$$a \Rel b = (a-b)^2 \in \Q$$

\subsubsection{Reflective:}
\begin{align*}
	a \Rel b &= (a-b)^2 \in \Q\\
	x \Rel x &= (x-x)^2 = 0^2 = 0 \in \Q
\end{align*}

\subsubsection{Symetric:}
\begin{align*}
	x \Rel y
		&\implies y \Rel x\\
	(x - y)^2 \in \Q
		&\implies (y-x)^2 \in \Q\\
	(x-y)^2 = (y-x)^2
		&\therefore \ x \Rel y \implies y \Rel x
\end{align*}

\subsubsection{Transative:}
\begin{align*}
	x \Rel y \land y \Rel z
		&\implies x \Rel z\\
	\qty((x - y)^2 \in \Q) \land \qty((y - z)^2 \in \Q)
		&\implies \qty((x-z)^2 \in \Q)\\
	\qty(x^2 - 2xy + y^2 \in \Q) \land \qty(y^2 - 2yz + z^2 \in \Q)
		&\implies \qty(x^2 - 2xz + z^2\in \Q)\\
	(x^2 \in \Q) \land (y^2 \in \Q) \land (z^2 \in \Q) \land (-2xy \in \Q) \land (-2yz \in \Q)
		&\implies (x^2 \in \Q) \land (z^2 \in \Q) \land (-2xz \in \Q)\\
	(xy \in \Q) \land (yz \in \Q)
		&\implies (xz \in \Q)
\end{align*}
Which is clearly transitive, so $a \Rel b = (a-b)^2 \in \Q$ is transitive.

\subsection{}
Let $X = \Z \cross \N$,
$x = (x_1, x_2)$ and $y = (y_1, y_2)$
are in $\Rel$ if $x_1 y_2 = x_2 y_1$.

i.e.
$$a_1, b_1 \in \Z, a_2, b_2 \in \N,$$
$$a_1 b_2 = a_2 b_1 \implies (a_1, a_2) \Rel (b_1, b_2)$$

\subsubsection{Reflective:}
\begin{align*}
	a_1 b_2 = a_2 b_1 &\implies (a_1, a_2) \Rel (b_1, b_2)\\
	x_1 x_2 = x_2 x_1 &\implies (x_1, x_2) \Rel (x_1, x_2)
\end{align*}

\subsubsection{Symetric:}
\begin{align*}
	x \Rel y
		&\implies y \Rel x\\
	(x_1 y_2 = x_2 y_1 \implies (x_1, x_2) \Rel (y_1, y_2))
		&\implies (y_1 x_2 = y_2 x_1 \implies (y_1, y_2) \Rel (x_1, x_2))
\end{align*}

\subsubsection{Transative:}
\begin{align*}
	x \Rel y \land y \Rel z
		&\implies x \Rel z\\
	(x_1 y_2 = x_2 y_1 \implies (x_1, x_2) \Rel (y_1, y_2))
	\land (y_1 z_2 = y_2 z_1 \implies (y_1, y_2) \Rel (z_1, z_2))
	&\implies (x_1 z_2 = x_2 z_1 \implies (x_1, x_2) \Rel (z_1, z_2))\\
	(x_1 y_2 = x_2 y_1) \land (y_1 z_2 = y_2 z_1)
	&\implies (x_1 z_2 = x_2 z_1)\\
\end{align*}

Which is clearly transitive, so $(a_1,a_2) \Rel (b_1,b_2)$ is transitive, and therefore 
an equivalence relation.

\newpage
% Problem 4
\section{}
For the relation $(x,y) \succeq (a,b)$ if $(x \geq a)$ and $(y \geq b)$ on the set 
of ordered pairs of $\{1,2,3\} \cross \{1,2,3\}$.
i.e. $x,a \in \{1,2,3\}$ and $y,b \in \{1,2,3\}$,
$$(x \geq a) \land (y \geq b) \implies (x,y) \succeq (a,b)$$

\subsection{Show that the above relation is an order relation.}
An ordered relation requires (i) reflexitivity, (ii) anti-symmetry, and (iii) transivity.

\subsubsection{Reflective:}
\begin{align*}
	(x \geq a) \land (y \geq b) 
		&\implies (x,y) \succeq (a,b)\\
	(x \geq x) \land (y \geq y) 
		&\implies (x,y) \succeq (x,y)
\end{align*}

\subsubsection{Anti-Symmetry:}
\begin{align*}
	((x,y) \succeq (a,b)) \land ((a,b) \succeq (x,y))
		&\implies (x,y) = (a,b)\\
	((x \geq a) \land (y \geq b)) \land ((a \geq x) \land (b \geq y))
		&\implies (x,y) = (a,b)\\
	((x \geq a) \land (a \geq x)) \land ((b \geq y) \land (y \geq b))
		&\implies (x,y) = (a,b)
\end{align*}

\subsubsection{Transivity}
\begin{align*}
	(x,y) \Rel (a,b) \land (a,b) \Rel (\alpha,\beta) 
		&\implies (x,y) \Rel (\alpha,\beta)\\
	((x \geq a) \land (y \geq b)) \land ((a \geq \alpha) \land (b \geq \beta))
		&\implies (x \geq \alpha) \land (y \geq \beta)\\
	((x \geq a) \land (a \geq \alpha)) \land ((y \geq b) \land (b \geq \beta))
		&\implies (x \geq \alpha) \land (y \geq \beta)\\
	(x \geq a \geq \alpha) \land (y \geq b \geq \beta)
		&\implies (x \geq \alpha) \land (y \geq \beta)\\
\end{align*}

Therefore, the relation $(x,y) \succeq (a,b)$ is an order relation.

\subsection{Can you make it the total order?}

\subsubsection{Totality:}
\begin{align*}
	\forall (x,y),(a,b) \in \{1,2,3\} \cross \{1,2,3\}
		&\implies ((x,y) \succeq (a,b)) \lor ((a,b) \succeq (x,y))\\
	\forall x,a \in \{1,2,3\}, \forall y,b \in \{1,2,3\}
		&\implies ((x \geq a) \land (y \geq b)) \lor ((a \geq x) \land (y \geq b))
\end{align*}

% However, it is not a universal total order. A counter example is:
% \begin{align*}
% 	(x,y) &= (1,3)\\
% 	(a,b) &= (3,1)\\
% 	(x,y) \succeq (a,b) &= ((x \geq a) \land (y \geq b)) \lor ((a \geq x) \land (b \geq y))\\
% 	&= ((1 \geq 3) \land (3 \geq 1)) \lor ((3 \geq 1) \land (1 \geq 3))
% \end{align*}
% which is false, so a it is not a complete total order.

\subsection{How many different total orderings can be constructed?}
Multiple total orderings of relation on subsets can satisfy this.
A network constructed from the following relations would demonstrate the multiple paths.
\begin{align*}
	(3,3) &\succeq (x,y) \forall (x,y) \in \{1,2,3\}\cross\{1,2,3\}\\
	(3,2) &\succeq (x,y) \forall (x,y) \in \{1,2,3\}\cross\{1,2\}\\
	(3,1) &\succeq (x,y) \forall (x,y) \in \{1,2,3\}\cross\{1\}\\
	(2,3) &\succeq (x,y) \forall (x,y) \in \{1,2\}\cross\{1,2,3\}\\
	(2,2) &\succeq (x,y) \forall (x,y) \in \{1,2\}\cross\{1,2\}\\
	(2,1) &\succeq (x,y) \forall (x,y) \in \{1,2\}\cross\{1\}\\
	(1,3) &\succeq (x,y) \forall (x,y) \in \{(1,1),(1,2),(1,3)\}\\
	(1,2) &\succeq (x,y) \forall (x,y) \in \{(1,1),(1,2)\}\\
	(1,1) &\succeq (1,1)
\end{align*}

If you only count the total orders of length 2 
(the number of actual network edges with self-edges) this would be 36.
If drawns as a tree, these 36 pairs would be created as directed edges and the 
total number of unique tree paths can also be counted using various algorithms.

\newpage
% Problem 5
\section{}
Provide and example of $f : \Z \to \N$ such that

\subsection{$f$ is surjective, but not injective}

\begin{displaymath}
	y = f(x) =
	\begin{cases}
		x & x \geq 0\\
		-x & x < 0
	\end{cases}
\end{displaymath}

\subsection{$f$ is injective, but not surjective}
\begin{displaymath}
	y = f(x) =
	\begin{cases}
		x^2 & x \geq 0\\
		x^2 + 1 & x < 0
	\end{cases}
\end{displaymath}

\subsection{$f$ is surjective and injective (bijective)}

\begin{displaymath}
	y = f(x) =
	\begin{cases}
		2 x & x \geq 0\\
		-2 x - 1 & x < 0
	\end{cases}
\end{displaymath}

\subsection{$f$ is niether surjective nor injective}

\begin{displaymath}
	y = f(x) = 0
\end{displaymath}


\newpage
% Problem 6
\section{}
\textbf{Problem:}
Is the following statement correct?\\

\textbf{Theorem 1.}
\textit{If the relation $\R$ on $A$ is symetric and transistive, then it is reflexive.}\\
\textit{Proof:}
For any $a \in A$ let $b \in A$ is such that $a \Rel b$. Then by symmetry $b \Rel a$. 
Then by symmetry $a \Rel a$.\\

\textbf{Solution:}
No. Specifically, the final statement of the proof states to use symmetry to conclude that 
it is reflexive, but it actually requires the transivity property to make that conclusion.\\

The following is a proposed corrected statement:\\

\textbf{Theorem 1.}
\textit{If the relation $\Rel$ is symetric and transistive on $A$, then $Rel$ is also reflexive on $A$.}\\
\textit{Proof:}
Let $a \in A$ and $b \in A$ be selected so that $a \Rel b$
Since $\Rel$ is symetric: $$a \Rel b \implies b \Rel a$$
Since $\Rel$ is transitive: $$(a \Rel b) \land (b \Rel a) \implies a \Rel a$$
Therefore, $\Rel$ is also reflexive.





\end{document}
