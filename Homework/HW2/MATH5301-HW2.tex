\documentclass[]{article}

\usepackage{graphicx}

\usepackage[margin=1in]{geometry}

\setlength\parindent{0pt}

\usepackage{physics}
\usepackage{amsmath}
\usepackage{amsfonts}
\usepackage{amssymb}

\usepackage{listings}

\usepackage{enumitem}
\renewcommand{\theenumi}{\alph{enumi}}
\renewcommand*{\thesection}{Problem \arabic{section}}
\renewcommand*{\thesubsection}{\arabic{section}\alph{subsection})}
\renewcommand*{\thesubsubsection}{\quad \roman{subsubsection})}

%Custom Commands
\newcommand{\Rel}{\mathcal{R}}
\newcommand{\R}{\mathbb{R}}
\newcommand{\C}{\mathbb{C}}
\newcommand{\N}{\mathbb{N}}
\newcommand{\Z}{\mathbb{Z}}
\newcommand{\Q}{\mathbb{Q}}

%opening

\title{MATH 5301 Elementary Analysis - Homework 2}

\author{Jonas Wagner}

\date{2021, September 08}



\begin{document}

\maketitle

\section{}
For a function $f : A \rightarrow B$, show the follwoing for any $X \subset A, Y, Z \supset B$

\subsection{$X \subset f^{-1}(f(X))$}




\subsection{$f(f^{-1}(Y)) \subset Y$}


\subsection{$f^{-1}(Y \cup Z) = f^{-1}(Y) \cup f^{-1}(Z)$}




\subsection{$f^{-1}(Y \cap Z) = f^{-1}(Y) \cap f^{-1}(Z)$}





\newpage
\section{}
Show that:

\subsection{
	$A \cap \bigcup\limits_{\lambda \in \Lambda}  A_\lambda 
	= \bigcup\limits_{\lambda \in \Lambda} (A_\lambda \cap A)$
}

Let $\Lambda := \{1, 2, \cdots, n\}$,

\begin{align*}
	A \cap \bigcup\limits_{\lambda \in \Lambda}  A_\lambda
	&= A \cap \qty(A_1 \cup A_2 \cup \cdots \cup A_n)\\
	&= (A \cap A_1) \cup (A \cap A_2) \cup \cdots \cup (A \cap A_n)\\
	% A \cap \bigcup\limits_{\lambda \in \Lambda}  A_\lambda
	&= \bigcup\limits_{\lambda \in \Lambda} (A_\lambda \cap A)
\end{align*}

Therefore,
\begin{displaymath}
	\boxed{A \cap \bigcup\limits_{\lambda \in \Lambda}  A_\lambda
	= \bigcup\limits_{\lambda \in \Lambda} (A_\lambda \cap A)}
\end{displaymath}


\subsection{
	$\qty(\bigcap\limits_{\lambda\in\Lambda} A_{\lambda}) \cup
	 \qty(\bigcap\limits_{\lambda\in\Lambda} B_{\lambda})
	 \subseteq \bigcap\limits_{\lambda\in\Lambda} (A_\lambda \cup B_\lambda)$
}

Let $\Lambda_A := \{1, 2, \cdots, n\}$ and $\Lambda_B := \{1, 2, \cdots, m\}$,

\begin{align*}
	\bigcap\limits_{\lambda\in\Lambda} (A_\lambda \cup B_\lambda)
	&= (A_1 \cup B_1) \cap (A_1 \cup B_2) \cap 
	\cdots \cap (A_1 \cup B_m) \cap (A_2 \cup B_1) \cap
	\cdots \cap (A_n \cup B_m)\\
	% &= \qty((A_1 \cap A_2) \cup (A_1 \cap B_2) \cup (B_1 \cap A_2) \cup (B_1 \cap B_2)) \cap
	% (A_1)
	&= \qty(A_1 \cup \qty(B_1 \cap B_2 \cap \cdots \cap B_m)) \cap \cdots
	\cap \qty(A_n \cup \qty(B_1 \cap B_2 \cap \cdots \cap B_m))\\
	&= (A_1 \cap A_2) \cup (A_1 \cap A_3) \cup \cdots (A_2 \cap A_3) \cdots 
	\cup (A_{n-1} \cap A_n)\\
	&\qquad \cup (A_1 \cap B_1) \cup \cdots \cup (A_1 \cap B_m) \cup \cdots \cup (A_n \cap B_m)\\
	&\qquad \cup (B_1 \cap B_2) \cup \cdots	\cup (B_1 \cap B_m) \cup \cdots \cup (B_{m-1} \cap B_{m})\\
	&= \qty(\bigcap\limits_{\lambda\in\Lambda} A_{\lambda}) \cup 
	\qty(\bigcap\limits_{\lambda\in\Lambda} B_{\lambda})
	\cup (A_1 \cap B_1) \cup \cdots \cup (A_1 \cap B_m) \cup \cdots \cup (A_n \cap B_m)\\
\end{align*}
Therefore,
\begin{align*}
	\boxed{
	\qty(\bigcap\limits_{\lambda\in\Lambda} A_{\lambda}) \cup
	\qty(\bigcap\limits_{\lambda\in\Lambda} B_{\lambda})
	\subseteq \qty(\bigcap\limits_{\lambda\in\Lambda} A_{\lambda}) \cup 
	\qty(\bigcap\limits_{\lambda\in\Lambda} B_{\lambda})
	\cup (A_1 \cap B_1) \cup \cdots \cup (A_1 \cap B_m) \cup \cdots \cup (A_n \cap B_m)
	}
\end{align*}











\newpage
\section{}
\textbf{Problem:}
Which of these are equivalence relations?\\

% \textbf{Note:}
% Equivalence relations must satisfy the following:
% \subsubsection{Reflective:}
% $$x \Rel x$$
% \subsubsection{Symetric:}
% $$x \Rel y \implies y \Rel x$$
% \subsubsection{Transitive:}
% $$x \Rel y \land y \Rel z \implies x \Rel z$$

\textbf{Solution:}
a, c, \& d\\
(The following explain why)


\subsection{}
for $a, b \in \R$,
let $a \Rel b$
if $a - b \in \Q$

\subsubsection{Reflective:}
\begin{align*}
	x \Rel x 
		&= x - x = 0 \in \Q
\end{align*}

\subsubsection{Symetric:}
\begin{align*}
	x \Rel y 
		&\implies y \Rel x\\
	x \Rel y = x - y \in \Q
		&\implies y - x \in \Q
	\intertext{Since $x-y = -(y-x)$, $(x-y)$ and $(y-x)$ will both be either rational
	or not rational, this is true.}
\end{align*}

\subsubsection{Transitive:}
\begin{align*}
	x \Rel y \land y \Rel z 
		&\implies x \Rel z\\
		\qty(x - y \in \Q) \land \qty(y - z \in \Q)
		&\implies \qty(x - z \in \Q)\\
	\qty(\frac{x_a}{x_b} - \frac{y_a}{y_b} \in \Q) \land \qty(\frac{y_a}{y_b} - \frac{z_a}{z_b} \in \Q)
		&\implies \qty(\frac{x_a}{x_b} - \frac{z_a}{z_b} \in \Q)
\end{align*}
This also means that:
\begin{align*}
	(x_a y_b - x_b y_a \in \N) \land (x_b y_b \neq 0 \in N)
		&\land (y_a z_b - y_b z_a \in \N) \land (y_b z_b \neq 0 \in N)\\
		&\implies (x_a z_b - x_b z_a \in \N) \land (x_b z_b \neq 0 \in N)
\end{align*}
Since this statments indicates that $x_b, y_b, z_b \neq 0$
and that $x_a y_b - x_b y_a, y_a z_b - y_b z_b \in \N$,
the following will always be true as well: $x_a z_b - x_b z_a$.
Therefore, the relation is transitive.


\subsection{}
for $a, b \in \R$,
let $a \Rel b$
if $a - b \notin \Q$

\subsubsection{Reflective:}
The relationship is NOT reflective:
\begin{align*}
	a \Rel b &= a - b \notin \Q\\
	x \Rel x 
		&= x - x = 0 \in \Q
\end{align*}

\newpage
\subsection{}
for $a, b \in \R$,
let $a \Rel b$
if $a - b$ is a square root of a rational number.\\
i.e.
$$a \Rel b = (a-b)^2 \in \Q$$

\subsubsection{Reflective:}
\begin{align*}
	a \Rel b &= (a-b)^2 \in \Q\\
	x \Rel x &= (x-x)^2 = 0^2 = 0 \in \Q
\end{align*}

\subsubsection{Symetric:}
\begin{align*}
	x \Rel y
		&\implies y \Rel x\\
	(x - y)^2 \in \Q
		&\implies (y-x)^2 \in \Q\\
	(x-y)^2 = (y-x)^2
		&\therefore \ x \Rel y \implies y \Rel x
\end{align*}

\subsubsection{Transative:}
\begin{align*}
	x \Rel y \land y \Rel z
		&\implies x \Rel z\\
	\qty((x - y)^2 \in \Q) \land \qty((y - z)^2 \in \Q)
		&\implies \qty((x-z)^2 \in \Q)\\
	\qty(x^2 - 2xy + y^2 \in \Q) \land \qty(y^2 - 2yz + z^2 \in \Q)
		&\implies \qty(x^2 - 2xz + z^2\in \Q)\\
	(x^2 \in \Q) \land (y^2 \in \Q) \land (z^2 \in \Q) \land (-2xy \in \Q) \land (-2yz \in \Q)
		&\implies (x^2 \in \Q) \land (z^2 \in \Q) \land (-2xz \in \Q)\\
	(xy \in \Q) \land (yz \in \Q)
		&\implies (xz \in \Q)
\end{align*}
Which is clearly transitive, so $a \Rel b = (a-b)^2 \in \Q$ is transitive.

\subsection{}
Let $X = \Z \cross \N$,
$x = (x_1, x_2)$ and $y = (y_1, y_2)$
are in $\Rel$ if $x_1 y_2 = x_2 y_1$.
















\newpage
\section{}















\end{document}
