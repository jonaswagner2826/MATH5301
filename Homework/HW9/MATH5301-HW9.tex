% Standard Article Definition
\documentclass[]{article}

% Page Formatting
\usepackage[margin=1in]{geometry}
\setlength\parindent{0pt}

% Graphics
\usepackage{graphicx}

% Math Packages
\usepackage{physics}
\usepackage{amsmath, amsfonts, amssymb, amsthm}
\usepackage{mathtools}

% Code Def
\usepackage{listings}

% Section Heading Settings
\usepackage{enumitem}
\renewcommand{\theenumi}{\alph{enumi}}
\renewcommand*{\thesection}{Problem \arabic{section}}
\renewcommand*{\thesubsection}{\alph{subsection})}
\renewcommand*{\thesubsubsection}{\quad \quad \roman{subsubsection})}

%Custom Commands
\newcommand{\Rel}{\mathcal{R}}
\newcommand{\R}{\mathbb{R}}
\newcommand{\C}{\mathbb{C}}
\newcommand{\N}{\mathbb{N}}
\newcommand{\Z}{\mathbb{Z}}
\newcommand{\Q}{\mathbb{Q}}

\newcommand{\toI}{\xrightarrow{\textsf{\tiny I}}}
\newcommand{\toS}{\xrightarrow{\textsf{\tiny S}}}
\newcommand{\toB}{\xrightarrow{\textsf{\tiny B}}}

\newcommand{\divisible}{ \ \vdots \ }
\newcommand{\st}{\ : \ }


% Theorem Definition
\newtheorem{definition}{Definition}
\newtheorem{assumption}{Assumption}
\newtheorem{theorem}{Theorem}
\newtheorem{lemma}{Lemma}
\newtheorem{proposition}{Proposition}


%opening

\title{MATH 5301 Elementary Analysis - Homework 8}

\author{Jonas Wagner}

\date{2021, October 29\textsuperscript{th}}

\begin{document}

\maketitle

% Problem 1 ----------------------------------------------
\section{}
Let $\norm{\cdot}_a$ and $\norm{\cdot}_b$ be two equivalent norms on $\R^n$.
\begin{definition}
    For $\norm{\cdot}_a, \norm{\cdot}_b$ on $S$, 
    $\norm{\cdot}_a$ is said to be \underline{stronger} then $\norm{\cdot}_b$ if 
    \[
        \forall \{x_n\} \subset S \st x_n \xrightarrow[d_a]{} x \implies x_n \xrightarrow[d_b]{} x
    \]
\end{definition}
\begin{definition}
    $\norm{\cdot}_a$ and $\norm{\cdot}_b$ are said to be \underline{equivalent},  $\norm{\cdot}_a \sim \norm{\cdot}_b$,
    if $\norm{\cdot}_a$ is stronger then $\norm{\cdot}_b$ 
    and $\norm{\cdot}_b$ is stronger then $\norm{\cdot}_a$. 
    This means that
    \[
        \norm{\cdot}_a \sim \norm{\cdot}_b 
            \iff \exists{\alpha,\beta \in \R_{>0}} : 
            \forall_{x\in S} \alpha \norm{\cdot}_b \leq \norm{\cdot}_a \leq \beta \norm{x}_b
    \]
\end{definition}

%Part a
\subsection{Prove that if the set $A$ is closed in the $a$-norm, then it is closed in $b$-norm.}

\begin{definition}
    The set $A \subset V$ is called \underline{open} if 
    $$\forall_{x\in A} \exists_{\epsilon>0} : B_\epsilon(x)\subset A$$
    or equivalently,
    \[\forall_{x \in A} \exists_{\epsilon>0} : \forall_{y \in V}\norm{x - y} < \epsilon \implies y \in A\]
\end{definition}
\begin{definition}
    The set $A \subset V$ is called \underline{closed} if $A^c$ is open.
\end{definition}

\begin{theorem}
    If the set $A$ is closed in the $a$-norm, then it is closed in $b$-norm.
    \begin{proof}
        Set $A$ being closed in $a$-norm implies $A^c$ is open in $a$-norm.
        \[\forall_{x \in A^c} \exists_{\epsilon_a>0} : \forall_{y \in S} \ \norm{x - y}_a < \epsilon_a \implies y \in A^c\]
        Additionally, since $\norm{\cdot}_a$ is equivalent to $\norm{\cdot}_b$ means thats
        \[\exists_{\alpha,\beta > 0} : 
        \forall_{x\in S} \  \alpha \norm{\cdot}_b \leq \norm{\cdot}_a \leq \beta \norm{\cdot}_b\]
        Therefore, $\norm{x - y}_a \leq \beta \norm{x-y}_b$ and then 
        \begin{align*}
            \forall_{x \in A^c} \exists_{\epsilon_a>0} 
            &\st \forall_{y \in S} \ \norm{x - y}_a \leq \beta \norm{x - y}_b < \epsilon_a \implies y \in A^c\\
            \forall_{x \in A^c} \exists_{\epsilon_b>0} &\st \forall_{y \in S} \ \norm{x - y}_b < \epsilon_b \implies y \in A^c
        \end{align*}
        where $ \epsilon_b \geq \cfrac{\epsilon_a}{\beta}$
    \end{proof}
\end{theorem}

\newpage
%Part b
\subsection{Prove that if the set $A$ is compact in the $a-norm$ then it is compact in the $b$-norm.}
\begin{definition}
    Let $(S,d)$ be a metric space with $A \subset S$,
    \begin{enumerate}
        \item For $\{U_\alpha\}_{\alpha \in A}, \ U_\alpha \subset S$, is a \textbf{\underline{cover}} of the set $A$ if 
        $$A \subset \bigcup_{\alpha\in A} U_\alpha$$
        \item A cover $\{U_\alpha\}_{\alpha \in A}$ of $A$ is an \textbf{\underline{open cover}} if $\forall_{\alpha \in A}$ $U_\alpha$ is an open set.
        \item $\{V_\beta\}_{\beta\in B}$ is called a \textbf{\underline{subcover}} of $\{U_\alpha\}_{\alpha \in A}$ if
        \begin{enumerate}
            \item $\{V_\beta\}_{\beta\in B}$ is a cover of $A$
            \item $\forall_{\beta \in B} \exists_{\alpha \in A} V_\beta = U_\alpha$
        \end{enumerate}
        \item A cover with a finite number of sets is called a \textbf{\underline{finite cover}}.
    \end{enumerate}
\end{definition}
\begin{definition}
    For $A \subset (S,d)$, $A$ is \textbf{\underline{compact}} if for every open cover of $A$ there exists a finite sub cover.
    Which is equivalent to saying all sequences within $A$ converge to a set point in $A$. (i.e)
    \[\forall_{{a_k}_{k\in \N}} \exists_{{a_{n_k}}} : a_{n_k} \to a \in A\]
\end{definition}
\begin{definition}
    A sequence $\{x_{n}\}$is called \underline{Cauchy} if 
    \[
        \forall_{\epsilon>0} \ \exists_{N \in \N} \ \forall_{l_1,l_2 \geq N} \norm{x_{n_{l_1}} - x_{n_{l_2}}} < \epsilon
    \]
\end{definition}

\begin{theorem}
    If the set $A$ is compact in the $a$-norm, then it is compact in the $b$-norm.
    \begin{proof}
        Set $A$ being compact in $a$-norm means that every sequence in $A$ satisfies the Cauchy sequence property:
        \[
            \forall_{\epsilon>0} \ \exists_{N \in \N} \ \forall_{l_1,l_2 \geq N} \norm{x_{n_{l_1}} - x_{n_{l_2}}}_a < \epsilon
        \]
        Additionally, since $\norm{\cdot}_a$ is equivalent to $\norm{\cdot}_b$ means thats
        \[
            \exists_{\alpha,\beta > 0} : \forall_{x\in S} \ \alpha \norm{\cdot}_b \leq \norm{\cdot}_a \leq \beta \norm{\cdot}_b
        \]
        Therefore, $\norm{x - y}_a \leq \beta \norm{x-y}_b$ and then 
        \begin{align*}
            \forall_{\epsilon_a>0} \ \exists_{N \in \N} \ \forall_{l_1,l_2 \geq N} 
            &\norm{x_{n_{l_1}} - x_{n_{l_2}}}_a \leq \beta \norm{x_{n_{l_1}} - x_{n_{l_2}}}_b < \epsilon_a\\
            \forall_{\epsilon_b>0} \ \exists_{N \in \N} \ \forall_{l_1,l_2 \geq N} 
            &\norm{x_{n_{l_1}} - x_{n_{l_2}}}_b < \epsilon_b
        \end{align*}
        where $ \epsilon_b \geq \cfrac{\epsilon_a}{\beta}$
    \end{proof}
\end{theorem}

% Problem 2
\newpage
\section{}
Consider the set $l^{\infty}$ of all real-valued sequences, endowed with the sup-norm: $\norm{l}_\infty = \sup_{n\in \N} \abs{l_n}$.

%Part a
\subsection{Prove that $l^\infty$ is complete.}

\begin{definition}
    A metric space is $(S,d)$ is a \underline{complete metric space} if every Cauchy sequence in $S$ converges.
\end{definition}
\begin{definition}
    The set $A$ in norm space is $(S,\norm{\cdot})$ is a \underline{complete} set if every Cauchy sequence in $A$ converges to a limit in $A$.
\end{definition}

\begin{definition}
    Let the set $l^\infty$ be the set of real-valued sequences:
    \[
        l^{\infty} := \qty{\{l_n\}_{n\in \N} \st l_n \in \R}
    \]
\end{definition}
\begin{definition}
    Let the norm space be defined as $(l^\infty, \norm{l}_\infty)$ where \[\norm{l}_\infty = \sup_{n\in\N} \abs{l_n}\]
\end{definition}

\begin{theorem}
    The set $l^\infty$ is complete.
    \begin{proof}
        asdf
    \end{proof}
\end{theorem}



%Part b
\subsection{Prove that $l^\infty$ is not compact.}






% Problem 3
\newpage
\section{}
Consider the set $\mathbb{B}([0,1],\R)$ of all bounded real-valued functions on the unit interval endowed with the sup-norm: $\norm{f}_\infty = \sup_{x \in [0,1]} \abs{f(x)}$. Denote the closed unit ball as $B_1 := \qty{f \in \mathbb{B}([0,1],\R) \st \norm{f}_\infty \leq 1}$.


%Part a
\subsection{Prove $B_1$ is closed.}
\begin{theorem}
    $B_1$ is the closed.
    % \begin{proof}
    %     $B_1$ being closed implies $B_1^c$ is open:
    %     \[\]
    % \end{proof}
    \begin{proof}
        Set $B_1$ being closed implies $B_1^c$ is open.
        \begin{align*}
            \forall_{f \in B_1^c} \exists_{\epsilon>0} \st \forall_{g \in \mathbb{B}} \ \norm{f-g}_\infty < \epsilon &\implies g \in B_1^c\\
            \forall_{f \in \mathbb{B}} \norm{f}_\infty > 1 \exists_{\epsilon>0}\st \forall_{g \in \mathbb{B}} \ \norm{f-g}_\infty < \epsilon &\implies g \in \mathbb{B} \st \norm{g}_\infty > 1\\
            % \implies\\
            % \implies 
            \forall_{f, g \in \mathbb{B}} \exists_{\epsilon_1>0} \st \norm{f}_\infty > 1 \land  \norm{f-g}_\infty < \epsilon_1 &\implies \norm{g}_\infty > 1
            % \forall_{f, g \in \mathbb{B}} \exists_{\epsilon_2>0} \st \norm{f}_\infty > 1 \land  \norm{f-g}_\infty \leq \norm{f}_\infty + \norm{g}_\infty < \epsilon_2 &\implies \norm{g}_\infty > 1
            % \intertext{with $\epsilon_2 \geq \epsilon_1$.}
            % \forall_{f, g \in \mathbb{B}} \exists_{\epsilon_1>0} \st \qty(\sup_{x \in [0,1]} \abs{f(x)}) > 1 \land  \qty(\sup_{x \in [0,1]} \abs{f(x)-g(x)}) < \epsilon_1 &\implies \qty(\sup_{x \in [0,1]} \abs{g(x)}) > 1\\
            % \intertext{Since the supremum is the lowest upper bound, it will always be greater then value of the bounded function}
        \end{align*}
        Additionally, since $\norm{f}_\infty > 1$ and $\norm{f-g}_\infty$ is bounded, the only way these are both true this must also be true: $\norm{g}_\infty > 1$.

        Alternatively, you can just recognize that and $f \in \mathbb{B}\st \norm{\cdot}_\infty > 1 \implies f \notin B_1 \implies f \in B_1^c$ which demonstrates $B_1^c$ is open, and therefore, $B_1$ is closed.
    \end{proof}    
\end{theorem}

%Part b
\subsection{Prove that $B_1$ is bounded.}

\begin{theorem}
    $B_1$ is bounded, i.e.
    \[\exists_{N} \st \forall_{f \in B_1} \ \norm{f}_\infty < N\]
    \begin{proof}
        Since, by definition, $\norm{f} \leq 1$, $B_1$ is clearly bounded for any $N > 1$.
    \end{proof}
\end{theorem}

%Part c
\subsection{Prove that $B_1$ is not compact.}

\begin{theorem}
    $B_1$ is not compact.
    \begin{proof}
        $B_1$ not being compact is equivalent to saying
        \begin{align*}
            \lnot \qty(\forall_{{f_k}_{k\in \N}} f_k \in \mathbb{B} \st \exists_{{f_{n_k}}} : f_{n_k} \to f \in B_1)\\
            \exists_{{f_k}_{k\in \N}} f_k \in \mathbb{B} \st \forall_{{f_{n_k}}} : f_{n_k} \to f \notin B_1
        \end{align*}
    \end{proof}
\end{theorem}



% Problem 4
\newpage
\section{}
Let $\{V, \norm{\cdot}\}$ be a normed space.
Show that the function $f(x) = \norm{x} \st V \to \R$ is continuos on $V$.

\begin{definition}
    A function $f : (S_1,d_1) \to (S_2,d_2)$ is continuous on $S_1$ if 
    \[
        \forall_{x\in S_1} \forall_{\epsilon>0} \exists_{\delta(x,\epsilon)>0} \forall_{y\in S_1} d_1(x,y) < \epsilon \implies d_2(f(x),f(y)) < \delta
    \]
\end{definition}

\begin{theorem}
    The function $f(x)$ is continuous on $V$, i.e.
    \[
        \forall_{x\in V} \forall_{\epsilon>0} \exists_{\delta(x,\epsilon)>0} \forall_{y\in V} \norm{x - y} < \epsilon \implies \abs{f(x) - f(y)} < \delta
    \]
    \begin{proof}
        \begin{align*}
            % \forall_{x\in V} \forall_{\epsilon>0} \exists_{\delta(x,\epsilon)>0} \forall_{y\in V} \norm{x - y} < \epsilon 
            %     &\implies \abs{f(x) - f(y)} < \delta\\
            \forall_{x\in V} \forall_{\epsilon>0} \exists_{\delta(x,\epsilon)>0} \forall_{y\in V} \norm{x - y} < \epsilon 
                &\implies \abs{\norm{x} - \norm{y}} < \delta\\
            \forall_{x\in V} \forall_{\epsilon_1>0} \exists_{\delta_2(x,\epsilon_1)>0} \forall_{y\in V} \norm{x - y} \leq \norm{x} + \norm{y} < \epsilon_1 
                &\implies \abs{\norm{x} - \norm{y}}\leq \norm{x} + \norm{y} < \delta_2\\
            \forall_{x\in V} \forall_{\epsilon_1>0} \exists_{\delta_2(x,\epsilon_1)>0} \forall_{y\in V} \norm{x} + \norm{y} < \epsilon_1 
                &\implies \norm{x} + \norm{y} < \delta_2\\
        \end{align*}
        which is clearly true, therefore $f(x) = \norm{x}$ is continuous on $V$.
    \end{proof}
\end{theorem}


% Problem 5
\newpage
\section{}
Let $(X,d_1)$ and $(Y,d_2)$ are two metric spaces.
Assume also that $Y$ is a vector space.
Construct and example of two continuous functions $f,g : X \to Y$ such that $f + g$ is discontinuous.

\begin{definition}
    Let $f : X \to Y$ be defined by
    \[
        f(x) := 
    \]
\end{definition}
\begin{definition}
    Let $g : X \to Y$ be defined by
    \[
        g(x) := 
    \]
\end{definition}

\begin{theorem}
    Functions $f$ and $g$ are continuous, but $f + g$ is discontinuous.
    \begin{proof}
        
        \subsection{}
        \begin{lemma}
            $f$ is a continuous function.
            \begin{proof}
                \begin{align*}
                    \forall_{x \in X} \forall_{\epsilon > 0} \exists_{\delta(x, \epsilon)} \forall_{y \in X} \st d_1(x,y) < \epsilon
                        &\implies d_2(f(x), f(y)) < \delta\\
                        &\implies d_2() < \delta
                \end{align*}
            \end{proof}
        \end{lemma}
    \end{proof}
\end{theorem}




% Problem 6
\newpage
\section{}
Construct an example of a sequence $\{f_n\}$ of nowhere continuous functions $[0,1] \to \R$ such that $f_n$ converge in the sup-norm to continuous functions.






\end{document}
