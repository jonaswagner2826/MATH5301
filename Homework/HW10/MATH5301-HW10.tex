% Standard Article Definition
\documentclass[]{article}

% Page Formatting
\usepackage[margin=1in]{geometry}
\setlength\parindent{0pt}

% Graphics
\usepackage{graphicx}

% Math Packages
\usepackage{physics}
\usepackage{amsmath, amsfonts, amssymb, amsthm}
\usepackage{mathtools}

% Code Def
\usepackage{listings}

% Section Heading Settings
\usepackage{enumitem}
\renewcommand{\theenumi}{\alph{enumi}}
\renewcommand*{\thesection}{Problem \arabic{section}}
\renewcommand*{\thesubsection}{\alph{subsection})}
\renewcommand*{\thesubsubsection}{\quad \quad \roman{subsubsection})}

%Custom Commands
\newcommand{\Rel}{\mathcal{R}}
\newcommand{\R}{\mathbb{R}}
\newcommand{\C}{\mathbb{C}}
\newcommand{\N}{\mathbb{N}}
\newcommand{\Z}{\mathbb{Z}}
\newcommand{\Q}{\mathbb{Q}}

\newcommand{\toI}{\xrightarrow{\textsf{\tiny I}}}
\newcommand{\toS}{\xrightarrow{\textsf{\tiny S}}}
\newcommand{\toB}{\xrightarrow{\textsf{\tiny B}}}

\newcommand{\divisible}{ \ \vdots \ }
\newcommand{\st}{\ : \ }


% Theorem Definition
\newtheorem{definition}{Definition}
\newtheorem{assumption}{Assumption}
\newtheorem{theorem}{Theorem}
\newtheorem{lemma}{Lemma}
\newtheorem{proposition}{Proposition}


%opening
\title{MATH 5301 Elementary Analysis - Homework 10}
\author{Jonas Wagner}
\date{2021, November 12\textsuperscript{th}}

\begin{document}

\maketitle

% Problem 1 ----------------------------------------------
\section{}
Compute the derivatives of the following functions:

% Derivative
\begin{definition}
    Let $f : (a,b) \to \R$ be a function.    
    \begin{enumerate}
        \item The \emph{\underline{derivative of the function at point $x_0$}} is defined as
        \[
            f'(x_0) := \lim_{x \to x_0} \cfrac{f(x) - f(x_0)}{x - x_0}
        \]
        \item If the derivative is defined at $x_0$, then it is \emph{\underline{differentiable at $x_0$}}.
        \item If the derivative is defined for all $x_0 \in (a,b)$, then the function $f$ is said to be \emph{\underline{differentialble}}.
        \item When $f$ is differentiable, the \underline{\emph{derivative}} of $f(x)$ is defined as:
        \[
            f'(x) := \lim_{h \to 0} \cfrac{f(x + h) - f(x)}{h}
        \]
    \end{enumerate}
\end{definition}

%Part a
\subsection{
    $x^2 \sin{\frac{1}{x}}$
}
Let
\[
    f(x) = x^2 \sin{\frac{1}{x}}
\]
Then, by definition, the derivative of $f(x)$ is calculated as
\begin{align*}
    f'(x) 
    &= \lim_{h \to 0} \cfrac{f(x + h) - f(x)}{h}\\
    &= \lim_{h \to 0} 
        \cfrac{
            (x + h)^2 \sin(\frac{1}{x + h}) - x^2 \sin(\frac{1}{x})
            }{h}\\
    &= \lim_{h \to 0} 
        \cfrac{
            \dv{h}\qty((x + h)^2 \sin(\frac{1}{x + h}) - x^2 \sin(\frac{1}{x}))
            }{
                \dv{h} h
            }\\
    &= \lim_{h \to 0} 
        \cfrac{
            2 (x + h) \sin(\frac{1}{x + h}) + (x + h)^2 \qty(\frac{-1}{(x+h)^2}) \cos(\frac{1}{x + h})
            }{1}\\
    &= \lim_{h \to 0} 2 (x + h) \sin(\frac{1}{x + h}) - \cos(\frac{1}{x + h})\\
    \Aboxed{
        f'(x)
        &= 2 x \sin(\frac{1}{x}) - \cos(\frac{1}{x})
        }
\end{align*}

%Part b
\subsection{
    $\cfrac{e^x + e^{-x}}{2}$
}
Let
\[
    f(x) = \cosh(x) = \cfrac{e^x + e^{-x}}{2}
\]
Then, by definition, the derivative of $f(x)$ is calculated as
\begin{align*}
    f'(x) 
    &= \lim_{h \to 0} \cfrac{f(x + h) - f(x)}{h}\\
    &= \lim_{h \to 0} 
        \cfrac{
            \cfrac{e^{(x+h)} + e^{-(x + h))}}{2} + \cfrac{e^x + e^{-x}}{2}
            }{h}\\
    &= \lim_{h \to 0}
        \cfrac{
            e^{x + h} + e^{-x - h} + e^{x} + e^{-x}
        }{
            2 h
        }\\
    &= \lim_{h \to 0}
        \cfrac{
            e^{x} e^{h} + e^{-x} e^{-h} + e^{x} + e^{-x}
        }{
            2 h
        }\\
    &= \lim_{h \to 0}
        \cfrac{
            \dv{h} \qty(e^{x} e^{h} + e^{-x} e^{-h} + e^{x} + e^{-x})
        }{
            \dv{h} 2 h
        }\\
    &= \lim_{h \to 0}
        \cfrac{
            e^{x} e^{h} - e^{-x} e^{-h}
        }{
            2
        }\\
    &= \lim_{h \to 0}
        \cfrac{
            e^{x + h} - e^{-x - h}
        }{
            2
        }\\
    \Aboxed{
        f'(x)
        &= \cfrac{e^{x} - e^{-x}}{2} = \sinh(x)
        }
\end{align*}

%Part c
\subsection{
    $\cfrac{e^x - e^{-x}}{2}$
}
Let
\[
    f(x) = \sinh(x) = \cfrac{e^x - e^{-x}}{2}
\]
Then, by definition, the derivative of $f(x)$ is calculated as
\begin{align*}
    f'(x) 
    &= \lim_{h \to 0} \cfrac{f(x + h) - f(x)}{h}\\
    &= \lim_{h \to 0} 
        \cfrac{
            \cfrac{e^{(x+h)} - e^{-(x + h))}}{2} + \cfrac{e^x - e^{-x}}{2}
            }{h}\\
    &= \lim_{h \to 0}
        \cfrac{
            e^{x + h} - e^{-x - h} + e^{x} - e^{-x}
        }{
            2 h
        }\\
    &= \lim_{h \to 0}
        \cfrac{
            \dv{h} \qty(e^{x + h} - e^{-x - h} + e^{x} - e^{-x})
        }{
            \dv{h} 2 h
        }\\
    &= \lim_{h \to 0}
        \cfrac{
            e^{x + h} + e^{-x - h}
        }{
            2
        }\\
    &= \lim_{h \to 0}
        \cfrac{
            e^{x + h} + e^{-x - h}
        }{
            2
        }\\
    \Aboxed{
        f'(x)
        &= \cfrac{e^{x} + e^{-x}}{2} = \cosh(x)
        }
\end{align*}

%Part d
\subsection{
    $e^x + e^{e^x} + e^{e^{e^x}}$
}
Let
\[
    f(x) = e^x + e^{e^x} + e^{e^{e^x}}
\]
Then, by definition, the derivative of $f(x)$ is calculated as
\begin{align*}
    f'(x) 
    &= \lim_{h \to 0} \cfrac{f(x + h) - f(x)}{h}\\
    &= \lim_{h \to 0} 
        \cfrac{e^{x+h} + e^{e^{x+h}} + e^{e^{e^{x+h}}} - \qty(e^x + e^{e^x} + e^{e^{e^x}})
            }{h}\\
    &= \lim_{h \to 0} \cfrac{e^{x+h} - e^x}{h}
        + \lim_{h \to 0} \cfrac{e^{e^{x + h}} - e^{e^x}}{h}
        + \lim_{h \to 0} \cfrac{e^{e^{e^{x + h}}} - e^{e^{e^x}}}{h}\\
    &= \lim_{h \to 0} \cfrac{
        \dv{h} \qty(e^{x+h} - e^x)
        }{\dv{h} h}
        + \lim_{h \to 0} \cfrac{
            \dv{h} \qty(e^{e^{x + h}} - e^{e^x})
            }{h}
        + \lim_{h \to 0} \cfrac{
            \dv{h} \qty(e^{e^{e^{x + h}}} - e^{e^{e^x}})
            }{\dv{h} h}\\
    &= \lim_{h \to 0} 
            \cfrac{(1)e^{x+h}}{1} 
        + \lim_{h \to 0} 
            \cfrac{
                (1)(e^{x+h})(e^{e^{x+h}})
                }{1}
        + \lim_{h \to 0} 
            \cfrac{
                (1)(e^{x+h})(e^{e^{x+h}})(e^{e^{e^{x+h}}})
                }{1}\\
    &= \lim_{h \to 0} e^{x+h}
        + \lim_{h \to 0} e^{x + h} e^{e^{x + h}}
        + \lim_{h \to 0} e^{x + h} e^{e^{x + h}} e^{e^{e^{x + h}}}\\
    \Aboxed{
        f'(x)
        &=  e^{x} + e^{x} e^{e^{x}} + e^{x} e^{e^{x}} e^{e^{e^{x}}}
        = e^{x} + e^{x + e^{x}} + e^{x + e^{x} + e^{e^{x}}}
    }
\end{align*}

\newpage
%Part e
\subsection{
    $x^{x^{x^x}}$
}
Let
\[
    f(x) = x^{x^{x^{x}}}
\]
Then, by definition, the derivative of $f(x)$ is calculated as
\begin{align*}
    f'(x) 
    &= \lim_{h \to 0} \cfrac{f(x + h) - f(x)}{h}\\
    &= \lim_{h \to 0} 
        \cfrac{
            (x + h)^{(x + h)^{(x + h)^{(x + h)}}} - x^{x^{x^{x}}}
        }{
            h
        }\\
    &= \lim_{h \to 0} 
        \cfrac{
            \dv{h} \qty((x + h)^{(x + h)^{(x + h)^{(x + h)}}} - x^{x^{x^{x}}})
        }{
            \dv{h} h
        }\\
    &= \lim_{h \to 0}
        \cfrac{
            \dv{h} \qty((x + h)^{(x + h)^{(x + h)^{(x + h)}}})
        }{
            1
        }
    \intertext{
        Exponent Rule: $a^b = e^{b \ln{a}}$
    }
    &= \lim_{h \to 0} \dv{h} \qty(
        e^{\qty(
            (x + h)^{(x + h)^{(x + h)}}
            \ln(x + h)
        )}
    )
    \intertext{
        Chain Rule: $\dv{f}{x} = \dv{f}{u} \dv{u}{x}$
    }
    &= \lim_{h \to 0} 
        \dv{e^{u}}{u}\dv{u}{h}
    \intertext{
        with $u = (x + h)^{(x + h)^{(x + h)}} \ln(x + h)$
    }
    &= \lim_{h \to 0} e^{u} \dv{h} \qty(
        (x + h)^{(x + h)^{(x + h)}} \ln(x + h)
    )\\
    &= \lim_{h \to 0} e^{u} \Big[
        \qty((x + h)^{(x + h)^{(x + h)}}) \frac{1}{x + h}\\
        &\quad + \ln(x + h) \dv{h} e^{
            (x + h)^{(x + h)}
            \ln(x + h)
        }
    ]\\
    &\vdots\\
    &= \lim_{h \to 0}
        (x + h)^{(x + h)^{(x + h)^{(x + h)}}} \\
        &\qquad \Bigg[(x + h)^{(x + h)^{(x + h)}-1}\\
            &\qquad \qquad + \Bigg(
                (x + h)^{(x + h)^{(x + h)}} \ln(x + h) \Big(
                    (x + h)^{h + x - 1}\\
                    &\qquad \qquad \qquad \qquad 
                    + (x + h)^{x + h} \ln(x + h) \qty(
                        1 + \ln(x + h)
                    )
                \Big)
            \Bigg)
        \Bigg]
\end{align*}

Then
\[\boxed{
    f'(x) = x^{x^{x^{x}}} \qty(
        x^{x^{x}}\ln(x) \qty(
            x^{x}\ln(x) \qty(\ln(x) + 1) + x^{x^{x - 1}}
        )
        +x^{x^{x} - 1}
    )
}\]

% Problem 2
\newpage
\section{}

% Differentiable Function
\begin{definition}
    A \emph{\underline{differentiable function}} over $(a,b)$ is a function $f$ so that $f'(x)$ exists for all $x \in (a,b)$.
\end{definition}

% Bounded Function
\begin{definition}
    A \emph{\underline{bounded function}} is a function so that 
    \[
        \exists_{N \in \R} \st \forall_{x \in (a,b)} \abs{f(x)} < N
    \]
\end{definition}

% Unbounded Function
\begin{definition}
    An \emph{\underline{unbounded function}} is a function that is not bounded.
\end{definition}

\begin{definition}
    A function $f : X \to Y$ is \emph{\underline{Uniformly Continous}} if 
    \[
        \forall_{\epsilon > 0} \exists_{\delta(\epsilon) > 0} \st \forall_{x,y \in X} \norm{x - y} < \delta \implies \norm{f(x) - f(y)} < \epsilon
    \]
    Note: this is stricter then continuity itself with the only difference that $\exists_{\delta}$ is constant $\forall_{x,y \in X}$; whereas a \emph{\underline{Continous}} function only requires that this is true for some $\delta$ dependent on $\epsilon$, $x$, and $y$:
    \[
        \forall_{\epsilon > 0} \forall_{x,y \in X} \exists_{\delta(\epsilon,x,y) > 0} \st \norm{x - y} < \delta \implies \norm{f(x) - f(y)} < \epsilon
    \]
\end{definition}

%Part a
\subsection{Prove the following:}
\begin{theorem}
    If $f \st (-1,1) \to \R$ is a differentiable unbounded function, then $f'$ is also unbounded on $[-1,1]$.
    \begin{proof}
        When $f$ is unbounded,
        \[
            \forall_{N \in \R} \exists_{x \in (-1,1)} \st \abs{f(x)} \geq N
        \]
        then $f'$ is defined by
        \[
            f'(x) := \lim_{h \to 0} \cfrac{f(x + h) - f(x)}{h}
        \]
        Since the derivative is an instantaneous 'rise' over 'run', in order for no bound to exists within a bounded interval $(-1,1)$, the differentiable (and therefore continuous) function must have an asymptote at the boundaries. This implies that within the closed interval, the derivative must also be unbounded to obtain the the asymptote on the boundary.
    \end{proof}
\end{theorem}

%Part b
\subsection{
    Provide an example of bounded differentiable function on $[-1,1]$ with an unbounded derivative.
}
\begin{theorem}
    $f(x) = \sqrt{x + 1}$ is a bounded differentiable function on $[-1,1]$, but has an unbounded derivative.
    \begin{proof}
        Clearly, $f(x)$ is fully defined and bounded on $[-1,1]$.
        The domain of $\sqrt{x + 1}$ is $\{x \in \R \st x \geq -1\}$ while the range for $x \in [-1,1]$ is $\{y \in [0,\sqrt(3)]\}$.

        The derivative of $f$ is defined as 
        \[
            f'(x) = \cfrac{1}{2 \sqrt{x + 1}}
        \]
        which is actually bounded on $x \in (-1,1]$, but an asymptote exists at $x = -1$ in which 
        \[
            \lim_{x \to -1^+} \cfrac{1}{2 \sqrt{x + 1}} = + \infty
        \]
        and is clearly unbounded.
    \end{proof}
\end{theorem}

%Part c
\subsection{Prove the following:}
\begin{theorem}
    If $f \st (-1,1) \to \R$ is a differentiable function, such that $f'$ is bounded on $[-1,1]$, then $f$ is uniformly continuous.
    \begin{proof}
        $f$ differentiable on $x \in (-1,1)$ means 
        \[
            \exists f'(x) = \lim_{h \to 0} \cfrac{f(x + h) - f(x)}{h}
        \]
        $f'$ bounded on $x \in [-1,1]$ means $f'(x)$ is bounded on $x \in (-1, 1)$ and that the following is bounded:
        \[
            \lim_{h \to 0^{+/-}} \cfrac{f(x + h) - f(x)}{h}
        \]
        which is equivalent to saying that these two limits exists and are bounded.
        \[
            \lim_{x \to (+/-)1^{+/-}} f(x)
        \]
        Since it is already known that $f$ is differentiable on $x \in (-1,1)$, this implies that $f$ is continuous, i.e:
        \[
            \forall_{\epsilon > 0} \forall_{x,y \in (-1,1)} \exists_{\delta(\epsilon,x,y) > 0} \st \norm{x - y} < \delta \implies \norm{f(x) - f(y)} < \epsilon
        \]
        and since $f'$ is bounded on the entire closed domain, a constant finite bound $\delta$ will exist to bound the 'run' ($\norm{x - y}$) for an arbitrary 'rise' ($\norm{f(x) - f(y)}$).
        Additionally, the lack of asymptotes at the boundaries also implies that continuity of the open domain will extend to the closed domain.
        Therefore $f$ is uniformly continuous, i.e:
        \[
            \forall_{\epsilon > 0} \exists_{\delta(\epsilon) > 0} \st \forall_{x,y \in [-1,1]} \norm{x - y} < \delta \implies \norm{f(x) - f(y)} < \epsilon
        \]
    \end{proof}
\end{theorem}

% Problem 3
\newpage
\section{}
Find $f^{(n)}(0)$ for the following functions:

% n-th Derivative
\begin{definition}
    Let $f : (a,b) \to \R$ be a differentiable function.
    \begin{enumerate}
        \item If $f' : (a,b) \to \R$ is also differentiable, $f$ is called \emph{\underline{twice differentiable}}.
        \item The derivative of $f'$ is then denoted as $f''$ and is called the \emph{\underline{Secound Derivative}}.
        \item The \emph{\underline{n-th Derivative}}, denoted by $f^{(n)} : (a,b) \to \R, n \in \N$, is defined by repeating differentiation on the next derivative:
        \begin{enumerate}
            \item $n = 0, \ f^{(0)} = f$
            \item If $f^{(n)}$ is defined, for $n \geq 0$, then $f^{(n+1)} = \dv{x} (f^{(n)})$.
        \end{enumerate}
        \item If $f$ has derivatives up until the order $n, n \geq 1$, then $f$ is \emph{\underline{n-times differentiable}}.
        \item If $f^{(n)}$ is continuous, the $f$ is \emph{\underline{n-times continously differentiable}}; which is also known as $f$ is of class $C^n$.
    \end{enumerate}
\end{definition}

% Leibnitz Formula
\begin{theorem}
    \textbf{Leibnitz Formula:}
    Let $f,g : (a,b) \to \R$ be n-differentiable functions. 
    Then for $1\leq m \leq n$, the m-th derivative of $f(x)g(x)$ is given by:
    \[
        \dv[m]{x}\qty(f(x) g(x)) = \sum_{k=0}^m \mqty(m \\ k) f^{(m-k)}(x) g^{(k)}(x)
    \]
    \begin{proof}
        Proof provided in the lecture notes: Theorem 6.20
    \end{proof}
\end{theorem}

%Part a
\subsection{
    $\sin(ax)\cos(bx)$
}
It is known, and can be proven using Euler's identity, that the derivatives of sinusoidal functions follow the following progression:
    \begin{enumerate}
        \item $\dv{x} \cos(x) = - \sin(x)$
        \item $\dv{x} -\sin(x) = - \cos(x)$
        \item $\dv{x} -\cos(x) = \sin(x)$
        \item $\dv{x} \sin(x) = \cos(x)$
    \end{enumerate}
Additionally, from the chain rule, the m-th derivative of a domain scaled function, $f(a x)$ will be given by
\[
    \dv[m]{x} f(a x) = a^{m} f^{(m)}(u), \ u = a x
\]

Let $f_1(x) = \sin(ax)$ and $f_2(x) = \cos(bx)$, by Euler's formula,
\[
    f_1(x) = \sin(ax) = \cfrac{e^{j a x} - e^{- j a x}}{2 j}
\]
and
\[
    f_2(x) = \cos(bx) = \cfrac{e^{j b x} + e^{-j b x}}{2}
\]

The m-th derivatives can be defined in many ways, but the following begins by recognizing that $g(x) = a^{-1} f'(x) = a^{-1} f^{(1)}(x)$; and additionally, $g^{(m)}(x) = a^{-m - 1} b^{m} f^{(m + 1)}(\frac{b}{a}x)$.


The m-th derivatives of $f$ and $g$ are then clearly given as:
\[
    f_1^{(m)}(x) 
    = a^m (j)^{m} \cfrac{e^{j a x} - (-1)^{m} e^{- j a x}}{j 2}
    = a^m (j)^{m - 1} \cfrac{e^{j a x} + (-1)^{m - 1} e^{- j a x}}{2}
\]
and 
\[
    f_2^{(m)}(x) 
    = b^m (j)^{m} \cfrac{e^{j b x} + (-1)^{m} e^{- j b x}}{2}
\]

By the Leibnitz Formula, the m-th derivative of $f_1(x) f_2(x)$ can be calculated as follows:
\[
    \dv[m]{x}\qty(f_1(x) g_2(x)) 
    = \sum_{k=0}^m \mqty(m \\ k) f_1^{(m-k)}(x) f_2^{(k)}(x)
\]
\begin{align*}
    \dv[m]{x}\qty(\sin(ax) \cos(bx)) 
        &= \sum_{k=0}^m \mqty(m \\ k) 
            \qty(a^{m - k} (j)^{m - k - 1} \cfrac{e^{j a x} + (-1)^{m - k - 1} e^{- j a x}}{2})
            \qty(b^k (j)^{k} \cfrac{e^{j b x} + (-1)^{k} e^{- j b x}}{2})\\
    \Aboxed{
    \dv[m]{x}\qty(\sin(ax) \cos(bx)) 
        &= \sum_{k=0}^m \mqty(m \\ k) \qty(a^{m - k} b^k) 
            \qty((j)^{m - k - 1} \cfrac{e^{j a x} + (-1)^{m - k - 1} e^{- j a x}}{2})
            \qty((j)^{k} \cfrac{e^{j b x} + (-1)^{k} e^{- j b x}}{2})
    }\\
\end{align*}
Which, using Euler's formula again, can be converted back into many different configurations of sinusoidal forms, but regardless, this can be evaluated for $f^{(m)}(0)$ within this complex exponential form:
\begin{align*}
    f^{(m)} (0) 
    &= \sum_{k=0}^m \mqty(m \\ k) \qty(a^{m - k} b^k) 
        \qty((j)^{m - k - 1} \cfrac{e^{j a (0)} + (-1)^{m - k - 1} e^{- j a (0)}}{2})
        \qty((j)^{k} \cfrac{e^{j b (0)} + (-1)^{k} e^{- j b (0)}}{2})\\
    &= \sum_{k=0}^m \mqty(m \\ k) \qty(a^{m - k} b^k) 
        \qty((j)^{m - k - 1} \cfrac{(1) + (-1)^{m - k - 1} (1)}{2})
        \qty((j)^{k} \cfrac{(1) + (-1)^{k} (1)}{2})\\
    &= \sum_{k=0}^m \mqty(m \\ k) \qty(a^{m - k} b^k) 
        \qty((j)^{m - k - 1 + k} \cfrac{(1) + (-1)^{m - k - 1}}{2})
        \qty[\begin{cases}
            \qty(\cfrac{(1) + (-1)^{k}}{2})
            &k \divisible 2\\
            \qty(\cfrac{(1) + (-1)^{k}}{2})
            &(k - 1) \divisible 2
        \end{cases}]
    \intertext{
        Clearly, when $(k-1) \divisible 2$, $\frac{1 + (-1)^k}{2} = 0$, 
        otherwise, $\frac{1 + (-1)^k}{2} = 1$.
    }
    &= \sum_{k=0}^m \mqty(m \\ k) \qty(a^{m - k} b^k) 
        \qty((j)^{m - 1} \cfrac{(1) + (-1)^{m - k - 1}}{2})
        \qty[\begin{cases}
            (1)
            &k \divisible 2\\
            (0)
            &(k - 1) \divisible 2
        \end{cases}]\\
    &= \sum_{k=0}^m \mqty(m \\ k) \qty(a^{m - k} b^k (j)^{m - 1})
        \qty[\begin{cases}
            \cfrac{1 - 1}{2} = 0
            &(k \divisible 2) \land (m - k \divisible 2)\\
            \cfrac{1 + 1}{2} = 1
            &(k \divisible 2) \land (m - k - 1 \divisible 2)\\
            0
            &(k - 1) \divisible 2
        \end{cases}]
    \intertext{
        Since $m - 1 \divisible 2 \implies (k \divisible 2) \land (m - k - 1 \divisible 2)$,
    }
    &= \begin{cases}
        \sum_{k=0}^m \mqty(m \\ k)
        \qty[\begin{cases}
            a^{m - k} b^k (j)^{m - 1}
            &k \divisible 2\\
            0
            &(k - 1) \divisible 2
        \end{cases}]
        & m - 1 \divisible 2\\
        0 & m \divisible 2
    \end{cases}
\end{align*}
Ultimately this means that 
\[
    f^{(n)}(0) = \begin{cases}
        \sum_{k = 1}^{m/2} a^{m - 2 k} b^{2 k}
        &(m  - 1 \divisible 2) \land \lnot (m \divisible 4)\\
        - \sum_{k = 1}^{m/2} a^{m - 2 k} b^{2 k}
        &(m - 1 \divisible 2) \land (m - 1 \divisible 4)\\
        0
        & m \divisible 2
    \end{cases}
\]
    
%Part b
\subsection{
    $x^k \sin{\frac{1}{x}}$
}
Assuming that $k \in \N$, Let
\[
    f(x) = x^k \sin{\frac{1}{x}}
\]

Let $f_1(x) = x^k$ and $f_2(x) = \sin(\frac{1}{x})$.

By Leibnitz Formula, the m-th derivative of $f_1(x) f_2(x)$ can be calculated as follows:
\[
    \dv[m]{x}\qty(f_1(x) g_2(x)) 
    = \sum_{k=0}^m \mqty(m \\ k) f_1^{(m-k)}(x) f_2^{(k)}(x)
\]
Therefore,
\[
    \dv[m]{x} x^k \sin(\frac{1}{x})
        = \sum_{l = 0}^m \mqty(m \\ l)
            \dv[m - l]{x} x^k
            \dv[l]{x} \sin(\frac{1}{x})
\]

It is known by the power rule, that
\[
    \dv[m]{x} x^k = \begin{cases}
        k^{m} x^{k - m} &m \leq k\\
        0 &m > k
    \end{cases}
\]

Since $f^{m}(0)$ will always let $x = 0$, 
\[
    \dv[m]{x} x^k = 0 \forall_{m}
\]

Therefore,
\begin{align*}
    \dv[m]{x} x^k \sin(\frac{1}{x})
        &= \sum_{l = 0}^m \mqty(m \\ l)
            \dv[m - l]{x} x^k
            \dv[l]{x} \sin(\frac{1}{x})\\
        &= \sum_{l = 0}^m \mqty(m \\ l)
            (0)
            \dv[l]{x} \sin(\frac{1}{x})
\end{align*}
\[
    \boxed{f^{m}(0) = 0}
\]

\newpage
%Part c
\subsection{
    $f(x) =
    \begin{cases}
        e^{-\frac{1}{x^2}}, &x > 0\\
        0,                  &x \leq 0
    \end{cases}$
}

Since $\lim_{x \to 0^+} - \frac{1}{x^2} = - \infty$,
\[
    \lim_{x \to 0^+} e^{-\frac{1}{x^2}} = 0
\]

Therefore, the function is smooth and differentiable on all $\R$.

Additionally, since $\forall_{x \leq 0} f(x) = 0$,
\[
    \forall_{x \leq 0} f'(x) = 0
\]

By the Chain rule,
\[
    \forall_{x \leq 0} \ \dv{x} e^{- \frac{1}{x^2}} = \frac{2}{x^3} e^{-\frac{1}{x^2}}
\]
Similarly, by the chain rule and product rule,
\[
    \forall_{x \leq 0} \ \dv[2]{x} e^{- \frac{1}{x^2}} 
    = \dv{x} \frac{2}{x^3} e^{-\frac{1}{x^2}}
    = \frac{-6}{x^4} e^{-\frac{1}{x^2}} + \frac{2}{x^3} \frac{2}{x^3} e^{-\frac{1}{x^2}}
    = \frac{2}{x^6} (2 - 3 x^2) e^{-\frac{1}{x^2}}
\]
This can then be expanded further with Leibnitz rule,
to show that 
\[
    \forall_{x \leq 0} \ \dv[m]{x} e^{- \frac{1}{x^2}}
    = \sum_{k = 0}^m \mqty(m \\ k) \qty(\frac{2}{x^3})^k e^{-\frac{1}{x^2}}
\]
and since $\lim_{x \to 0^+} e^{-\frac{1}{x^2}} = 0$,
\[
    \lim_{x \to 0^+} \dv[m]{x} e^{- \frac{1}{x^2}} = 0
\]
And therefore $f^{(n)}(0) = 0$ since the two limits are the same. 
Although technically, the understanding of the smoothness of $f(x)$ and the fact that $f(0) = 0$ should have been enough to demonstrate this...

% Problem 4
\newpage
\section{}
Construct an example of infinitely many times differentiable function $f(x)$ such that $f(x) = 0$ for $x \leq 0$, $f(x) = 1$ for $x \geq 1$ and $f(x)$ is strictly monotone on the interval $(0,1)$.

Using such function you could construct, for example, a monotone function $g(x)$ such that $\lim_{x \to +\infty} g(x) = 0$ but $\lim_{x \to +\infty} g'(x) \neq 0$. (How?)

\begin{definition}
    Let $f : \R \to \R$ be defined as
    \[
        f(x) := \begin{cases}
            0   &x \leq 0\\
            \frac{1 - \cos(\pi x)}{2} & x \in (0,1)\\
            1   &x \geq 1
        \end{cases}
    \]   
\end{definition}


\begin{theorem}
    $f$ is a strictly monotone function over $(0,1)$, is infinitely differentiable function, has $f(x) = 0$ for $x \leq 0$ and $f(x) = 1$ for $x \geq 1$.
    \begin{proof}
        \begin{enumerate}
            \item $f$ is strictly monotone over $(0,1)$ means
                \[
                    \forall_{x \in (0,1)} \ f'(x) > 0 
                \]
                Since within the interval $(0,1)$, 
                $f(x) = cos(\pi x)$, 
                \[
                    f'(x) = \pi \sin(\pi x)
                \]
                which is strictly positive over $(0,1)$ implying $f(x)$ is strictly increasing.
                By the definition of a monotone, $f$ is therefore a monotone function.
            \item $f$ being infinitely differentiable means that the n-th derivative is defined over the entire interval forall $n \in \N$.
                
                The nth derivative is in fact defined defined by the following, 
                \[
                    f^{(n)}(x) = \begin{cases}
                        0 & x \in (-\infty, 0] \cup [1,\infty)\\
                        \begin{cases}
                            \pi^n \cos(\pi x) &n \divisible 4\\
                            -\pi^n \sin(\pi x) &n + 1 \divisible 4\\
                            -\pi^n \cos(\pi x) &n + 2 \divisible 4\\
                            \pi^n \sin(\pi x) &n + 3 \divisible 4
                        \end{cases}
                        &0 < x < 1
                    \end{cases}
                \]
            \item $f(x) = 0$ for $x \leq 0$ is true by the piecewise definition of $f(x)$
            \item $f(x) = 1$ for $x \leq 1$ is true by the piecewise definition of $f(x)$
        \end{enumerate}
    \end{proof}
\end{theorem}

Perhaps it is possible to complete use such a function to construct the desired function $g(x)$ using one of the monotone functions on that region, but my guess is that such a function will be monotone forall $x$ instead of only within the region as described like I did with the cosine function.
Perhaps a sigmoid variant would be useful, but I believe that any variation based on the cosine function with the piecewise will not satisfy that condition (at least with all real-values).

% \newpage
% \begin{definition}
%     Let $g : (0,\infty) \to \R$ be defined in terms of $f$ as:
%     \[
%         g(x) := x\cfrac{f(x) - 1}
%         = \begin{cases}
%             \cfrac{e^{x}(-1 - \cos(\pi x))}{2x}
%             & 0 < x < 1\\
%             0 & x \geq 0
%         \end{cases}
%     \]
% \end{definition}

% \begin{theorem}
%     The function $g(x)$ converges to 0, 
%     \[
%         \lim_{x \to +\infty} g(x) = 0
%     \]
%     but the same is not the case for $g'(x)$:
%     \[
%         \lim_{x \to +\infty} \neq 0
%     \]
%     \begin{proof}
%         \begin{enumerate}
%             \item Clearly, by definition of $g(x)$, $g(x)$ converges to 0.
%             \item The derivative does not converge to zero:
%                 \begin{align*}
%                     g'(x) = \begin{cases}
                        
%                     \end{cases}
%                 \end{align*}
%         \end{enumerate}
%     \end{proof}
% \end{theorem}

% Problem 5
\newpage
\section{}
Find the following limits

Trigonometric Derivative Reference:
\begin{align*}
    \dv{x} \sin(x) &= \cos(x)\\
    \dv{x} \cos(x) &= -\sin(x)\\
    \dv{x} \tan(x) &= \secant^2(x)\\
    \dv{x} \sin[-1](x) &= \frac{1}{\sqrt{1-x^2}}\\
    \dv{x} \cos[-1](x) &= -\frac{1}{\sqrt{1-x^2}}\\
    \dv{x} \tan[-1](x) &= \frac{1}{1 + x^2}
\end{align*}

%Part a
\subsection{
    $\lim_{x \to 0} \frac{\tan{x} - x}{x^3}$
}

\begin{align*}
    \lim_{x \to 0} \frac{\tan{x} - x}{x^3}
        &= \lim_{x \to 0} \frac{\tan{x}}{x^3} - \frac{x}{x^2}\\
        &= \lim_{x \to 0} 
            \frac{\dv{x} \tan{x} - x}{\dv{x} x^3}\\
        &= \lim_{x \to 0}
            \frac{\sec^2{x} - 1}{3 x^2}\\
        &= \lim_{x \to 0}
            \frac{2 \sec^2{x}\tan{x}}{6 x}\\
        &= \lim_{x \to 0}
            \frac{(4 \sin[2](x) + 2) \sec[4](x)}{6}\\
        &= \frac{(4 (0)^2 + 2) (1)^4}{6}\\
    \Aboxed{
        \lim_{x \to 0} \frac{\tan{x} - x}{x^3}
        &= \frac{2}{6} = \frac{1}{3}
    }
\end{align*}

\newpage
%Part b
\subsection{
    $\lim_{x \to 0} \frac{\arctan(\arcsin{x}) - \arcsin(\arctan{x})}{\sin{x} - \tan{x}}$
}

\begin{align*}
    \lim_{x \to 0} \frac{\arctan(\arcsin{x}) - \arcsin(\arctan{x})}{\sin{x} - \tan{x}}
    &= \lim_{x \to 0}
        \frac{
            \dv{x} \qty(\tan[-1](\sin[-1](x)) - \sin[-1](\tan[-1](x)))
        }{
            \dv{x} \qty(\sin(x) - \tan(x))
        }\\
    &= \lim_{x \to 0}
        \frac{
            \dv{\tan[-1](\sin[-1](x))}{\sin[-1](x)} \dv{\sin[-1](x)}{x}
            - \dv{\sin[-1](\tan[-1](x))}{\tan[-1](x)}\dv{\sin[-1](x)}{x}
        }{
            \dv{\sin(x)}{x} - \dv{\tan(x)}{x}
        }\\
    &= \lim_{x \to 0}
        \frac{
            \frac{1}{\sqrt{1-x^2} (1 + \sin[-1](x)^2)} 
            - \frac{1}{\sqrt{1+x^2} (1 - \tan[-1](x)^2)}
        }{
            \cos(x) - \sin(x)
        }\\
    &= \lim_{x \to 0}
        \frac{
            \frac{
                - (\sqrt{1-x^2} (1 + \sin[-1](x)^2))
                + (\sqrt{1+x^2} (1 - \tan[-1](x)^2))
            }{
                (\sqrt{1-x^2} (1 + \sin[-1](x)^2))
                (\sqrt{1+x^2} (1 - \tan[-1](x)^2))
            }
        }{
            \cos(x) - \sin(x)
        }\\
    &= \frac{
            \frac{
                -(\sqrt{1 - (0)} (1 + (0)^2))
                +(\sqrt{1 + (0)} (1 - (0)^2))
            }{
                (\sqrt{1-0^2} (1 + (0)^2))
                (\sqrt{1+0^2} (1 - (0)^2))
            }
        }{
            (1) - (0)
        }\\
    &= \frac{
        \frac{
            -1 + 1
        }{
            (1)(1)
        }
    }{1}
\end{align*}
\begin{equation*}
    \centering
    \boxed{
        \lim_{x \to 0} \frac{\arctan(\arcsin{x}) - \arcsin(\arctan{x})}{\sin{x} - \tan{x}}
        = 0
    }
\end{equation*}

%Part c
\subsection{
    $\lim_{x \to +\infty} \frac{x^{\ln{x}}}{(\ln{x})^x}$
}

\begin{align*}
    \lim_{x \to +\infty} \frac{x^{\ln{x}}}{(\ln{x})^x}
    &= \lim_{x \to + \infty}
        x^{\ln(x)} \ln(x)^{-x}\\
    &= \lim_{x \to + \infty}
        e^{\ln(x^{\ln(x)})} e^{\ln(\ln(x)^{-x})}\\
    &= \lim_{x \to + \infty}
        e^{\ln(x) \ln(x)} e^{- x \ln(\ln(x))}\\
    &= \lim_{x \to + \infty}
        \frac{
            e^{(\ln(x))^2}
        }{
            e^{x \ln(\ln(x))}
        }\\
    &= \lim_{x \to + \infty}
        \frac{
            \dv{x} e^{(\ln(x))^2}
        }{
            \dv{x} e^{x \ln(\ln(x))}
        }\\
    &= \lim_{x \to + \infty}
        \frac{
            \frac{2 \ln(x)}{x} e^{(\ln(x))^2}
        }{
            \ln^x(x) \qty(\ln(\ln(x))+ \frac{1}{\ln(x)})
        }\\
    &= \lim_{x \to + \infty}
        \frac{
            2 \ln(x) e^{\ln^2(x)}
        }{
            x \ln^x(x) \qty(\ln(\ln(x))+ \frac{1}{\ln(x)})
        }\\
    % &= \lim_{x \to +\infty} 
    %     \frac{
    %         \dv{x} x^{\ln{x}}
    %     }{
    %         \dv{x} (\ln{x})^x
    %     }\\
    % &= \lim_{x \to +\infty} 
    %     \frac{
    %         \dv{x^{\ln{x}}}{\ln{x}}{x} \dv{\ln{x}}{x}
    %     }{
    %         \dv{x} e^{\ln{(\ln{x})^x}}
    %     }\\
    % &= \lim_{x \to +\infty} 
    %     \frac{
    %         \ln{x} x^{\ln{x} - 1} \frac{1}{x}
    %     }{
    %         \dv{
    %             x \ln(\ln(x))
    %         }{x}
    %         e^{x \ln(\ln{x})}
    %     }\\
    % &= \lim_{x \to +\infty} 
    %     \frac{
    %         \frac{\ln{x} x^{\ln{x} - 1}}{x}
    %     }{
    %         \qty(
    %             \ln(\ln(x)) 
    %             + \frac{1}{\ln(x)}
    %         )
    %         e^{
    %             x \ln(\ln{x})
    %         }
    %     }\\
    % &= \lim_{x \to +\infty}
    %     \frac{
    %         x^{\ln(x) - 2} \ln(x)
    %     }{
    %         \qty(\ln(x))^x
    %         \qty(
    %             \ln(\ln(x) + \frac{1}{\ln(x)})
    %         )
    %     }\\
    % &= \lim_{x \to +\infty}
    %     \frac{
    %         \dv{x} x^{\ln(x) - 2} \ln(x)
    %     }{
    %         \dv{x} \qty(\ln(x))^x
    %         \qty(
    %             \ln(\ln(x) + \frac{1}{\ln(x)})
    %         )
    %     }\\
    % &= \lim_{x \to +\infty}
    %     \cfrac{
    %         x^{\ln(x) - 3}
    %         \qty(
    %             2 (\ln(x))^2 - 2 \ln(x) + 1
    %         )
    %     }{
    %         x^{-1}
    %         (\ln(x))^{x - 2}
    %         \qty(
    %             x 
    %             + x (\ln(x))^2 (\ln(\ln(x)))^2
    %             + 2 x \ln(x) \ln(\ln(x))
    %             + \ln(x)
    %             - 1
    %         )
    %     }\\
    % &= \lim_{x \to +\infty}
    %     \cfrac{
    %         x^{\ln(x)}
    %         (\ln(x))^2
    %         \qty(
    %             2 (\ln(x))^2 - 2 \ln(x) + 1
    %         )
    %     }{
    %         x^4
    %         (\ln(x))^{x}
    %         \qty(
    %             x 
    %             + x (\ln(x))^2 (\ln(\ln(x)))^2
    %             + 2 x \ln(x) \ln(\ln(x))
    %             + \ln(x)
    %             - 1
    %         )
    %     }
\end{align*}

After a lot of manipulation I am making the assumption that with
$\lim_{x \to + \infty} \ln{x} = + \infty$
and since for large enough $x$, 
$x^{\ln(x)} > (\ln(x))^x$
the resulting ratio will be unbounded and will tend to infinity.

It is also true that
$e^{\ln^2(x)} > \ln^x(x)$
and ultimately the original hypothesis is supported as well.

Therefore, 
\[
    \lim_{x \to + \infty} \frac{x^{\ln(x)}}{(\ln(x))^x} = + \infty
\]

% Problem 6
\newpage
\section{}
Find the example of a function $f(x)$ which is continuous at every point of the interval $(0,1)$, but is not differentiable at every point $(0,1)$.

Read about the construction of the function, which is differentiable at every point of $(0,1)$ but whose derivative is continuous at every point of $(0,1)$.
















\end{document}
