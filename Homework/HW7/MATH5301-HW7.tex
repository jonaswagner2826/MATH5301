% Standard Article Definition
\documentclass[]{article}

% Page Formatting
\usepackage[margin=1in]{geometry}
\setlength\parindent{0pt}

% Graphics
\usepackage{graphicx}

% Math Packages
\usepackage{physics}
\usepackage{amsmath, amsfonts, amssymb, amsthm}
\usepackage{mathtools}

% Code Def
\usepackage{listings}

% Section Heading Settings
\usepackage{enumitem}
\renewcommand{\theenumi}{\alph{enumi}}
\renewcommand*{\thesection}{Problem \arabic{section}}
\renewcommand*{\thesubsection}{\alph{subsection})}
\renewcommand*{\thesubsubsection}{\quad \quad \roman{subsubsection})}

%Custom Commands
\newcommand{\Rel}{\mathcal{R}}
\newcommand{\R}{\mathbb{R}}
\newcommand{\C}{\mathbb{C}}
\newcommand{\N}{\mathbb{N}}
\newcommand{\Z}{\mathbb{Z}}
\newcommand{\Q}{\mathbb{Q}}

\newcommand{\toI}{\xrightarrow{\textsf{\tiny I}}}
\newcommand{\toS}{\xrightarrow{\textsf{\tiny S}}}
\newcommand{\toB}{\xrightarrow{\textsf{\tiny B}}}

\newcommand{\divisible}{ \ \vdots \ }
\newcommand{\st}{\ : \ }


% Theorem Definition
\newtheorem{definition}{Definition}
\newtheorem{assumption}{Assumption}
\newtheorem{theorem}{Theorem}
\newtheorem{lemma}{Lemma}
\newtheorem{proposition}{Proposition}


%opening

\title{MATH 5301 Elementary Analysis - Homework 7}

\author{Jonas Wagner}

\date{2021, October 22\textsuperscript{nd}}

\begin{document}

\maketitle

% Problem 1 ----------------------------------------------
\section{Provide examples of the sets $A,B \subset \R^2$ so that:}

\begin{definition}
    Set $A$ is \emph{connected} if it is not disconnected.
\end{definition}
\begin{definition}
    Set $A$ is \emph{disconnected} if $\forall U, V$ - open: 
    \begin{enumerate}
        \item $U \cap A \neq \emptyset$, $V \cap A \neq \emptyset$
        \item $\qty(U \cap V) \cap A \neq \emptyset$ \quad \underline{Note:} this implies $U \neq \emptyset$ and $V \neq \emptyset$
        \item $A \subset U \cup V \neq \emptyset$
    \end{enumerate}
\end{definition}

%Part a
\subsection{$A$ and $B$ are connected but $A \cup B$ is not.}
Sets $A$ and $B$ are individually connected but that just aren't near each other. An example are these open balls:
\begin{align*}
    A = \mathcal{B}_{r}((0,0))
        &:= \qty{(x,y)\in\R \st x^2 + y^2 < r^2}\\
    B = \mathcal{B}_{r}((-5r,5r))
        &:= \qty{(x,y)\in\R \st (x+5r)^2 + (y-5r)^2 < r^2}
\end{align*}

%Part b
\subsection{$A$ and $B$ are connected but $A \cap B$ is not.}
Sets $A$ and $B$ are individually connected, but they intersect in two areas resulting in a disjoint intersection... Such as this donut and square sets:
\begin{align*}
    A = \mathcal{B}^{(2)}_{R}((0,0)) \backslash \mathcal{B}^{(2)}_{r}((0,0))
        &= \qty{(x,y)\in\R \st r^2 < x^2 + y^2 < R^2}\\
    B = \mathcal{B}^{(\infty)}_{r}((0,0))
        &= \qty{(x,y) \in \R \st x < r \land y < r}
\end{align*}

%Part c
\subsection{$A$ and $B$ are not connected but $A \cup B$ is connected.}
$A$ and $B$ individually are disjoint shapes but each individual disjointed part intersects with one from the other set. An example would be a couple of disjoint (then connected when a union) regions.
\begin{align*}
    A &:= \qty{(x,y) \in \R \st x \in [-1,0] \cup [1,2] \land y \in [-2,2]}\\
    B &:= \qty{(x,y) \in \R \st x \in [-2,1.25] \cup [1.75,4] \land y \in [-1,3]}
\end{align*}

%Part d
\subsection{$A$ and $B$ are not connected but $A \cap B$ is connected.}
$A$ and $B$ individually are disjoint shapes, there is then only one connected region where they intersect. An example is two overlapping regions, each with random other disjoint parts.
\begin{align*}
    A &:= \qty{(x,y) \in \R \st x \in [0,1] \cup [15,30], y \in [0,1]}\\
    B &:= \qty{(x,y) \in \R \st x \in [-1, 1] \cup [-15,-20], y \in [0,1]}
\end{align*}

%Part e
\subsection{$A$ and $B$ are not connected but $A \ \backslash \ B$ is connected.}
$A$ and $B$ individually are disjoint shapes, with a part of $B$ covering all but one disjoint region of $A$. An example is a two overlapping regions with disjoint parts and $A$ having one (only one) disjoint part not fully covered by $B$.
\begin{align*}
    A &:= \qty{(x,y) \in \R \st x \in [0,1] \cup [15,30], y \in [0,1]}\\
    B &:= \qty{(x,y) \in \R \st x \in [-1, 0.5] \cup [15,30], y \in [0,1]}
\end{align*}


% Problem 2 ----------------------------------------------
\newpage
\section{}
%Part a
\subsection{Prove that every monotone bounded sequence in $\R$ converge.}
\begin{definition}
    The function $f : \R \to \R$ is called \emph{monotone} if and only if it is either entirely non-increasing or non-decreasing.
    \begin{enumerate}
        \item $f$ is \emph{non-decreasing} if $\forall_{x,y \in \R} x \leq y \implies f(x) \leq f(y)$.
        \item $f$ is \emph{non-increasing} if $\forall_{x,y \in \R} x \leq y \implies f(x) \geq f(y)$.
    \end{enumerate}
\end{definition}
\begin{definition}
    The function $f : \R \to \R$ is called \emph{bounded} if 
    \[\exists_{N\in \R} : \forall_{x\in\R} \ \abs{f(x)} \leq N\]
\end{definition}
\begin{definition}
    A sequence $\{a_n\}$ is said to \emph{converge} to limit $a$ if 
    \[
        \forall_{\epsilon>0} \exists_{N(\epsilon)} : \forall_{n>N} \abs{a_n - a} < \epsilon 
    \]
\end{definition}
\begin{theorem}
    Every monotone bounded sequence in $\R$ converges.
    \begin{proof}
        Define sequence $\{a_n\}$ bounded and monotone. (i.e)
        \[\qty(\qty(\forall_{n,m\in \N : m > n} a_n \geq a_m) 
            \lor \qty(\forall_{n,m\in \N : m > n} a_n \leq a_m))
            \land \qty(\exists_{N \in \R} \forall_{n\in\N} \abs{a_n} \leq N)
        \]
        Taking only the positive case (non-decreasing and bounded from above), a proof can be made without loss of generality (as thats how $\abs{\cdot}$ works...).
        \begin{align*}
            \qty(\forall_{n,m\in \N : n>m} a_n \leq a_m) 
                \land \qty(\exists_{a \in \R} \forall_{n\in\N} \abs{a_n} \leq a)
                &\implies \forall_{\epsilon>0} \exists_{N(\epsilon)} : \forall_{n>N} \abs{a_n - a} < \epsilon\\
            \exists_{a \in \R} \forall_{n,m \in \N, n > m} a_n \leq a_m \leq a  
                &\implies \forall_{\epsilon>0} \exists_{N(\epsilon)} \forall_{n>N} a - a_n < \epsilon\\
            \forall_{\epsilon>0} \exists_{N(\epsilon)} \exists_{a \in \R} \forall_{n,m \in \N, n > m > N} a_n \leq a_m \leq a  
                &\implies  a - a_n < \epsilon
        \end{align*}
        Clearly since the sequence is non-decreasing and bounded from above it will converge to the upper boundary.\\
        This can then be applied again for below to fully prove this.
    \end{proof}
\end{theorem}

%Part b
\subsection{Provide an example of the set $A \in \R$ having exactly four limit points.}
\begin{definition}
    For a set $A \subset S$ in metric space $(S,d)$, $x \in S$ is called a \emph{limit point} of $A$ if 
    \[\forall_{\epsilon>0}\exists_{y\in A} : 0 < d(x,y) < \epsilon\]
\end{definition}

Define $A$ as a simple union of simple single limit point sets:
\[A := \{\frac{1}{n}, n \in \N\} 
    \cup \{\sqrt{2} + \frac{1}{n}, n \in \N\}
    \cup \{\pi + \frac{1}{n}, n \in \N\}
    \cup \{e + \frac{1}{n}, n \in \N\}
\]
In this case the limit points are: $0$, $\sqrt{2}$, $\pi$, and $e$.


%Part c
\subsection{Provide an example of a sequence $\{a_n\}$, so that every point in the interval $[2019,2021]$ is a limit point of it.}

Because sinusoidal functions are fun, lets use:
\[\{a_n\} := \{a_n = 2000 + \sin(n), \ n = 1, 2, \dots\}\]


% Problem 3 ----------------------------------------------
\newpage
\section{}
%Part a
\subsection{}
Provide an example of a sequence $\{a_n\}$ so that $a_n$ diverges, but $\lim_{n\to\infty} (a_n - a_{2n}) = 0$.

This is perhaps going too fun with it, but how about:
\[
    \{a_n\} := \{a_n = 
    \begin{cases}
        a_{n/2} + \frac{1}{n} &n \divisible 2\\
        n^2 &\text{else}
    \end{cases}\}
\]

%Part b
\subsection{}
Provide an example of two sequences $\{a_n\}$ and $\{b_n\}$ so that
    
\[
    \qty(\liminf_{n\to\infty} {a_n} + \liminf_{n\to\infty} {b_n}) 
    < \liminf_{n\to\infty} (a_n + b_n)
    < \qty(\liminf_{n\to\infty} {a_n} + \limsup_{n\to\infty} {b_n})
    < \limsup_{n\to\infty} (a_n + b_n)
    < \qty(\limsup_{n\to\infty} {a_n} + \limsup_{n\to\infty} {b_n})
\]

\begin{definition}
    A \emph{Limit Superior}, denoted as $\limsup_{n\to\infty}$, is the limit of an superior function on the extremes of a sequence. This is defined by 
    \[\limsup_{n\to\infty} x_n := \lim_{n\to\infty} \qty(\sup_{m\geq n}x_m)\]
\end{definition}
\begin{definition}
A \emph{Limit Inferior}, denoted as $\liminf_{n\to\infty}$, is the limit of an inferior function on the extremes of a sequence. This is defined by 
    \[\liminf_{n\to\infty} x_n := \lim_{n\to\infty} \qty(\inf_{m\geq n}x_m)\]
\end{definition}

Defining two sinusoidal functions with the same frequency but with a phase shift to incur the strictness requirements.
\begin{align*}
    \{a_n\} := \{a_n = \cos(\omega_1 n + \theta_1)\}_{n \in \N}\\
    \{b_n\} := \{b_n = \cos(\omega_2 n + \theta_2)\}_{n \in \N}
\end{align*}
with $\omega_1 = \omega_2 = \frac{\pi}{25}$, $\theta_1 = 0$, and $\theta_2 = \frac{\pi}{8}$.




% Problem 4 ----------------------------------------------
\newpage
\section{}
Show the equivalence of the norms $\norm{\cdot}_1,\norm{\cdot}_2,\norm{\cdot}_p, p>1$ and $\norm{\cdot}_\infty$ on $\R^n$.
\begin{definition}
    A \emph{norm} is a function $\norm{\cdot} : S \to \R_+$ satisfying:
    \begin{enumerate}
        \item $\norm{\va{x}} > 0, \ \va{x} \neq 0$
        \item $\norm{\lambda \va{x}} = \abs{\lambda} \norm{\va{x}}$
        \item $\norm{\va{x} + \va{y}} \leq \norm{\va{x}} + \norm{\va{y}}$
    \end{enumerate}
\end{definition}
\begin{definition}
    For $\norm{\cdot}_a, \norm{\cdot}_b$ on $S$, 
    $\norm{\cdot}_a$ is said to be \emph{stronger} then $\norm{\cdot}_b$ if 
    \[\forall \{x_n\} \subset S \st x_n \xrightarrow[d_a]{} x \implies x_n \xrightarrow[d_b]{} x\]
    Which is equivalent to 
\end{definition}
\begin{definition}
    $\norm{\cdot}_a$ and $\norm{\cdot}_b$ are said to be \emph{equivalent},  $\norm{\cdot}_a \sim \norm{\cdot}_b$,
    if $\norm{\cdot}_a$ is stronger then $\norm{\cdot}_b$ 
    and $\norm{\cdot}_b$ is stronger then $\norm{\cdot}_a$. 
    This means that
    \[
        \norm{\cdot}_a \sim \norm{\cdot}_b 
            \iff \exists{\alpha,\beta \in \R_{>0}} : 
            \forall_{x\in S} \alpha \norm{\cdot}_b \leq \norm{\cdot}_a \leq \beta \norm{x}_b
    \]
\end{definition}

\begin{definition} The standard norms are defined as
    \begin{enumerate}
        \item $\norm{\cdot}_1 := \norm{x}_1 = \sum_{i=1}^n \abs{x_i} = \abs{x_1} + \abs{x_2} + \dots + \abs{x_n}$
        \item $\norm{\cdot}_2 := \norm{x}_2 = \qty(\sum_{i=1}^n \abs{x_i}^2)^{1/2} = \qty(\abs{x_1}^2 + \abs{x_2}^2 + \dots + \abs{x_n}^2)^{1/2}$
        \item $\norm{\cdot}_p := \norm{x}_p = \qty(\sum_{i=1}^n \abs{x_i}^p)^{1/p} =\qty(\abs{x_1}^p + \abs{x_2}^p + \dots + \abs{x_n}^p)^{1/p}, \ p > 1$
        \item $\norm{\cdot}_\infty := \norm{x}_\infty = \max_{i=1}^n \abs{x_i} = \max(\abs{x_1}, \abs{x_2}, \dots, \abs{x_n})$
    \end{enumerate}
\end{definition}

\newpage
\begin{theorem}
    The norms $\norm{\cdot}_1,\norm{\cdot}_2,\norm{\cdot}_p,$ and $\norm{\cdot}_\infty$ are all equivalent on $\R^n$.
    \begin{proof}
        % 1-norm ~ 2-norm
        \begin{lemma}\label{lem:1to2}
            $\norm{\cdot}_1 \sim \norm{\cdot}_2$
            \begin{proof}
                $\norm{\cdot}_1 \sim \norm{\cdot}_2$ is true iff
                \begin{multline*}
                    \forall_{x \in \R^n} \exists_{\alpha,\beta \in \R_+} :\\
                    \alpha \norm{x}_2 
                        \leq \norm{x}_1 
                        \leq \norm{x}_2\\
                    \alpha \qty(\sum_{i=1}^n \abs{x_i}^2)^{1/2}
                        \leq \sum_{i=1}^n \abs{x_i}
                        \leq \beta \qty(\sum_{i=1}^n \abs{x_i}^2)^{1/2}\\
                \end{multline*}
                From the Holder's inequality we have 
                \begin{align*}
                    \norm{x}_1 = \sum_{i=1}^n \abs{x_i}
                        &= \sum_{i=1}^n \abs{x_i} (1)\\
                        &\leq \qty(\sum_{i=1}^n \abs{x_i}^2)^{1/2}
                            \qty(\sum_{i=1}^n \abs{1}^{2})^{1/2}\\
                        &\leq n^{1/2} \qty(\sum_{i=1}^n \abs{x_i}^2)^{1/2} 
                \end{align*}
                So for $0 < \alpha \leq n^{1/2}$ and $\beta \geq n^{1/2}$,
                \begin{align*}
                    &\alpha \qty(\sum_{i=1}^n \abs{x_i}^2)^{1/2}
                        &&\leq \sum_{i=1}^n \abs{x_i}
                        &&\leq \beta \qty(\sum_{i=1}^n \abs{x_i}^2)^{1/2}\\
                    &\qty(\sum_{i=1}^n \abs{x_i}^2)^{1/2}
                        &&\leq n^{1/2} \qty(\sum_{i=1}^n \abs{x_i}^2)^{1/2} 
                        &&\leq n^{1/2} \qty(\sum_{i=1}^n \abs{x_i}^2)^{1/2}
                \end{align*}
                Therefore,
                \[\norm{x}_2 \leq \norm{x}_1 \leq n^{\frac{1}{1-p}} \norm{x}_2\]
                which proves $\norm{\cdot}_1 \sim \norm{\cdot}_2$.
            \end{proof}
        \end{lemma}
        \newpage
        % 1-norm ~ p-norm
        \begin{lemma}\label{lem:1toP}
            $\norm{\cdot}_1 \sim \norm{\cdot}_p$
            \begin{proof}
                $\norm{\cdot}_1 \sim \norm{\cdot}_p$ is true iff
                \begin{multline*}
                    \forall_{x \in \R^n} \exists_{\alpha,\beta \in \R_+} :\\
                    \alpha \norm{x}_p 
                        \leq \norm{x}_1 
                        \leq \norm{x}_p\\
                    \alpha \qty(\sum_{i=1}^n \abs{x_i}^p)^{1/p}
                        \leq \sum_{i=1}^n \abs{x_i}
                        \leq \beta \qty(\sum_{i=1}^n \abs{x_i}^p)^{1/p}\\
                \end{multline*}
                From the Holder's inequality we have 
                \begin{align*}
                    \norm{x}_1 = \sum_{i=1}^n \abs{x_i}
                        &= \sum_{i=1}^n \abs{x_i} (1)\\
                        &\leq \qty(\sum_{i=1}^n \abs{x_i}^p)^{1/p}
                            \qty(\sum_{i=1}^n \abs{1}^{(1-p)})^{1/(1-p)}\\
                        &\leq n^{1/(1-p)} \qty(\sum_{i=1}^n \abs{x_i}^p)^{1/p} 
                \end{align*}
                So for $0 < \alpha \leq n^{1/(1-p)}$ and $\beta \geq n^{1/(1-p)}$,
                \begin{align*}
                    &\alpha \qty(\sum_{i=1}^n \abs{x_i}^p)^{1/p}
                        &&\leq \sum_{i=1}^n \abs{x_i}
                        &&\leq \beta \qty(\sum_{i=1}^n \abs{x_i}^p)^{1/p}\\
                    &\qty(\sum_{i=1}^n \abs{x_i}^p)^{1/p}
                        &&\leq n^{1/(1-p)} \qty(\sum_{i=1}^n \abs{x_i}^p)^{1/p} 
                        &&\leq n^{1/(1-p)} \qty(\sum_{i=1}^n \abs{x_i}^p)^{1/p}
                \end{align*}
                Therefore,
                \[\norm{x}_p \leq \norm{x}_1 \leq n^{\frac{1}{1-p}} \norm{x}_p\]
                which proves $\norm{\cdot}_1 \sim \norm{\cdot}_p$.
            \end{proof}
        \end{lemma}
        \newpage
        %1-norm ~ \infty-norm
        \begin{lemma}\label{lem:1toInfty}
            $\norm{\cdot}_1 \sim \norm{\cdot}_\infty$
            \begin{proof}
                $\norm{\cdot}_1 \sim \norm{\cdot}_\infty$ is true iff
                \begin{multline*} 
                    \forall_{x \in \R^n} \exists_{\alpha,\beta \in \R_+} :\\
                    \alpha \norm{x}_\infty \leq \norm{x}_1 \leq \beta \norm{x}_\infty\\
                    \alpha \max_{i=1}^n \abs{x_i} \leq \sum_{i=1}^n \abs{x_i} \leq \beta \max_{i=1}^n \abs{x_i}\\
                \end{multline*}
                Clearly, this is true for when $\alpha \in (0,1]$. Similarily, when $\beta \geq n$ then $\sum_{i=1}^n \max_{i=1}^n \abs{x_i}$ and then clearly greater then the $\norm{x}_1$; therefore $\norm{\cdot}_1 \sim \norm{\cdot}_\infty$.
            \end{proof}
        \end{lemma}
        From, Lemma \ref{lem:1to2}, Lemma \ref{lem:1toP}, and Lemma \ref{lem:1toInfty}, it is clear that $\forall_{p > 1}$ (which implies $\norm{x}_2$ as well):
        \[
            \norm{x}_\infty 
            \leq \norm{x}_p 
            \leq \norm{x}_1 
            \leq n^{1/{1-p}} \norm{x}_p 
            \leq n \norm{x}_\infty
        \]
        Therefore, $\norm{\cdot}_1 \sim \norm{\cdot}_2 \sim \norm{\cdot}_p \sim \norm{\cdot}_\infty$ on $\R^n$ ($\forall_{p > 1}$).
    \end{proof}
\end{theorem}




% Need to do this again... the way I first did it was pretty terrible math...











% Problem 5 ----------------------------------------------
\newpage
\section{}
Are there any open sets $A$ and $B$ in $\R^4$ so that $dist(A,B) = 0$ but $A \cap B = \emptyset$?

\begin{definition}
    For sets $A,B \subset S$ in metric space $(S,d)$, 
    \[dist(A,B) := \inf\{d(x,y) \st x \in A, y \in B\}\]
\end{definition}

\begin{proposition}
    There exists open sets $A$ and $B$ in $\R^2$ so that $d(A,B) = 0$ but $A \cap B = \emptyset$.
    \begin{proof}
        $A,B \in \R^4$ open means 
        \[
            \forall_{x \in A} \exists_{\epsilon>0} \forall_{y \in \R^4} d(x,y) < \epsilon \implies y \in A
        \]
        and 
        \[
            \forall_{x \in B} \exists_{\epsilon>0} \forall_{y \in \R^4} d(x,y) < \epsilon \implies y \in B
        \]
        $dist(A,B) = 0$ means that 
        \[dist(A,B) = \inf\{d(x,y) \st x \in A, y \in B\} = 0\]
        Additionally, $A \cap B = \emptyset$ implies that $\forall_{x \in A} \forall_{y \in B} x \neq y$.

        In order for $dist(A,B) = 0$ and $A \cap B = \emptyset$, 
        \[\exists_{x \in A} : \inf_{y \in B} \{d(x,y)\} = 0 \land x \notin B\]
        One solution to this is that
        \[\exists_{x \in \partial{A}, y \in \partial{B}} \st d(x,y) = 0\]
        Therefore, it is possible for open sets $A$ and $B$ in $\R^4$ so that $dist(A,B) = 0$ and $A \cap B = \emptyset$.

        Essentially, $\partial{A} \cap \partial{B} \neq \emptyset$.

        An example would be: 
        \[A := \{(x,y) \in \R^2 : x^2 + y^2 < r_1^2\}\]
        and 
        \[B := \{(x,y) \in \R^2: (x + 2r_2)^2 + y^2 < r_2^2\}\]
        with $r_1 = r_2 = r \in \R$.
    \end{proof}
\end{proposition}

% Problem 6 ----------------------------------------------
\newpage
\section{}
Let $\mathcal{B}([0,1])$ denote the set of all bounded functions from $[0,1]$ to $\R$. Define the metric on $\mathcal{B}([0,1])$ as $d(f,g) = \sup_{x \in [0,1]} \abs{f(x) - g(x)}$.

\begin{definition}
    For  ordered field $S$ and $A$ bounded from above, $c \in S$ is a \emph{supremum} of $A$, 
    $c = \sup A$, if: 
    \[
        c = \sup A \iff 
            \qty(\forall_{a\in A} a \leq c) 
            \land \qty(\forall_{\epsilon>0} \exists_{a\in A} \st c - \epsilon < a)
    \]
\end{definition}

%Part a
\subsection{Show that this is indeed a metric.}

\begin{theorem}
    The metric $d : \mathcal{B}([0,1]) \to \R$
    \[d(f,g) := \sup_{x \in [0,1]} \abs{f(x) - g(x)}\]
    is a metric on $\mathcal{B}([0,1])$.
\end{theorem}
\begin{proof}
    A metric must satisfy (i) non-negativity, 
    (ii) Symmetry, and (iii) Triangle Inequality.
    \subsubsection{Non-negativy \[d(f,g) = \sup_{x \in [0,1]} \abs{f(x) - g(x)} \geq 0\]}
        Since $\abs{f(x) - g(x)} \geq 0$, $\sup_{x \in [0,1]} \abs{f(x) - g(x)} \geq 0$ and therefore $d(f,g) \geq 0$.
    \subsubsection{Symmetry \[d(f,g) = d(g,f)\]}
    \begin{align*}
        d(f,g) = \sup_{x \in [0,1]} \abs{f(x) - g(x)} &= d(g,f) = \sup_{x \in [0,1]} \abs{g(x) - f(x)}\\
        \sup_{x \in [0,1]} \abs{f(x) - g(x)} &= \sup_{x \in [0,1]} \abs{(-1)(f(x) -g(x))}\\
        \sup_{x \in [0,1]} \abs{f(x) - g(x)} &= \sup_{x \in [0,1]} \abs{(f(x) -g(x))}
    \end{align*}
    Therefore it is symmetric.
    \subsubsection{Triangle Inequality \[d(f,h) \leq d(f,g) + d(g,h)\]}
    \begin{align*}
        d(f,h) &\leq d(f,g) + d(g,h)\\
        \sup_{x \in [0,1]} \abs{f(x) - h(x)} 
            &\leq  = \sup_{x \in [0,1]} \abs{f(x) - g(x)} + \sup_{x \in [0,1]} \abs{g(x) - h(x)}\\
        \forall_{x \in [0,1]} \qty(\abs{f(x) - h(x)} < d(f,h))
            &\leq \forall_{x \in [0,1]} \qty(\abs{f(x) - g(x)} < d(f,g)) + \forall_{x \in [0,1]} \qty(\abs{g(x) - h(x)} < d(g,h))\\
        \forall_{x_1,x_2,x_3 \in [0,1]} (\abs{f(x_1) - h(x_1)}
            & \leq \abs{f(x_2) - g(x_2)} + \abs{g(x_3) - h(x_3)})
    \end{align*}
    Suppose this is false
    \begin{align*}
        \exists_{x_1,x_2,x_3 \in [0,1]} (\abs{f(x_1) - h(x_1)}
        & > \abs{f(x_2) - g(x_2)} + \abs{g(x_3) - h(x_3)})\\
        \exists_{x_1,x_2,x_3 \in [0,1]} \st \Big(\abs{f(x_1) - h(x_1)} 
        > \abs{f(x_1)} + \abs{h(x_1)}
        &> \abs{f(x_2) - h(x_2)} + \abs{h(x_3) - g(x_3)}\\
        &> \abs{f(x_2)} - \abs{g(x_2)} + \abs{g(x_3)} - \abs{g(x_3)})\Big)\\
    \end{align*}
    Which due to the triangular inequality property of $\abs{\cdot}$, is clearly not possible and therefore the metric satisfies the Triangular inequality.
\end{proof}

\newpage
%Part b
\subsection{Prove that the space $(\mathcal{B}([0,1]),d)$ is a complete metric space.}
\begin{definition}
    Metric space $(S,d)$ is called a \emph{complete metric space} if every cauchy sequence $\{a_n\}\subset S$ converges in $S$.
    \[\forall_{\{a_n\}\subset S \st \{a_n\} \ \textnormal{cauchy} \ \implies \exists_{a \in S} \st \lim_{n\to\infty} a_n = a}\]
\end{definition}
\begin{theorem}
    $(\mathcal{B}([0,1]),d)$ is a complete metric space.
    \begin{proof}
        Let $\{a_n\}$ be a cauchy sequence (i.e.)
        \[\{a_n\} : \forall_{\epsilon>0} \exists_{N} : \forall_{n,m > N} \implies d(a_n,a_m) < \epsilon\]
        Consider set $D_N = \{x \in S \st \forall_{n < N} a_n < x\}$
        \begin{enumerate}
            \item $D$ is bounded (i.e $x \in D < a_N + \epsilon$)
            \item $D$ is nonempty (i.e $a_N - \epsilon \in D$)
        \end{enumerate}
        For a fixed $\epsilon$, 
        \[\exists_{N} \st \forall_{n > N + 1} \implies d(a_n, a_{N+1}) < \epsilon\]
        We then take $a = \sup\{D\}$.

        Claim $a = \lim_{n\to\infty} a_n$,
        \[\forall_{\epsilon>0} \exists_N \st \forall_{n>N} \implies d(a,a_n) < \epsilon\]
        Claim $y < a \implies y \in D$,
        \[y \in D \land z < y \implies z \in D\]
        \[\forall_{\epsilon>0} \exists_{N(\epsilon)} \st \forall_{n>N} a_n > a - \frac{\epsilon}{2}\]
        However, it is also true by definition of cauchy sequence: 
        \[d(a_m, a_n) < \frac{\epsilon}{2} \ \forall_{m,n > \hat{N}}\]
        Also,
        \[n > \max[N,\hat{N}] \st d(a_n,a) \leq d(a_n,a_m) + d(a_m,a) \leq \epsilon\]
        Since $\epsilon>0$ is arbitrarily selected and $d(a_n,a) < \epsilon$ whenever $n\geq N$, this implies $\{a_n\}$ converges to $a$.
    \end{proof}
\end{theorem}

\newpage
%Part c
\subsection{Is the unit ball $\mathcal{B}_1(0) = \{f(x) \st d(f,0) \leq 1\}$ compact?}
\begin{definition}
    Let $(S,d)$ be a metric space with $A \subset S$,
    \begin{enumerate}
        \item For $\{U_\alpha\}_{\alpha \in A}, \ U_\alpha \subset S$, is a \textbf{\underline{cover}} of the set $A$ if 
        $$A \subset \bigcup_{\alpha\in A} U_\alpha$$
        \item A cover $\{U_\alpha\}_{\alpha \in A}$ of $A$ is an \textbf{\underline{open cover}} if $\forall_{\alpha \in A}$ $U_\alpha$ is an open set.
        \item $\{V_\beta\}_{\beta\in B}$ is called a \textbf{\underline{subcover}} of $\{U_\alpha\}_{\alpha \in A}$ if
        \begin{enumerate}
            \item $\{V_\beta\}_{\beta\in B}$ is a cover of $A$
            \item $\forall_{\beta \in B} \exists_{\alpha \in A} V_\beta = U_\alpha$
        \end{enumerate}
        \item A cover with a finite number of sets is called a \textbf{\underline{finite cover}}.
    \end{enumerate}
\end{definition}
\begin{definition}
    For $A \subset (S,d)$, $A$ is \textbf{\underline{compact}} if for every open cover of $A$ there exists a finite sub cover.
\end{definition}

\begin{proposition}
    $\mathcal{B}_1(0) = \{f(x) \st d(f,0) \leq 1\}$ is compact.
    \begin{proof}
        \begin{align*}
            \mathcal{B}_1(0) &= \qty{f(x) \st d(f,0) \leq 1}\\
            &= \qty{f(x) \st \sup_{x \in [0,1]} \abs{f(x) - 0} \leq 1}\\
            &= \qty{f(x) \st \sup_{x \in [0,1]} \abs{f(x)} \leq 1}\\
            &= \qty{f(x) \st \forall_{x \in [0,1]} \abs{f(x)} \leq 1}\\
            &= \qty{f(x) \st \forall_{x \in [0,1]} -1 \leq f(x) \leq 1}\\
        \end{align*}
        In other words, the ball is totally bounded. 
        Since it is also known that this bounded subset of $\mathcal{B}([0,1])$ is complete, it is therefore also compact.
    \end{proof}
\end{proposition}

\end{document}
