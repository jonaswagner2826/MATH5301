% Standard Article Definition
\documentclass[]{article}

% Page Formatting
\usepackage[margin=1in]{geometry}
\setlength\parindent{0pt}

% Graphics
\usepackage{graphicx}

% Math Packages
\usepackage{physics}
\usepackage{amsmath, amsfonts, amssymb, amsthm}
\usepackage{mathtools}

% Extra Packages
\usepackage{listings}
\usepackage{hyperref}

% Section Heading Settings
\usepackage{enumitem}
\renewcommand{\theenumi}{\alph{enumi}}
\renewcommand*{\thesection}{Problem \arabic{section}}
\renewcommand*{\thesubsection}{\alph{subsection})}
\renewcommand*{\thesubsubsection}{\quad \quad \roman{subsubsection})}

%Custom Commands
\newcommand{\Rel}{\mathcal{R}}
\newcommand{\R}{\mathbb{R}}
\newcommand{\C}{\mathbb{C}}
\newcommand{\N}{\mathbb{N}}
\newcommand{\Z}{\mathbb{Z}}
\newcommand{\Q}{\mathbb{Q}}

\newcommand{\toI}{\xrightarrow{\textsf{\tiny I}}}
\newcommand{\toS}{\xrightarrow{\textsf{\tiny S}}}
\newcommand{\toB}{\xrightarrow{\textsf{\tiny B}}}

\newcommand{\divisible}{ \ \vdots \ }
\newcommand{\st}{\ : \ }


% Theorem Definition
\newtheorem{definition}{Definition}
\newtheorem{assumption}{Assumption}
\newtheorem{theorem}{Theorem}
\newtheorem{lemma}{Lemma}
\newtheorem{proposition}{Proposition}
\newtheorem{example}{Example}


%opening
\title{MATH 5301 Elementary Analysis - Final Exam}
\author{Jonas Wagner}
\date{2021, December 7\textsuperscript{th}}

\begin{document}

\maketitle

% Problem 1 ----------------------------------------------
\section{}
For each $n \in \N$ define the set
\[
    Q_n := \qty{
        \frac{1}{pq} \st 0 < p < q \leq n; \ p + q > n; \ \textnormal{gcd}(p,q) = 1
    }
\]
Let $f(n)$ be the sum of all elements of $Q_n$.

Find $\inf_n f(n)$.

\begin{definition}\label{def:pblm1_Qn}
    Let the set $Q_n$ be defined for all $n \in \N$ as
    \[
        Q_n := \qty{
            \frac{1}{pq} \st 0 < p < q \leq n; \ p + q > n; \ \gcd(p,q) = 1
        }
    \]
\end{definition}

\begin{definition}\label{def:pblm1_fn}
    Let $f(n)$ be the sum of all elements within $Q_n$.
\end{definition}

\begin{definition}\label{def:infimum}
    A lower bound of subset $A$ in the partially ordered set $(S,\leq)$ is defined by
    \[
        a \in S \st a \leq x \forall_{x \in A}
    \]
    A lower bound of $a$ is called an \emph{\underline{infimum}} of set $A \in (S,\leq)$,
    denoted as $a = \inf A$, is the greatest lower bound. i.e.
    \[
        \forall_{y \in S : a \leq x \forall_{x \in A}} y \leq a
    \]
\end{definition}

\begin{definition}
    The \underline{\emph{Greatest Common Divisor}} of two nonzero integers $a,b \in \Z \neq 0$, $\gcd(a,b)$, 
    is defined as the largest positive integer, $d \in \Z_+$, so that $d$ is a divisor of both $a$ and $b$.
    i.e:
    \[
        \gcd(a,b) := d \in \Z_+ \st (a \divisible d) \land (b \divisible d) 
                    \land (\forall_{x \in \Z_+ \st a,b \divisible x} d \geq x)
    \]
    Additionally, $a$ and $b$ are considered \emph{\underline{coprime}} if $\gcd(a,b) = 1$.
\end{definition}

\begin{assumption}
    For this problem it is assumed that $\gcd$ is only defined within $\Z_+$, 
    although I believe this can also be expanded to other less-strict ordered sets in the same way.
\end{assumption}

\begin{assumption}
    It is assumed that the sum of all elements in the empty set is 0, i.e. $\sum_{i} \emptyset = 0$.
\end{assumption}

\newpage
\begin{theorem}
    \[
        \inf_{n \in \N} f(n) = 0
    \]

    \begin{proof}
        Proof by induction.

        For $n = 1$, 
            $\lnot \exists_{p,q \in \Z \st 0<p<q\leq 1}$ meaning that $Q_1 = \emptyset$.
            
            This implies that $f(1) = \sum_{i} \emptyset = 0$ and that $f(1) \geq 0$.

        For $n = 2$, 
        \[
            (p,q) \in \qty{(p,q) \st 0 < p < q \leq 2; \ p + q > n; \ \gcd(p,q) = 1} = \qty{(1,2)}
        \]    
        The set $Q_2$ is then defined as
        \[
            Q_2 = \qty{\frac{1}{pq} \st (p,q) \in \qty{(1,2)}}
                = \qty{\frac{1}{(1)(2)}}
                = \qty{\frac{1}{2}}
        \]
        Therefore,
        \[
            f(2) = \sum_{i} \qty{\frac{1}{2}} = \frac{1}{2}
        \]
        It is clear that $f(2) = \frac{1}{2} \geq 0$.

        For $n = 3$,
        \[
            (p,q) \in \qty{(p,q) \st 0 < p < q \leq 3; \ p + q > n; \ \gcd(p,q) = 1} 
                = \qty{(1,3),(2,3)}
        \]
        The set $Q_3$ is then defined as
        \[
            Q_3 = \qty{\frac{1}{pq} \st (p,q) \in \qty{(1,3),(2,3)}}
                = \qty{\frac{1}{(1)(3)}, \frac{1}{(2)(3)}}
                = \qty{\frac{1}{3}, \frac{1}{6}}
        \]
        Therefore,
        \[
            f(3) = \sum_{i} \qty{\frac{1}{3}, \frac{1}{6}} 
                = \frac{1}{3} + \frac{1}{6} 
                = \frac{2 + 1}{6}
                = \frac{3}{6}
                = \frac{1}{2}
        \]
        It is clear that $f(3) = \frac{1}{2} \geq 0$.

        For $n = 4$,
        \[
            (p,q) \in \qty{(p,q) \st 0 < p < q \leq 4; \ p + q > n; \ \gcd(p,q) = 1} 
                = \qty{(1,4),(2,3),(3,4)}
        \]
        The set $Q_4$ is then defined as
        \[
            Q_4 = \qty{\frac{1}{pq} \st (p,q) \in \qty{(2,3),(3,4)}}
                = \qty{\frac{1}{(1)(4)}, \frac{1}{(2)(3)}, \frac{1}{(3)(4)}}
                = \qty{\frac{1}{4},\frac{1}{6}, \frac{1}{12}}
        \]
        Therefore,
        \[
            f(4) = \sum_{i} \qty{\frac{1}{6}, \frac{1}{12}}
                = \frac{1}{4} + \frac{1}{6} + \frac{1}{12}
                = \frac{3 + 2 + 1}{12}
                = \frac{6}{12}
                = \frac{1}{2}
        \]
        It is clear that $f(4) = \frac{1}{2} \geq 0$.

        % For $n = 5$,
        % \[
        %     (p,q) \in \qty{(p,q) \st 0 < p < q \leq 5; \ p + q > n; \ \gcd(p,q) = 1} 
        %         = \qty{(1,5),(2,5),(3,4),(3,5),(4,5)}
        % \]
        % The set $Q_5$ is then defined as
        % \[
        %     Q_5 = \qty{\frac{1}{pq} \st (p,q) \in \qty{(1,5),(2,5),(3,4),(3,5),(4,5)}}
        %         = \qty{\frac{1}{(1)(5)}, \frac{1}{(2)(5)}, \frac{1}{(3)(4)}, \frac{1}{(3)(5)}, \frac{1}{(4)(5)}}
        %         = \qty{\frac{1}{5}, \frac{1}{10}, \frac{1}{12}, \frac{1}{15}, \frac{1}{20}}
        % \]
        % Therefore,
        % \[
        %     f(5) = \sum_{i} \qty{\frac{1}{5}, \frac{1}{10}, 
        %                         \frac{1}{12}, \frac{1}{15}, \frac{1}{20}}
        %         = \frac{1}{5} + \frac{1}{10} + \frac{1}{12} + \frac{1}{15} + \frac{1}{20}
        %         = \frac{1}{2}
        % \]
        % It is clear that $f(5) = \frac{1}{2} \geq 0$


        For an arbitrary $n \in \N$,
        \begin{align*}
            (p,q) &\in \qty{(p,q) \st 0 < p < q \leq n; \ p + q > n; \ \gcd(p,q) = 1} =\\
                &= \qty{(1,n), (2,n - \star), (3, n - \star) \dots, (n-2, n-1), (n-1, n)}
        \end{align*}
        \begin{align*}
            Q_n &= \qty{\frac{1}{pq} \st (p,q) \in \qty{(1,n), (2,n-\star), \dots, (n-2, n-1), (n-1, n)}}\\
                &= \qty{\frac{1}{(1)(n)}, \frac{1}{(2)(n-1)}, \dots, \frac{1}{(n-2)(n-1)}, \frac{1}{(n-1)(n)}}\\
                &= \qty{\frac{1}{n}, \frac{1}{2(n-\star)}, \dots, \frac{1}{(n-2)(n-1)}, \frac{1}{n(n-1)}}
        \end{align*}

        where $\star$ is dependent for on divisibility properties between $n$ and 2, 3, 4, etc.
        It is important to note that each increase of $n$ will cause every term to decrease in magnitude individually but additional elements are added that result to adding up to $\frac{1}{2}$ again.

        However, eventually this will reach a point where a lack of prime numbers in a region makes it so that the only coprime numbers satisfying the conditions are adjacent to one another, which leads to the following:
        \begin{align*}
            f(n)    &= \sum_{i} Q_n = \frac{1}{n} + \dots + \frac{1}{(\frac{n}{2}) (\frac{n}{2}+1)} + \dots + \frac{1}{n (n-1)}\\
            f(n+1)  &= \qty(\sum_{i} Q_n) \qty(\frac{n!}{(n+1)!}) + \frac{1}{(n+1)}\\
                    &= \frac{1}{n} \frac{n!}{(n+1)!} + \dots + \frac{1}{(\frac{n}{2}) (\frac{n}{2}+1)} \frac{n!}{(n+1)!} + \dots + \frac{1}{n (n-1)} \frac{n!}{(n+1)!} + \frac{1}{n+1}\\
                    &= \frac{n!}{n(n+1)n!} + \dots + \frac{n!}{\frac{n}{2}(\frac{n}{2}-1)(n+1)n!} + \dots + \frac{n!}{n (n-1) (n+1) n!} + \frac{1}{n+1}\\
                    &= \sum_{i} Q_{n+1} = \frac{1}{n+1} + \dots + \frac{1}{(\frac{n+1}{2}) (\frac{n+1}{2}+1)} + \dots + \frac{1}{n (n+1)}
        \end{align*}
        essentially every $(p,q)$ becomes $(q,q+1)$ and the new $\frac{1}{(n+1)}$ is added.


        Anyway, the point is that $\forall_{n\in\N \st n>1} f(n) \geq \frac{1}{2}$; 
        however, because $f(n)$ is included, $\frac{1}{2} \leq f(n) \forall_{n \in N}$ since $Q_1 = \emptyset \implies f(1) = 0$.

        Therefore,
        \[
            \inf_n f(n) = 0
        \]

        % Therefore,
        % \begin{align*}
        %     f_{even}(n) &= \sum_{i} \qty{\frac{1}{n}, \frac{1}{2(n-1)}, \dots, \frac{1}{(n-2)(n-1)}, \frac{1}{n(n-1)}}\\
        %         &= \frac{1}{n} + \frac{1}{2(n-1)} + \dots + \frac{1}{(n-2)(n-1)} \frac{1}{n(n-1)}\\
        %         &= \frac{
        %                 (n-1)(n-2)\cdots(3)(2) + (n)(n-2)\cdots(3)(1) + \dots + (n-2)(n-3)\cdots(2)(1)
        %             }{
        %                 n(n-1)(n-2)\cdots(3)(2)(1)
        %             }\\
        %         &= \frac{
        %                 \sum_{i=1}^n \frac{n!}{(i)(n-i)}
        %             }{
        %                 n!
        %             }
        % \end{align*}

        % Something similar is true for $n \in \N \st n + 1 \divisible 2$,
        % but in reality it isn't important for the proof,
        % \begin{align*}
        %     f_{odd}(n) &= \sum_{i} \qty{\frac{1}{n}, \frac{1}{2(n)}, \dots, \frac{1}{(n-2)(n-1)}, \frac{1}{n(n-1)}}\\
        %         &= \frac{1}{n} + \frac{1}{2(n)} + \dots + \frac{1}{(n-2)(n-1)} \frac{1}{n(n-1)}\\
        %         &= \frac{
        %             \sum_{i=1}^n \frac{n!}{(n)(n-i)}
        %         }{n!}
        % \end{align*}
        
        
        % When it is known that $f(n) = \frac{1}{2}$,
        % \begin{align*}
        %     f(n)_{even} = \frac{1}{2}
        %         &= \frac{1}{n} + \frac{1}{2(n-1)} + \dots + \frac{1}{(n-2)(n-1)} \frac{1}{n(n-1)}
        % \end{align*}
        
        % Then this is also true for $f(n+1)$ as shown 
        % (with not much algebraic detail since the answer itself remains trivial...)
        % \begin{align*}
        %     f_{odd}(n+1) &= \frac{1}{n+1} + \frac{2(n+1)} + \dots + \frac{(n-2)(n-1)} + frac{(n+1)(n + 1 - 1)}\\
        %                 &= 
        % \end{align*}

        


        % Since it is known that the set with the least number of elements is $\emptyset$
        %     and that $f(n) = 0 \forall_{n} \st Q_n = \emptyset$, and that 
    \end{proof}
\end{theorem}




% Problem 2 ----------------------------------------------
\newpage
\section{}
Let $(X,d)$ be a metric space.
Let $B_r(a)$ denote the open ball of radius $r$ centered at $a$.
i.e.
Can it happen that $B_{r_1}(a) \subset B_{r_2}(a)$ but $r_1 > r_2$?

\begin{definition}\label{def:open_ball}
    Within the metric space $(X,d)$, 
    the open ball of radius $r \in X$ centered at $a \in X$, 
    denoted as $B_r(a)$, is defined as:
    \[
        B_r(a) := \qty{x \in X \st d(a,x) < r}
    \]
\end{definition}

\begin{assumption}\label{ass:normed_metric}
    First it will be assumed that $(X,d)$ is a normed vector space.
    This restricts the metric and metric space into a normed space.
    This can also be denoted as $(X,\norm{\cdot})$ to distinguish between them.
    It is also assumed that $X$ is complete.
\end{assumption}

\begin{theorem}
    For  $r_1 > r_2$ then it is not possible for $B_{r_1}(a) \subset B_{r_2}(b)$ within $(X, \norm{\cdot})$:
    \begin{proof}
        Proof by contradiction.
        
        Let \[
            B_{r_1}(a), B_{r_2}(b) \subset X
        \]
        with $0 < r_2 < r_1$
        and $a \in B_{r_2}(b)$.

        To minimize the amount of the set existing outside of the set,
        we need to set $a = b$.
        Next, let $c$ be a point within the punctured open ball $B_{r_2}(b)$.
        i.e.\[
            c \in B_{r_2}(b) \backslash \{b\}
        \]
        $c$ can then be used to construct a point that is contained in $B_{r_2}(b)$ but not in $B_{r_1}(a)$:
        \[
            p + \frac{r_1 + r_2}{2} \frac{a c}{\norm{a c}} \in B_{r_1}(a) \backslash B_{r_2}(b)
        \]
        Meaning that there is no possible way for an open ball of greater radius (within a normed metric space).
    \end{proof}
\end{theorem}

\begin{assumption}
    The previous assumption, Assumption \ref{ass:normed_metric}, is now relax the metric so that $d$ is not restricted to be a norm (i.e. may not be linear).
\end{assumption}

\begin{theorem}
    It is possible for $B_{r_1}(a) \subset B_{r_2}(b)$ within $(X, d)$ when $r_1 > r_2$:
    \begin{proof}
        Proof by example:

        Let metric space $(X, d)$ be defined by
        \[X := {0} \cup [5, \infty)\]
        \[d(x,y) := \abs{x - y}\]

        For $r_1 = 4$, $r_2 = 3$, 

        Let $B_4(0)$ be defined as
        \[
            B_4(0) := \qty{4
                x \in X \st d(0,x) < 4
            } = \{0\} \cup [2,4)
        \]
        
        Let $B_3(2)$ be defined as
        \[
            B_3(2) := \qty{
                x \in X \st d(2,x) < 3
            } = \{0\} \cup [2, 5)
        \]

        Clearly, $B_3(2) \subset B_4(0)$.
        Since $r_1 = 4 > r_2 = 3$, this exists as an example that satisfies the conditions.
    \end{proof}
\end{theorem}


% Problem 3 ----------------------------------------------
\newpage
\section{}
Let $M$ be the set of all bounded sequences
\[
    M = \qty{\{a_j\}_{j=1}^{\infty} \st \abs{a_j} < \infty}
\]
Define $\rho(\{a_n\},\{b_n\}) = \max_{n \in \N} \abs{a_n - b_n}$

\begin{definition}\label{def:metric}
    Function $d : X \cross X \to \R$ is considered a \underline{\emph{metric}} 
    if it satisfies all of the following:
    \begin{enumerate}
        \item Non-negativity:
            \[d(a,b) \geq 0\]
        \item Symmetry:
            \[d(a,b) = d(b,a)\]
        \item Triangle Inequality:
            \[d(a,c) \leq d(a,b) + d(b,c)\]
    \end{enumerate}
\end{definition}

\subsection{Show that $(M,\rho)$ is a metric space.}

\begin{theorem}
    Let $M$ be defined as the set of all bounded sequences:\[
        M = \qty{\{a_j\}_{j=1}^{\infty} \st \abs{a_j} < \infty}
    \]
    Let the metric $\rho$ be defined on $M$ such that\[
        \rho(\{a_n\},\{b_n\}) = \max_{n\in\N} \abs{a_n - b_n}
    \]
    The metric space $(M, \rho)$ is in fact a metric space.
    \begin{proof}
        From Definition \ref{def:metric}, $\rho$ is a metric if 
        $\forall_{\{a_n\},\{b_n\},\{c_n\}} \in M$ 
        these three conditions are all satisfied: 
            (i) non-negativity, 
            (ii) Symmetry, and 
            (iii) Triangle Inequality.
        \begin{enumerate}
            \item Non-negativity:
            \begin{align*}
                d(a,b) &\geq 0\\
                \rho(\{a_n\},\{b_n\}) = \max_{n \in \N} \abs{a_n - b_n} &\geq 0\\
            \end{align*}
            \item Symmetry:
            \begin{align*}
                d(a,b) &= d(b,a)\\
                \rho(\{a_n\},\{b_n\}) = \max_{n \in \N} \abs{a_n - b_n} 
                    &= \max_{n \in \N} \abs{b_n - a_n} = \rho(\{a_n\},\{b_n\})
            \end{align*}
            \item Triangle Inequality:
            \begin{align*}
                d(a,c) &\leq d(a,b) + d(b,c)\\
                \rho(\{a_n\},\{c_n\}) 
                    &\leq \rho(\{a_n\},\{b_n\}) + \rho(\{b_n\},\{c_n\})\\
                \max_{n \in \N} \abs{a_n - c_n} 
                    \leq \max{n \in \N} \abs{a_n - b_n} + \abs{b_n - c_n}    
                    &\leq \max_{n \in \N} \abs{a_n - b_n} 
                        + \max_{n \in \N} \abs{b_n - c_n}
            \end{align*}
        \end{enumerate}
    \end{proof}
\end{theorem}


\newpage
\subsection{Show that $M$ does not contain a dense countable subset.}

% (Hint: recall the very first example of uncountable set...)

\begin{definition}\label{def:closure}
    The \underline{\emph{Closure}}, $\overline{A}$, of $A \subset X$ is defined as
    \[
        \overline{A} = A \cup \qty{
            \lim_{n\to\infty} a_n \st a_n \in A \forall_{n \in \N}
        }
    \]
\end{definition}

\begin{definition}\label{def:dense}
    A set $A \subset X$ is considered \underline{\emph{dense}} in $X$ if $\overline{A} = X$.
\end{definition}

\begin{theorem}\label{thm:cantor}
    For the power set, $\mathcal{P}(A)$, 
    defined as the collections of all sets constructed from the elements of $A$, 
    then the cardinality of $\mathcal{P}(A)$ will always be strictly greater then that of $A$.
    i.e.\[
        \abs{2^{A}} > \abs{A}
    \]
    \begin{itemize}
        \item This is also applicable to infinite sets with whether it is countable or not.
        i.e\[
            \abs{2^\N} = \aleph_1 > \abs{\N} = \aleph_0
        \]
        \item The theorem itself is that any mapping from $A$ to $\mathcal{P}(A)$ is not surjective which is then proven false.
        It then follows that $f : A \toI \mathcal{P}(A)$ is injective, 
        which is equivalent to saying that $\abs{A} < \abs{\mathcal{P}(A)}$.
    \end{itemize}
\end{theorem}

\begin{theorem}
    $M$ does not contain any dense countable subsets.
    \begin{proof}
        Proof by contradiction inspired by Cantor's Theorem (\ref{thm:cantor}).

        Let $A_N \subset M$ be defined as 
        \[
            A_N := \{\{a_j\}_{j=1}^{\infty} \st \abs{a_j} < N\}
        \]

        Similarly to Cantor's theorem, even when restricting $a_j$ from a finely sized set,
        the only mapping that exists from a countable set into $A_n$ are strictly injective. 
        
        Next, taking $A = \lim_{N \to \infty} A_N$, 
        we will prove that in order for $A$ to be dense, $A$ would no longer be countable.

        From Definition \ref{def:closure} and Definition \ref{def:dense}, 
        it is known that in order for $A$ to be dense within $M$, 
        $\overline{A} = M$. 
        Since $M$ itself is an infinite set, 
        even if for sequences of a finite set of numbers,
        $A$ would become infinite and ultimately uncountable with $\abs{A} \leq \abs{M}$.
    \end{proof}
\end{theorem}

% Problem 4 ----------------------------------------------
\newpage
\section{}
Does there exist a metric space, containing a sequence of nested bounded closed sets 
$F_1 \supset F_2 \supset \cdots \supset F_n \supset \cdots$
such that
\[
    \bigcap_{n \in \N} F_n = \emptyset
\]
Hint: If $d(x,y)$ is a usual Euclidean metric on $\R$, 
one can shown that $\frac{d(x,y)}{1 + d(x,y)}$ is also a metric.
Such metric is often called a bounded metric...

\begin{definition}
    The set $A$ in metric space $(X,d)$ is considered \emph{\underline{open}} if 
    \[
        \forall_{x \in A} \exists_{\epsilon > 0} \st \forall_{y \in X} d(x,y)<\epsilon
    \]
\end{definition}

\begin{definition}
    The set $A$ in metric space $(X,d)$ is considered \emph{\underline{closed}} if the set $A^c$ is open.
\end{definition}

\begin{definition}
    The set $A$ in metric space $(X,d)$ is called \emph{\underline{bounded}} if
    \[
        \forall_{x \in A} \exists_{R>0} \st \forall_{y\in A} d(x,y) < R
    \]
\end{definition}

\begin{theorem}
    There does exist a metric space $(X,d)$ 
    containing the sequence of nested bounded closed sets 
    $F_1 \supset F_2 \supset \cdots \supset F_n \supset \cdots$
    such that \[
        \bigcap_{n \in \N} F_n = \emptyset
    \]
    \begin{proof}
        Let the metric space $(X,d)$ be defined with\[
            X := \R \backslash \{0\}
        \]
        and endowed with the Euclidean metric $d : \R \cross \R$ defined by:\[
            d(x,y) := \sqrt{(x-y)^2}
        \]
        
        Let $F_1 \subset X$ be defined for:\[
            F_1 := B_{r_1}(0) = \qty{
                x \in X \st d(0,x) \leq r_1
            }
        \]
        where $r_1$ is initialized arbitrarily large.

        For $n = 2, 3, \dots$, 
        $F_n \supset \cdots \supset F_2 \subset F_1 \subset X$ is defined by:
        \[
           F_n := B_{r_n}(0) = \qty{
               x \in X \st d(0,x) \leq r_n
           }
        \]
        where $r_{n+1} = \frac{r_n}{1+r_n}$.

        Finally, the solution is very obvious that the origin is the only limit point of the intersection.\[
            \lim_{N \to \infty} \cap_{n < N} F_n = \{0\}
        \] 
        However, within this particular metric space, where $X = \R \backslash \{0\}$, this limit is not within the sets themselves.
        Therefore,
        \[
            \cap_{n \in \N} F_n = 0
        \]
    \end{proof}
\end{theorem}




% Problem 5 ----------------------------------------------
\newpage
\section{}
Show that there exists a unique continuous function, 
$f(x)$ on the interval $[0,1]$,
satisfying the equation
\[
    f(x) = \int_{0}^{1} \sin(x^2 + y^ 2) f(y) \dd y
\]

\begin{theorem}
    There exists a unique continuous function $f : [0,1] \to \R$
    that satisfies the following equation
    \begin{equation}\label{eq:pblm5}
        f(x) = \int_{0}^{1} \sin(x^2 + y^ 2) f(y) \dd y
    \end{equation}
    \begin{proof}
        I'll prove that the is only a single solution by essentially treating the integral statement as a set of systems of equations (but with an operator) and then demonstrate that the solution function is unique as a contradiction would occur otherwise.

        Let $T$ be a functional mapping of the $f(y)$ to $f(x)$, 
        $T : \mathbb{C}([0,1]) \to \mathbb{C}([0,1])$, defined as:
        \[
            T(f(y)) := \int_{0}^{1} \sin(x^2 + y^ 2) f(y) \dd{y}
        \]

        Now we take $T(f(x)) = f(x)$ and make the claim $f(x)$ is not unique.
        This would mean that $T(g(x)) = g(x)$ is another solution.
        If there are multiple solutions, then the following would be true:
        \begin{align*}
            T(f(x)) - f(x) 
                &= T(g(x)) - g(x)\\
            \int_{0}^{1} \sin(x^2 + y^2) f(y) \dd{y} - f(x)
                &= \int_{0}^{1} \sin(x^2 + y^2) g(y) \dd{y} - g(x)\\
            f(x) - g(x)
                &= \int_{0}^{1} \sin(x^2 + y^2) f(y) \dd{y}
                - \int_{0}^{1} \sin(x^2 + y^2) g(y) \dd{y}\\
            f(x) - g(x)
                &= \int_{0}^{1} \sin(x^2 + y^2) (f(y) - g(y)) \dd{y}
            \intertext{Since $\abs{\sin{x}}\leq 1 \implies \int_0^1 \sin(x) \dd{x} < 1$,}
                &< \int_{0}^{1} (f(y) - g(y)) \dd{y} \leq
            \intertext{Since we can look at the integral over the region as less then the maximum value times the width:}
                &\leq (1-0) \sup_{y \in [0,1]} (f(y) - g(y))\\
            f(x) - g(x)
                & < \sup_{y \in [0,1]} (f(y) - g(y))
        \end{align*}
        Which is not possible, leading to the claim that two solutions exist to be false.
    \end{proof}
\end{theorem}

% Problem 6 ----------------------------------------------
\newpage
\section{}
Let $V$ be a complete metric space without isolated points.
Show that $V$ is uncountable $(\abs{V} > \abs{\N})$.

\begin{definition}\label{def:complete_cauchy_limit}
    A metric space $(X,d)$ is considered \emph{\underline{Complete}} 
    if every Cauchy sequence of points in $X$ has a limit within $X$.
    \begin{itemize}
        \item A sequence $x_1, x_2, \dots$ in metric space $(X,d)$ 
        is considered \emph{\underline{Cauchy}} if
        \[
            \forall_{r > 0} \exists_N \st \forall_{m,n > N} d(x_m,x_n) < r
        \]
        \item $x$ is the \underline{\emph{limit}} of sequence ($x_n$), 
        $\lim_{n \to \infty} x_n$, if
        \[
            \forall_{\epsilon>0} \exists_{N \in \N} \st \forall_{n\geq N} \abs{x_n - x} < \epsilon
        \]
    \end{itemize}
\end{definition}

\begin{definition}\label{def:isolated_points}
    A point within metric space $(X,d)$ is considered an 
    \emph{\underline{isolated point}} of set $A \subset X$ 
    in which no other points are within the neighborhood of $x$.
    i.e.\[
        \exists_{\epsilon>0} \st \forall_{y \in X} 
            \st d(x,y) < \epsilon y \implies \not \in A
    \]
    \begin{itemize}
        \item A complete set $A$ that contains no isolated points is called 
            \emph{\underline{dense-in-itself}}.
    \end{itemize}
\end{definition}

\begin{definition}\label{def:sur/in/bijective_funcs}
    A \emph{\underline{one to one correspondance}} is also known as a 
    bijective function that maps $\N \to X$.
    \begin{itemize}
        \item A function $f : \N \to A$ is said to be \underline{\emph{surjective}} if\[
            \exists_{f : \N \toS A} \iff 
            \exists_{f : \N \to A} \st \forall_{x \in \N} \exists f(x) \in A
        \]

        \item A function $f : \N \to A$ is said to be \underline{\emph{injective}} if\[
            \exists_{f : \N \toI A} \iff 
            \exists_{f : \N \to A} \st \forall_{f(x) \in A} \exists_{x \in \N}
        \]

        \item A function $f : A \to B$ is said to be \underline{\emph{bijective}} 
        if $f$ is both surjective and injective.
        i.e.\[
            \exists_{f : A \toB B} \iff 
            \exists_{f : A \to B} \st 
                \qty(\forall{x \in A} \exists_{f(x) \in B}) \land
                \qty(\forall{y \in B} \exists_{f^{-1}(y) \in A})
        \]
    \end{itemize}
\end{definition}

\begin{definition}\label{def:cardinalityAndCountable}
    The \emph{\underline{Cardinality}} of set $A$,
        denoted as $\abs{A}$,
        is the number of unique elements contained within $A$.
    \begin{itemize}
        \item A set $A$ is considered \emph{\underline{Countable}} if $\abs{A} \leq \abs{\N}$. 
        This is also said to be true if a surjective function exists mapping $\N$ to $A$.
        
        \item Set $A$ and $B$ within metric space $(X,d)$ are said be of the same cardinality,
        $\abs{A}=\abs{B}$, 
        if there exists a bijective mapping between $A$ and $B$, $f : A \toB B$.

        \item If $A$ is an infinite set, then $A$ is \emph{\underline{Countably Infinite}},
        $\abs{A} = \aleph_0 = \abs{\N}$, 
        if there exists a one to one correspondence from $\N$ to $A$.
        
        \item For $A$ A set $A$ is considered \emph{\underline{uncountable}} if it is not countable.
        i.e. $\abs{A} > \abs{\N}$. 
        This is also said to be true if an injective function exists mapping $\N$ to $A$, 
        but that no surjective mappings exist.
    \end{itemize}
\end{definition}

\begin{theorem}
    A complete metric space, $(V,d)$, that contains no isolated point is uncountable.
    \begin{proof}
        From Definition \ref{def:complete_cauchy_limit}, 
        we have that all cauchy sequences in the complete metric space $(X,d)$ must have a limit in $X$.
        
        From Definition \ref{def:cardinalityAndCountable}, 
        it is known that within all countable sets there exists a one-to-one correspondence between $\N$ and the set $A$.
        
        For $A$ to be uncountable, an injective function mapping $\N$ to $A$, there exists, $f : \N \toI A$. 

        From Definition \ref{def:sur/in/bijective_funcs}, this means that 
        $\forall_{f(x) \in A} \exists_{x \in \N}$,
        however, since $\N$ is not a complete set, it is not possible for a one-to-one correspondence to exist.
        Therefore, the set is not countable and therefore uncountable.
    \end{proof}
\end{theorem}




\end{document}
